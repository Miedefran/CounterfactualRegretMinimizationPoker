\paragraph{Abstract}

Counterfactual Regret Minimization (CFR) ist ein etablierter Algorithmus zur Berechnung von Nash-Gleichgewichten in Spielen mit imperfekter Information, der insbesondere in der Pokerspielforschung eingesetzt wird.
Diese Arbeit untersucht die Leistungsfähigkeit verschiedener CFR-Varianten und Implementierungsansätze bei steigender Spielkomplexität.

Es wurden die Algorithmen Vanilla CFR, CFR+ und Discounted CFR (DCFR) in Python implementiert und auf vier Pokervarianten unterschiedlicher Komplexität angewendet: Kuhn Poker, Leduc Hold'em, Twelve Card Poker und Small Island Hold'em.
Die Implementierungen umfassen einen dynamischen Ansatz, einen rekursiven Tree-basierten Ansatz sowie eine Layer-basierte Architektur.
Zur Bewertung der Strategiequalität wurde die Exploitability mittels eines Best-Response-Agents auf Basis eines Public-State-Trees berechnet.

Die Ergebnisse zeigen, dass DCFR und CFR+ deutlich schneller konvergieren als Vanilla CFR, wobei DCFR generell besser abschneidet als CFR+, jedoch die Performance abhängig vom Spieltyp ist.
Bei kleinen Spielen wie Kuhn Poker sind die Unterschiede zwischen den Implementierungsansätzen vernachlässigbar, während bei größer werdenden Spielen die Unterschiede signifikant werden.
Bei Small Island Hold'em erweist sich Vanilla CFR mit der vorliegenden Implementierung als unpraktikabel, da selbst nach 45 Stunden Training keine hinreichend niedrige Exploitability erreicht werden konnte.

Die Arbeit demonstriert die Notwendigkeit algorithmischer und implementierungstechnischer Optimierungen für die Skalierung von CFR auf komplexere Pokervarianten und identifiziert die Problemstellen.

