\section{Erkenntnisse und Reflexion}

Dieser Abschnitt präsentiert die während des Entwicklungsvorgangs gewonnenen Erkenntnisse und reflektiert mögliche Verbesserungen.

Im Projektverlauf hat sich gezeigt, dass eine funktionsfähige Evaluation das wichtigste Werkzeug zur Sicherstellung der Korrektheit der Algorithmen ist.
Ohne eine verlässliche Exploitability-Berechnung können Implementierungsfehler lange unentdeckt bleiben.
Darüber hinaus hat sich gezeigt, dass Architekturentscheidungen, die zunächst wie reine Implementierungsdetails wirken, schwerwiegende Folgen haben können.
Das beste Beispiel hierfür ist die Modellierung der Zufallsknoten.
Solche Problemstellen sollten möglichst schnell behoben werden, da der Aufwand für Workaround-Code schnell wächst.

Im Verlauf der Arbeit wurden viele Erkenntnisse in Bezug auf die Strukturierung einer wissenschaftlichen Arbeitsweise gewonnen.
Zunächst hätte die Literaturrecherche breiter angelegt werden sollen, statt sich früh auf wenige Standardvarianten wie CFR, CFR+ und MCCFR zu versteifen.
Dadurch hätte früher erkannt werden können, dass es durchaus noch andere interessante CFR-Varianten wie DCFR gibt.
Ein guter Einstieg wäre der Kurs \emph{Computational Game Solving} gewesen \cite{sandholmAnagnostidesCS15888F25}.
Die Folien sind didaktisch deutlich besser zu verarbeiten als die zugrunde liegenden Paper und hätten in den Anfangsphasen einen breiten Überblick über das Thema geboten.

Die größte Schwierigkeit war das Setzen des Rahmens der Arbeit.
Dem lag die Arbeitsreihenfolge von Recherche, Implementierung und Dokumentation zugrunde.\\
Nach einer ersten Literaturrecherche wurden die Spielumgebung und die ersten Solver implementiert.
Als der Best Response Agent nicht funktionierte, wurde sich der Dokumentation gewidmet.
Dies war jedoch ein Fehler.
Große Teile der zu diesem Zeitpunkt verfassten Dokumentation erwiesen sich später als nicht mehr brauchbar.
Außerdem wurde der Best Response Agent erst in der Mitte der Bearbeitungszeit erfolgreich umgesetzt, sodass Fehler erst spät festgestellt wurden und Optimierungen warten mussten.\\
Gegen Ende der Arbeit wurden fortlaufend neue interessante Bereiche entdeckt, die man hätte untersuchen können. 
Jedoch reichte die Zeit nicht mehr aus, diese erfolgreich umzusetzen und anschließend zu dokumentieren.
Mit einer besseren zeitlichen Planung hätte ein breiteres und besser ausgewähltes Spektrum an Themen rund um Counterfactual Regret Minimization gewählt werden können.
Außerdem hätte die Entdeckung eines Kurses wie \emph{Computational Game Solving} zu einem frühen Zeitpunkt massiv beim Setzen des Rahmens geholfen.\\
Wenn diese Arbeit erneut geschrieben würde, wäre ein sinnvoller Ablauf: intensive Literaturrecherche, Rahmen festlegen, Implementierung, anschließend Dokumentation.
Die Vermischung der Phasen hat nur Probleme bereitet.

