\subsection{Definition und Konzepte}
Die Spieltheorie beschäftigt sich mit der Analyse strategischer Interaktionen zwischen Entscheidungsträgern und bietet hierfür eine Vielzahl analytischer Methoden.
Diese basieren auf zwei zentralen Annahmen~\cite[Kap.~1.1]{osborne1994course}.


Die erste Annahme ist, dass Entscheidungsträger rational handeln.
Ein rationaler Entscheidungsträger kennt seine Handlungsmöglichkeiten, kann Erwartungen über unbekannte Situationen bilden und hat klare Vorstellungen darüber, welche Ergebnisse er bevorzugt.
Er wählt seine Aktion bewusst nach einem Prozess der Optimierung.
Liegt keine Unsicherheit vor, wählt er die Aktion, die zu den für ihn besten Konsequenzen führt.
Diese Präferenzen lassen sich durch eine Nutzenfunktion beschreiben, die jedem möglichen Ergebnis einen Wert zuordnet.
Die beste Aktion ist die, die den höchsten Nutzen erzielt.
Sind die Konsequenzen von Aktionen unsicher, maximiert ein rationaler Entscheidungsträger den erwarteten Nutzen.
Hierfür muss er in der Lage sein, Wahrscheinlichkeiten für unsichere Ereignisse abzuschätzen~\cite[Kap.~1.4]{osborne1994course}.

Die zweite Annahme ist, dass Entscheidungsträger ihr Wissen oder ihre Erwartungen über das Verhalten anderer Spieler bei ihrer Entscheidungsfindung berücksichtigen. 
Dies nennt man auch strategisches Handeln.
Spieltheoretisches Denken berücksichtigt, dass jeder Entscheidungsträger vor seiner Entscheidung versucht, Informationen über das Verhalten der anderen Spieler zu erhalten.
Im Gegensatz dazu nimmt die Theorie des Wettbewerbsgleichgewichts an, dass jeder Akteur sich nur für Umgebungsparameter wie Preise interessiert, auch wenn diese durch die Aktionen aller Akteure bestimmt werden~\cite[Kap.~1.3]{osborne1994course}.\\
Ein anschauliches Beispiel hierfür ist das Gefangenendilemma. 
Zwei Verdächtige werden getrennt verhört und müssen jeweils entscheiden, ob sie schweigen oder den anderen belasten. 
Schweigen beide, erhalten beide zwei Jahre Haft.
Wenn beide gestehen, erhalten sie fünf Jahre Haft.
Gesteht jedoch nur einer, wird dieser freigelassen, während der andere zu zehn Jahren Haft verurteilt wird.
Die optimale Entscheidung ist abhängig von den Tendenzen anderer Entscheidungsträger.
Strategisches Handeln bedeutet, solche Erwartungen über das Verhalten anderer Spieler in die Entscheidungsfindung mit einzubeziehen~\cite[Kap.~1.1]{osborne1994course}.

Spieltheoretische Modelle befinden sich auf einer hohen Abstraktionsebene und können daher verwendet werden, um ein breites Spektrum an Phänomenen zu betrachten. 
Zur formalen Darstellung dieser Modelle werden mathematische Definitionen genutzt. 
Die zugrunde liegenden Konzepte sind jedoch nicht primär mathematischer Natur und lassen sich auch ohne formale Notation erläutern.\\
Die Spieltheorie findet Anwendung in unzähligen Bereichen.
Zu den klassischen Anwendungsgebieten zählen die Analyse wirtschaftlicher und politischer Wettbewerbssituationen. 
Darüber hinaus wird sie auch in der Evolutionsbiologie, der Soziologie und anderen Disziplinen genutzt, um strategische Interaktionen zu modellieren\\~\cite[Kap.~1.1]{osborne1994course}.

Ein Spiel beschreibt eine strategische Interaktion zwischen Spielern mit festgelegten Rahmenbedingungen. 
Es definiert, welche Aktionen den Spielern zur Verfügung stehen und welche Ergebnisse in ihrem Interesse sind.
Das Spiel legt damit die Struktur und die Regeln der Interaktion fest, bestimmt aber nicht, wie sich die Spieler tatsächlich verhalten.\\
Die Frage, welche Aktionen die Spieler tatsächlich ausführen, wird nicht durch das Spiel selbst beantwortet, sondern durch die Lösung des Spiels.
Eine Lösung ist ein Konzept, das systematisch beschreibt, welche Ergebnisse in einer Klasse von Spielen unter bestimmten Verhaltensannahmen zu erwarten sind.
Sie liefert damit eine Vorhersage über das tatsächliche Verhalten der Spieler unter den getroffenen Annahmen über Rationalität und strategisches Denken~\cite[Kap.~1.2]{osborne1994course}.\\
Spiele können anhand verschiedener Kriterien systematisch kategorisiert werden.
Grundlegend für alle spieltheoretischen Modelle ist das Konzept des Spielers. 
Ein Spieler kann dabei sowohl eine einzelne Person als auch eine Gruppe von Personen darstellen, die gemeinsam eine Entscheidung treffen.

Die erste grundlegende Unterscheidung betrifft kooperative und nicht-kooperative Spiele.
Nicht-kooperative Spiele modellieren die Aktionen einzelner Spieler. 
Kooperative Spiele hingegen konzentrieren sich auf Gruppen von Spielern.
Dabei wird nicht betrachtet, wie diese Gruppen intern funktionieren~\cite[Kap.~1.2]{osborne1994course}.

Eine weitere wichtige Kategorisierung betrifft die Darstellungsform des Spiels. 
Hierbei werden strategische Spiele, auch als Spiele in Normalform bezeichnet, von extensiven Spielen unterschieden. 
Bei strategischen Spielen wählen die Spieler ihre Strategie zu Beginn des Spiels und passen diese während des Spielverlaufs nicht mehr an. 
Die Entscheidungen werden zudem simultan getroffen, sodass keine Informationen über die Aktionen der Gegenspieler verfügbar sind. 
Extensive Spiele hingegen spezifizieren die möglichen Reihenfolgen von Ereignissen und erlauben es jedem Spieler, seinen Aktionsplan nicht nur zu Beginn, sondern auch bei jeder weiteren Entscheidung zu überdenken~\cite[Kap.~1.2]{osborne1994course}.

Die letzte hier zu nennende Unterscheidung betrifft den Informationsstand der Spieler. 
Bei Spielen mit perfekter Information sind alle Teilnehmer vollständig über die Züge der anderen Spieler informiert. 
Bei Spielen mit imperfekter Information hingegen können die Spieler unvollständig über die Aktionen der anderen Spieler oder über Ausgänge von Zufallsereignissen informiert sein~\cite[Kap.~1.2]{osborne1994course}.

Ein \emph{strikt kompetitives Spiel} (auch \emph{Nullsummenspiel}) liegt vor, wenn der Gewinn des einen Spielers genau dem Verlust des anderen entspricht.
Das bedeutet, dass für alle möglichen Spielausgänge die Summe der Nutzenwerte beider Spieler null ergibt.
Formal: Ist die Nutzenfunktion von Spieler 1 durch $u_1$ gegeben, so gilt für die Nutzenfunktion $u_2$ von Spieler 2 stets $u_1 + u_2 = 0$~\cite[Def.~21.1]{osborne1994course}.
