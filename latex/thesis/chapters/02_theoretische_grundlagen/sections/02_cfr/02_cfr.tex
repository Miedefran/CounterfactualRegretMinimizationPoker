\section{Counterfactual Regret Minimization}
Dieser Abschnitt stellt die theoretischen Grundlagen von Counterfactual Regret Minimization (CFR), dem Kernstück dieser Arbeit, vor.
Als fachliche Grundlage dient das Paper \glqq Regret Minimization in Games with Incomplete Information\grqq ~\cite{zinkevich2007regret}, in welchem CFR erstmals vorgestellt wurde.\\
Zuerst werden einige wichtige Werte und Zusammenhänge definiert, die essenziell für das Verständnis der darauffolgenden Konzepte sind.
Anschließend wird erklärt was Regret-Minimierung ist.
Darauf aufbauend wird das von Zinkevich et al.\ eingeführte Konzept des Counterfactual Regret sowie Regret Matching erläutert.
Abschließend fügen sich die einzelnen Komponenten zu einem Algorithmus zusammen.

\subsection{Definitionen}
\subsubsection{Wahrscheinlichkeiten und Werte}

Für ein Strategieprofil $\sigma$ werden folgende Wahrscheinlichkeiten und Werte definiert:

Die Wahrscheinlichkeit, dass eine Historie $h$ auftritt, wenn alle Spieler gemäß $\sigma$ handeln, wird mit $\pi^\sigma(h)$ bezeichnet.
Diese Wahrscheinlichkeit lässt sich in ein Produkt zerlegen:
\begin{equation}
\pi^\sigma(h) = \prod_{i \in N \cup \{c\}} \pi_i^\sigma(h),
\end{equation}
wobei $\pi_i^\sigma(h)$ die Wahrscheinlichkeit bezeichnet, dass Spieler $i$ in all seinen Entscheidungsknoten auf dem Pfad zu $h$ gemäß $\sigma$ die Aktionen gewählt hat, die zu $h$ führen.
Das Produkt der Wahrscheinlichkeiten aller anderen Spieler (inklusive Zufallsereignisse) wird mit $\pi_{-i}^\sigma(h)$ bezeichnet.

Für ein Informationset $I$ wird die Erreichbarkeitswahrscheinlichkeit definiert als:
\begin{equation}
\pi^\sigma(I) = \sum_{h \in I} \pi^\sigma(h).
\end{equation}
Analog sind $\pi_i^\sigma(I)$ und $\pi_{-i}^\sigma(I)$ als die Wahrscheinlichkeiten definiert, dass Spieler $i$ beziehungsweise die anderen Spieler das Informationset $I$ erreichen.

Der erwartete Nutzen für Spieler $i$ unter Strategieprofil $\sigma$, auch \emph{Overall Value} genannt, wird wie folgt berechnet:
\begin{equation}
u_i(\sigma) = \sum_{h \in Z} u_i(h) \pi^\sigma(h),
\end{equation}


\subsubsection{Nash-Gleichgewicht}
Ein Nash‑Gleichgewicht beschreibt ein Strategienprofil, bei dem kein Spieler durch eine einseitige Änderung seiner Strategie seinen erwarteten Nutzen erhöhen kann\\~\cite[Ch.~11, Sec.~11.5]{osborne1994course}.\\
Ein Nash-Gleichgewicht für ein Zweispielernullsummenspiel ist ein Strategieprofil $\sigma = (\sigma_1, \sigma_2)$, das folgende Ungleichungen erfüllt:
\begin{equation}
u_1(\sigma) \ge \max_{\sigma_1' \in \Sigma_1} u_1(\sigma_1', \sigma_2),
\qquad
u_2(\sigma) \ge \max_{\sigma_2' \in \Sigma_2} u_2(\sigma_1, \sigma_2')
\end{equation}
(\cite[Eq.~1]{zinkevich2007regret}).

Eine Approximation eines Nash-Gleichgewichts, auch $\varepsilon$-Nash-Gleichgewicht genannt, ist ein Strategieprofil $\sigma$, das folgende Ungleichungen erfüllt:
\begin{equation}
u_1(\sigma) + \varepsilon \ge \max_{\sigma_1' \in \Sigma_1} u_1(\sigma_1', \sigma_2),
\qquad
u_2(\sigma) + \varepsilon \ge \max_{\sigma_2' \in \Sigma_2} u_2(\sigma_1, \sigma_2')
\end{equation}
(\cite[Eq.~2]{zinkevich2007regret}).

\subsection{Regret-Minimierung}
Regret misst den Unterschied zwischen dem maximal erreichbaren Nutzen unter Verwendung der zu diesem Zeitpunkt bestmöglichen Strategie und dem tatsächlich erzielten Nutzen.\\
Um Regret-Minimierung zu definieren, wird das wiederholte Spielen eines extensiven Spiels betrachtet.
Sei $\sigma_i^t$ die von Spieler $i$ in Runde $t$ verwendete Strategie.
Der durchschnittliche \emph{Overall Regret} von Spieler $i$ zum Zeitpunkt $T$ ist:
\begin{equation}
R_i^T = \frac{1}{T}\max_{\sigma_i^\ast\in\Sigma_i}\sum_{t=1}^T \bigl(u_i(\sigma_i^\ast,\sigma_{-i}^t) - u_i(\sigma^t)\bigr).
\end{equation}
(\cite[Eq.~3]{zinkevich2007regret})

Die Durchschnittsstrategie $\bar\sigma_i^T$ für Spieler $i$ von Zeit $1$ bis $T$ wird für jedes Informationset $I \in \mathcal{I}_i$ und jede Aktion $a \in A(I)$ definiert als:
\begin{equation}
\label{eq:cfr-durchschnittsstrategie}
\bar\sigma_i^T(I)(a) = \frac{\sum_{t=1}^T \pi_i^{\sigma^t}(I)\,\sigma_i^t(I)(a)}{\sum_{t=1}^T \pi_i^{\sigma^t}(I)}.
\end{equation}
(\cite[Eq.~4]{zinkevich2007regret})

Die Durchschnittsstrategie gewichtet Aktionen proportional zur Erreichbarkeitswahrscheinlichkeit $\pi_i^{\sigma^t}(I)$ des Informationsets $I$ in Runde $t$.

Es besteht ein wichtiger Zusammenhang zwischen Regret und Nash-Gleichgewichten.
\label{thm:cfr-nash}
In einem Nullsummenspiel gilt: Ist der durchschnittliche Overall Regret beider Spieler zum Zeitpunkt $T$ kleiner als $\varepsilon$, dann ist $\bar\sigma^T$ ein $2\varepsilon$-Nash-Gleichgewicht (\cite[Thm.~2]{zinkevich2007regret}).

Ein Algorithmus zur Bestimmung von $\sigma_i^t$ ist Regret-minimierend, wenn der durchschnittliche Overall Regret von Spieler $i$ unabhängig von der Strategiefolge $\sigma_{-i}^t$ des Gegners gegen Null konvergiert, während $T$ gegen Unendlich läuft.
Daher können Regret-minimierende Algorithmen im Self-Play verwendet werden, um ein approximatives Nash-Gleichgewicht zu berechnen.


\subsection{Counterfactual Regret}
Die zentrale Idee, die Zinkevich et al. in ihrem Paper "Regret Minimization in Games with Incomplete Information" präsentieren, ist, den gesamten Regret-Wert in eine Menge aus additiven Termen zu zerlegen.
Diese Terme können dann unabhängig voneinander minimiert werden.
Insbesondere wird ein völlig neues Regret-Konzept mit dem Namen \emph{Counterfactual Regret} eingeführt, das für jedes Informationset definiert ist.

Um das Konzept zu erläutern, wird ein Informationset $I \in \mathcal{I}_i$ betrachtet.\\
Die \emph{Counterfactual Utility} oder auch \emph{Counterfactual Value} $u_i(\sigma, I)$ ist der erwartete Nutzen für Spieler $i$ unter der Annahme, dass das Informationset $I$ erreicht wird.
Dabei spielen alle Spieler gemäß dem Strategieprofil $\sigma$, wobei Spieler $i$ so handelt, dass $I$ tatsächlich erreicht wird.
Mit $\pi^\sigma(h, h')$ als Wahrscheinlichkeit, von $h$ nach $h'$ zu gelangen, gilt:
\begin{equation}
u_i(\sigma, I) = \frac{\sum_{h \in I,\, h' \in Z} \pi_{-i}^\sigma(h)\,\pi^\sigma(h, h')\,u_i(h')}{\pi_{-i}^\sigma(I)}.
\end{equation}
(\cite[Eq.~5]{zinkevich2007regret})

Für alle $a \in A(I)$ ist $\sigma|_{I \to a}$ ein Strategieprofil, das identisch zu $\sigma$ ist, außer dass Spieler $i$ immer die Aktion $a$ ausführt, wenn er das Informationset $I$ erreicht.
Der \emph{Immediate Counterfactual Regret} ist
\begin{equation}
R_{i,\mathrm{imm}}^{T}(I) = \frac{1}{T}\max_{a \in A(I)} \sum_{t=1}^{T} \pi_{-i}^{\sigma^t}(I)\,\bigl(u_i(\sigma^t|_{I \to a}, I) - u_i(\sigma^t, I)\bigr).
\end{equation}
(\cite[Eq.~6]{zinkevich2007regret})

Dieser Term misst den Regret im Informationset $I$, gewichtet mit der \emph{Counterfactual Probability}, dass $I$ erreicht werden würde, wenn Spieler $i$ es erzwingen wollte.
Die positiven Immediate Counterfactual Regret-Werte werden definiert als:
\begin{equation}
R_{i,\mathrm{imm}}^{T,+}(I) = \max\!\bigl(R_{i,\mathrm{imm}}^{T}(I), 0\bigr).
\end{equation}

Theorem 3 beschreibt die erste große Erkenntnis des Papers. 
\begin{equation}
    R_i^T \le \sum_{I \in \mathcal{I}_i} R_{i,\mathrm{imm}}^{T,+}(I)
\end{equation}
(\cite[Thm.~3]{zinkevich2007regret})\\
Das heißt, wenn der Immediate Counterfactual Regret minimiert wird, dann wird auch der Average Overall Regret minimiert.

\subsection{Regret Matching}
\label{sec:regret-matching}

Mit folgender Formel wird der Counterfactual Regret für eine konkrete Aktion $a$ in einem Informationset $I$ berechnet.
\begin{equation}
    R_i^{T}(I, a) = \frac{1}{T} \sum_{t=1}^{T} \pi_{-i}^{\sigma^{t}}(I)
    \left( u_i(\sigma^{t}|_{I \to a}, I) - u_i(\sigma^{t}, I) \right)
\end{equation}
(\cite[Eq.~7]{zinkevich2007regret})

Dabei sei $R_i^{T,+}(I, a) = \max(R_i^{T}(I, a), 0)$ als positiver Anteil des Immediate Counterfactual Regret für Aktion $a$ definiert.

Die Strategie für Iteration $T+1$ wird dann durch Regret Matching bestimmt:
\begin{equation}
\label{eq:cfr-regret-matching}
\sigma_i^{T+1}(I)(a) = \begin{cases}
\frac{R_i^{T,+}(I, a)}{\sum_{a' \in A(I)} R_i^{T,+}(I, a')} & \text{falls } \sum_{a' \in A(I)} R_i^{T,+}(I, a') > 0 \\
\frac{1}{|A(I)|} & \text{in jedem anderen Fall.}
\end{cases}
\end{equation}
(\cite[Eq.~8]{zinkevich2007regret})

Diese Formel weist Aktionen Wahrscheinlichkeiten proportional zu ihrem positiven Regret aus dem Nichtspielen dieser Aktion zu.
Gibt es keine positiven Regretwerte, werden alle Aktionen gleichverteilt gewählt.

Theorem 4 zeigt, dass bei Verwendung von \hyperref[eq:cfr-regret-matching]{Formel~\ref*{eq:cfr-regret-matching}} der Immediate Counterfactual Regret und der Overall Regret folgendermaßen beschränkt sind:
\begin{equation}
R_{i,\mathrm{imm}}^{T}(I) \le \frac{\Delta_{u,i}\sqrt{|A_i|}}{\sqrt{T}},
\end{equation}
\begin{equation}
R_i^T \le \frac{\Delta_{u,i}|\mathcal{I}_i|\sqrt{|A_i|}}{\sqrt{T}},
\end{equation}
wobei $\Delta_{u,i} = \max_z u_i(z) - \min_z u_i(z)$ der Nutzenbereich und $|A_i| = \max_{h:P(h)=i} |A(h)|$ die maximale Anzahl verfügbarer Aktionen für Spieler $i$ ist (\cite[Thm.~4]{zinkevich2007regret}).
Die Schranke für den Average Overall Regret ist linear in der Anzahl der Informationsets $|\mathcal{I}_i|$ und fällt asymptotisch mit $O(1/\sqrt{T})$.
Zusammen mit \hyperref[thm:cfr-nash]{Theorem~2} folgt, dass die Durchschnittsstrategie $\bar\sigma^T$ im Selbstspiel gegen ein Nash-Gleichgewicht konvergiert (\cite[Thm.~2]{zinkevich2007regret}).

\subsection{CFR-Algorithmus}

Das CFR-Verfahren verbindet die Komponenten aus den vorherigen Abschnitten zu einer wiederholten Selbstspielprozedur.
In jeder Iteration wird die Strategie gemäß \hyperref[eq:cfr-regret-matching]{Formel~\ref*{eq:cfr-regret-matching}} aktualisiert.
Für jedes Informationset $I$ und jede Aktion $a$ werden die kumulierten Regretwerte $R_i^t(I,a)$ gespeichert und nach jeder Iteration aktualisiert.
Parallel dazu wird die Durchschnittsstrategie gemäß \hyperref[eq:cfr-durchschnittsstrategie]{Formel~\ref*{eq:cfr-durchschnittsstrategie}} aktualisiert.
Nach $T$ Iterationen gibt der Algorithmus $(\bar\sigma_1^T, \bar\sigma_2^T)$ als approximatives Nash-Gleichgewicht zurück.


In diesem Abschnitt wurde die Basisversion des Counterfactual Regret Minimization Algorithmus vorgestellt.
In den kommenden Abschnitten folgen die optimierten Varianten, beginnend mit MCCFR.