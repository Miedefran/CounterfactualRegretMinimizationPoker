\subsection{Konventionelle Best Response Berechnung}

Die konventionelle Berechnung einer Best Response erfolgt durch eine rekursive Traversierung des Informationset Trees des betrachteten Spielers.
Der Algorithmus traversiert den Baum von der Wurzel zu den Terminalknoten und berechnet dabei für jedes Informationset den erwarteten Nutzen.

An Terminalknoten muss der Algorithmus alle möglichen Spielzustände berücksichtigen, die für den Spieler nicht unterscheidbar sind.
Da der Spieler nicht weiß, in welchem konkreten Spielzustand er sich befindet, berechnet er die Erreichbarkeitswahrscheinlichkeiten des Gegners für die verschiedenen Informationsets.
Der Nutzen wird als gewichtete Summe über alle möglichen Spielzustände berechnet, wobei jeder Nutzen mit der entsprechenden Erreichbarkeitswahrscheinlichkeit gewichtet wird.
Dieser Wert wird zurückgegeben.

Während der Rücktraversierung durch den Baum werden an den verschiedenen Knotentypen unterschiedliche Operationen durchgeführt.
An Entscheidungsknoten des betrachteten Spielers wählt der Algorithmus die Aktion mit dem höchsten erwarteten Nutzen und speichert diese Wahl als Teil der Best-Response-Strategie.
Der Wert dieser gewählten Aktion wird zurückgegeben.
An Entscheidungsknoten des Gegners und an Zufallsknoten wird der erwartete Nutzen als Summe der Werte der Kindknoten berechnet und zurückgegeben.
Wenn der Algorithmus zur Wurzel zurückkehrt, ist der zurückgegebene Wert der Best-Response-Wert gegen die gegebene Gegnerstrategie.

\begin{figure}[htbp]
    \centering
    \includegraphics[width=0.8\textwidth]{chapters/02_theoretische_grundlagen/sections/05_exploitability/subsections/03_public_state_tree/gametree.png}
    \caption{Game Tree für Kuhn Poker}
    \label{fig:br_game_tree}
\end{figure}

\begin{figure}[htbp]
    \centering
    \includegraphics[width=0.8\textwidth]{chapters/02_theoretische_grundlagen/sections/05_exploitability/subsections/03_public_state_tree/informationset_tree.png}
    \caption{Information Set Trees für beide Spieler in  Kuhn Poker}
    \label{fig:br_infoset_tree}
\end{figure}

Die Abbildungen zeigen das Beispiel Kuhn Poker zur Illustration des Algorithmus.
Im Game Tree (Abbildung~\ref{fig:br_game_tree}) repräsentieren die Knoten \emph{A,X}, \emph{A,Y}, \emph{B,X} und \emph{B,Y} die terminalen Spielzustände, wobei \emph{A} und \emph{B} die Aktionen von Spieler 1 und \emph{X} und \emph{Y} die Aktionen von Spieler 2 bezeichnen.
Im Information Set Tree von Spieler 2 (Abbildung~\ref{fig:br_infoset_tree}) kann dieser am Terminalknoten \emph{X} nicht unterscheiden, ob er sich im Game Tree Zustand \emph{A,X} oder \emph{B,X} befindet, da diese nur durch die private Information von Spieler 1 unterschieden werden.
Daher muss der Algorithmus die Erreichbarkeitswahrscheinlichkeiten für beide möglichen Zustände berechnen.

~\cite{johanson2011accelerating} beschreiben eine beschleunigte Version des Algorithmus, die vier Optimierungen umfasst und im Folgenden erläutert wird.
