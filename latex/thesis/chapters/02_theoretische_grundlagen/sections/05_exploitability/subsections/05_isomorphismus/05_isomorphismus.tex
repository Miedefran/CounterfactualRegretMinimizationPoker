\subsection{Spielspezifische Isomorphismen}

Ein weiterer Beschleunigungsschritt nutzt strategisch äquivalente Aktionen oder Ereignisse.
In vielen Kartenspielen sind Karten mit gleichem Rang, aber unterschiedlicher Farbe strategisch äquivalent.
In Poker ist das zumindest solange der Fall, bis weitere Karten aufgedeckt werden.
Bevor der Flop ausgeteilt wird, ist eine Herz zwei genauso gut wie eine Kreuz zwei.
Solche Mengen äquivalenter Zufallsereignisse werden als \emph{isomorph} bezeichnet, und eine wird willkürlich als Repräsentant gewählt.

Wenn die zu evaluierende Strategie für jedes Mitglied jeder Menge isomorpher Historien gleich ist, kann die Größe des Public State Trees erheblich reduziert werden, indem nur die Repräsentanten betrachtet werden.
Beim Zurückkehren durch einen Zufallsknoten während der Traversierung muss der Nutzen eines Repräsentanten mit der Anzahl der isomorphen States gewichtet werden, die er repräsentiert.
In Texas Hold'em führt diese Reduktion zu einem Public-State-Tree, der 21,5-mal kleiner ist als das vollständige Spiel.
