\subsection{Effiziente Evaluation der Terminalknoten}

Ein weiterer Beschleunigungsschritt betrifft die Evaluierung von Terminalknoten.
Eine naive Methode hätte $O(n^2)$ Komplexität, wobei $n$ die Anzahl der Informationsets pro Spieler ist.
In Pokerspielen kann diese auf $O(n)$ reduziert werden, indem die Informationsets nach Handstärke sortiert werden.
Da der Nutzen nur von der relativen Ordnung der Handstärken abhängt, können Indizes in die sortierte Liste des Gegners verwendet werden, die markieren, wo sich die Relation ändert (schwächer, gleich, stärker).
Um ein Informationset zu evaluieren, benötigt man nur die Gesamtwahrscheinlichkeit der Gegner-Informationsets in diesen drei Abschnitten.
Beim Übergang zum nächsten stärkeren Informationset werden die Indizes nur um einen Schritt verschoben, anstatt alle Gegner-Informationsets erneut zu betrachten.

Da die möglichen Hände der Spieler abhängig voneinander sind (eine Karte kann nicht von beiden Spielern gehalten werden), wird das Inklusions-Exklusions-Prinzip verwendet.
Beim Berechnen der Wahrscheinlichkeit von besseren oder schlechteren Gegnerhänden müssen unmögliche Hände ausgeschlossen werden, da der Gegner keine der eigenen Karten halten kann.
Hände, die eine der eigenen Karten enthalten, werden subtrahiert.
Hände, die beide eigene Karten enthalten, werden wieder addiert, da sie sonst doppelt subtrahiert würden (einmal für jede Karte).
