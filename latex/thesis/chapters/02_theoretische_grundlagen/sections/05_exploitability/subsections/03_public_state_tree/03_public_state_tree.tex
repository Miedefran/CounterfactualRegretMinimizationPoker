\subsection{Public State Tree}

Der zentrale Ansatz der beschleunigten Best-Response-Berechnung besteht darin, statt des Informationset Trees einen \emph{Public State Tree} zu traversieren.
Ein \emph{Public State} ist definiert als eine Partition der Historien, die folgende Eigenschaften erfüllt:
Keine zwei Historien aus demselben Informationset befinden sich in unterschiedlichen Public States.
Zwei Historien aus unterschiedlichen Public States haben keine Nachfahren im selben Public State, wodurch eine Baumstruktur entsteht.
Kein Public State enthält sowohl terminale als auch nicht-terminale Historien.

Informell ist ein Public State durch alle Informationen definiert, die beiden Spielern bekannt sind, also wie das Spiel für einen Beobachter ohne private Informationen aussieht.
Der Public State Tree ist deutlich kleiner als der Informationset Tree, da er nur die öffentlich sichtbaren Informationen enthält.

Der entscheidende Vorteil bei der Traversierung des Public State Trees liegt in der Wiederverwendung von Erreichbarkeitswahrscheinlichkeiten.
In der konventionellen Methode werden die Erreichbarkeitswahrscheinlichkeiten des Gegners an jedem Terminalknoten neu berechnet.
Im Public State Tree werden diese Wahrscheinlichkeiten jedoch einmal berechnet und für alle Informationsets des betrachteten Spielers in diesem Public State wiederverwendet.
Dies ist möglich, weil die Erreichbarkeitswahrscheinlichkeiten für alle eigenen Informationsets identisch sind, die sich im selben Public State befinden.

Statt wie in der konventionellen Methode einen einzelnen Wert für ein Informationset zurückzugeben, gibt der Algorithmus beim Public State Tree einen Vektor von Werten zurück.
Einen für jedes Informationset des betrachteten Spielers im Public State.
An Terminalknoten werden die Werte für alle Informationsets berechnet, wobei die Erreichbarkeitswahrscheinlichkeiten des Gegners einmal berechnet und für alle eigenen Informationsets verwendet werden.
An Entscheidungsknoten des betrachteten Spielers wird für jedes Informationset die beste Aktion gewählt, und es wird ein Vektor zurückgegeben, dessen Einträge die maximalen Aktionswerte darstellen.
An Entscheidungsknoten des Gegners und an Zufallsknoten werden die Wertevektoren der Kindknoten summiert und zurückgegeben.
Diese Umstrukturierung führt zu denselben Berechnungen wie die konventionelle Methode, ändert jedoch die Reihenfolge, sodass Strategieabfragen beim Gegner effizienter wiederverwendet werden können.
In Spielen wie Texas Hold'em, in denen jeder Spieler bis zu 1326 Informationsets pro Public State haben kann, ermöglicht dies eine erhebliche Reduzierung der Strategieabfragen.

\begin{figure}[htbp]
    \centering
    \includegraphics[width=0.35\textwidth]{chapters/02_theoretische_grundlagen/sections/05_exploitability/subsections/03_public_state_tree/public_tree.png}
    \caption{Public State Tree für Kuhn Poker}
    \label{fig:public_state_tree}
\end{figure}

Abbildung~\ref{fig:public_state_tree} zeigt den Public State Tree für das Beispiel Kuhn Poker.
Im Vergleich zum Game Tree (Abbildung~\ref{fig:br_game_tree}) und den Information Set Trees (Abbildung~\ref{fig:br_infoset_tree}) ist der Public State Tree deutlich kompakter, da er nur die öffentlich sichtbaren Informationen enthält.
Der Terminalknoten \emph{A,B,X,Y} repräsentiert alle vier möglichen terminalen Spielzustände \emph{A,X}, \emph{A,Y}, \emph{B,X} und \emph{B,Y} aus dem Game Tree.
An diesem Terminalknoten können die Erreichbarkeitswahrscheinlichkeiten des Gegners einmal berechnet und für alle Informationsets beider Spieler in diesem Public State wiederverwendet werden.
