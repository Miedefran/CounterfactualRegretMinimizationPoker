\section{Exploitability}\label{sec:exploitability}
Zur Evaluation der Strategiequalität wird die Metrik Exploitability verwendet.
Als theoretische Grundlage für dieses Kapitel dient \cite{johanson2011accelerating}.
Zur Berechnung der Exploitability wird die Best-Response-Methode genutzt.
Eine Best Response ist die optimale Strategie eines Spielers gegen eine gegebene Gegnerstrategie.
Formal ist die Best Response von Spieler $i$ gegen die Gegnerstrategie $\sigma_{-i}$ definiert als:
\begin{equation}
b_i(\sigma_{-i}) = \arg\max_{\sigma'_i \in \Sigma_i} u_i(\sigma'_i, \sigma_{-i}),
\end{equation}
wobei $\Sigma_i$ die Menge aller möglichen Strategien von Spieler $i$ bezeichnet.\\
Der Wert $u_i(\sigma_i, b_{-i}(\sigma_i))$ ist eine untere Schranke für den erwarteten Nutzen von Spieler $i$.
Die Exploitability einer Strategie misst, wie viel Nutzen gegen einen Gegner verloren geht, der die Best-Response-Strategie spielt, verglichen mit dem Wert, der von einer optimalen Strategie erreicht werden würde.


In einem Zweipersonen-Nullsummenspiel ist die Exploitability einer Strategie $\sigma_i$ definiert als:
\begin{equation}
\varepsilon_i(\sigma_i) = v_i - u_i(\sigma_i, b_{-i}(\sigma_i)),
\end{equation}
wobei $v_i$ der Spielwert für Spieler $i$ ist (die untere Schranke für den Nutzen eines optimalen Spielers in Position $i$) und $b_{-i}(\sigma_i)$ die Best Response des Gegners gegen $\sigma_i$ darstellt.
Eine Strategie ist optimal, wenn ihre Exploitability null ist.

Für Strategien, die in beiden Positionen gespielt werden, wird die Exploitability als Durchschnitt der Best-Response-Werte aus beiden Positionen berechnet:
\begin{equation}
\varepsilon(\sigma) = \frac{u_2(\sigma_1, b_2(\sigma_1)) + u_1(b_1(\sigma_2), \sigma_2)}{2}.
\end{equation}

In Nullsummenspielen ist die Exploitability eng mit dem Nash-Gleichgewicht verbunden.
Ein Nash-Gleichgewicht ist unexploitable, da es gegen jeden Gegner mindestens den Spielwert erreicht.
Daher ist die Exploitability eine zentrale Evaluationsmetrik.

\subsection{Konventionelle Best Response Berechnung}

Die konventionelle Berechnung einer Best Response erfolgt durch eine rekursive Traversierung des Informationset Trees des betrachteten Spielers.
Der Algorithmus traversiert den Baum von der Wurzel zu den Terminalknoten und berechnet dabei für jedes Informationset den erwarteten Nutzen.

An Terminalknoten muss der Algorithmus alle möglichen Spielzustände berücksichtigen, die für den Spieler nicht unterscheidbar sind.
Da der Spieler nicht weiß, in welchem konkreten Spielzustand er sich befindet, berechnet er die Erreichbarkeitswahrscheinlichkeiten des Gegners für die verschiedenen Informationsets.
Der Nutzen wird als gewichtete Summe über alle möglichen Spielzustände berechnet, wobei jeder Nutzen mit der entsprechenden Erreichbarkeitswahrscheinlichkeit gewichtet wird.
Dieser Wert wird zurückgegeben.

Während der Rücktraversierung durch den Baum werden an den verschiedenen Knotentypen unterschiedliche Operationen durchgeführt.
An Entscheidungsknoten des betrachteten Spielers wählt der Algorithmus die Aktion mit dem höchsten erwarteten Nutzen und speichert diese Wahl als Teil der Best-Response-Strategie.
Der Wert dieser gewählten Aktion wird zurückgegeben.
An Entscheidungsknoten des Gegners und an Zufallsknoten wird der erwartete Nutzen als Summe der Werte der Kindknoten berechnet und zurückgegeben.
Wenn der Algorithmus zur Wurzel zurückkehrt, ist der zurückgegebene Wert der Best-Response-Wert gegen die gegebene Gegnerstrategie.

\begin{figure}[htbp]
    \centering
    \includegraphics[width=0.8\textwidth]{chapters/02_theoretische_grundlagen/sections/05_exploitability/subsections/03_public_state_tree/gametree.png}
    \caption{Game Tree für Kuhn Poker}
    \label{fig:br_game_tree}
\end{figure}

\begin{figure}[htbp]
    \centering
    \includegraphics[width=0.8\textwidth]{chapters/02_theoretische_grundlagen/sections/05_exploitability/subsections/03_public_state_tree/informationset_tree.png}
    \caption{Information Set Trees für beide Spieler in  Kuhn Poker}
    \label{fig:br_infoset_tree}
\end{figure}

Die Abbildungen zeigen das Beispiel Kuhn Poker zur Illustration des Algorithmus.
Im Game Tree (Abbildung~\ref{fig:br_game_tree}) repräsentieren die Knoten \emph{A,X}, \emph{A,Y}, \emph{B,X} und \emph{B,Y} die terminalen Spielzustände, wobei \emph{A} und \emph{B} die Aktionen von Spieler 1 und \emph{X} und \emph{Y} die Aktionen von Spieler 2 bezeichnen.
Im Information Set Tree von Spieler 2 (Abbildung~\ref{fig:br_infoset_tree}) kann dieser am Terminalknoten \emph{X} nicht unterscheiden, ob er sich im Game Tree Zustand \emph{A,X} oder \emph{B,X} befindet, da diese nur durch die private Information von Spieler 1 unterschieden werden.
Daher muss der Algorithmus die Erreichbarkeitswahrscheinlichkeiten für beide möglichen Zustände berechnen.

~\cite{johanson2011accelerating} beschreiben eine beschleunigte Version des Algorithmus, die vier Optimierungen umfasst und im Folgenden erläutert wird.

\input{chapters/02_theoretische_grundlagen/sections/05_exploitability/subsections/02_accelerated_best_response/02_accelerated_best_response}
\subsection{Public State Tree}

Der zentrale Ansatz der beschleunigten Best-Response-Berechnung besteht darin, statt des Informationset Trees einen \emph{Public State Tree} zu traversieren.
Ein \emph{Public State} ist definiert als eine Partition der Historien, die folgende Eigenschaften erfüllt:
Keine zwei Historien aus demselben Informationset befinden sich in unterschiedlichen Public States.
Zwei Historien aus unterschiedlichen Public States haben keine Nachfahren im selben Public State, wodurch eine Baumstruktur entsteht.
Kein Public State enthält sowohl terminale als auch nicht-terminale Historien.

Informell ist ein Public State durch alle Informationen definiert, die beiden Spielern bekannt sind, also wie das Spiel für einen Beobachter ohne private Informationen aussieht.
Der Public State Tree ist deutlich kleiner als der Informationset Tree, da er nur die öffentlich sichtbaren Informationen enthält.

Der entscheidende Vorteil bei der Traversierung des Public State Trees liegt in der Wiederverwendung von Erreichbarkeitswahrscheinlichkeiten.
In der konventionellen Methode werden die Erreichbarkeitswahrscheinlichkeiten des Gegners an jedem Terminalknoten neu berechnet.
Im Public State Tree werden diese Wahrscheinlichkeiten jedoch einmal berechnet und für alle Informationsets des betrachteten Spielers in diesem Public State wiederverwendet.
Dies ist möglich, weil die Erreichbarkeitswahrscheinlichkeiten für alle eigenen Informationsets identisch sind, die sich im selben Public State befinden.

Statt wie in der konventionellen Methode einen einzelnen Wert für ein Informationset zurückzugeben, gibt der Algorithmus beim Public State Tree einen Vektor von Werten zurück.
Einen für jedes Informationset des betrachteten Spielers im Public State.
An Terminalknoten werden die Werte für alle Informationsets berechnet, wobei die Erreichbarkeitswahrscheinlichkeiten des Gegners einmal berechnet und für alle eigenen Informationsets verwendet werden.
An Entscheidungsknoten des betrachteten Spielers wird für jedes Informationset die beste Aktion gewählt, und es wird ein Vektor zurückgegeben, dessen Einträge die maximalen Aktionswerte darstellen.
An Entscheidungsknoten des Gegners und an Zufallsknoten werden die Wertevektoren der Kindknoten summiert und zurückgegeben.
Diese Umstrukturierung führt zu denselben Berechnungen wie die konventionelle Methode, ändert jedoch die Reihenfolge, sodass Strategieabfragen beim Gegner effizienter wiederverwendet werden können.
In Spielen wie Texas Hold'em, in denen jeder Spieler bis zu 1326 Informationsets pro Public State haben kann, ermöglicht dies eine erhebliche Reduzierung der Strategieabfragen.

\begin{figure}[htbp]
    \centering
    \includegraphics[width=0.35\textwidth]{chapters/02_theoretische_grundlagen/sections/05_exploitability/subsections/03_public_state_tree/public_tree.png}
    \caption{Public State Tree für Kuhn Poker}
    \label{fig:public_state_tree}
\end{figure}

Abbildung~\ref{fig:public_state_tree} zeigt den Public State Tree für das Beispiel Kuhn Poker.
Im Vergleich zum Game Tree (Abbildung~\ref{fig:br_game_tree}) und den Information Set Trees (Abbildung~\ref{fig:br_infoset_tree}) ist der Public State Tree deutlich kompakter, da er nur die öffentlich sichtbaren Informationen enthält.
Der Terminalknoten \emph{A,B,X,Y} repräsentiert alle vier möglichen terminalen Spielzustände \emph{A,X}, \emph{A,Y}, \emph{B,X} und \emph{B,Y} aus dem Game Tree.
An diesem Terminalknoten können die Erreichbarkeitswahrscheinlichkeiten des Gegners einmal berechnet und für alle Informationsets beider Spieler in diesem Public State wiederverwendet werden.

\subsection{Effiziente Evaluation der Terminalknoten}

Ein weiterer Beschleunigungsschritt betrifft die Evaluierung von Terminalknoten.
Eine naive Methode hätte $O(n^2)$ Komplexität, wobei $n$ die Anzahl der Informationsets pro Spieler ist.
In Pokerspielen kann diese auf $O(n)$ reduziert werden, indem die Informationsets nach Handstärke sortiert werden.
Da der Nutzen nur von der relativen Ordnung der Handstärken abhängt, können Indizes in die sortierte Liste des Gegners verwendet werden, die markieren, wo sich die Relation ändert (schwächer, gleich, stärker).
Um ein Informationset zu evaluieren, benötigt man nur die Gesamtwahrscheinlichkeit der Gegner-Informationsets in diesen drei Abschnitten.
Beim Übergang zum nächsten stärkeren Informationset werden die Indizes nur um einen Schritt verschoben, anstatt alle Gegner-Informationsets erneut zu betrachten.

Da die möglichen Hände der Spieler abhängig voneinander sind (eine Karte kann nicht von beiden Spielern gehalten werden), wird das Inklusions-Exklusions-Prinzip verwendet.
Beim Berechnen der Wahrscheinlichkeit von besseren oder schlechteren Gegnerhänden müssen unmögliche Hände ausgeschlossen werden, da der Gegner keine der eigenen Karten halten kann.
Hände, die eine der eigenen Karten enthalten, werden subtrahiert.
Hände, die beide eigene Karten enthalten, werden wieder addiert, da sie sonst doppelt subtrahiert würden (einmal für jede Karte).

\subsection{Spielspezifische Isomorphismen}

Ein weiterer Beschleunigungsschritt nutzt strategisch äquivalente Aktionen oder Ereignisse.
In vielen Kartenspielen sind Karten mit gleichem Rang, aber unterschiedlicher Farbe strategisch äquivalent.
In Poker ist das zumindest solange der Fall, bis weitere Karten aufgedeckt werden.
Bevor der Flop ausgeteilt wird, ist eine Herz zwei genauso gut wie eine Kreuz zwei.
Solche Mengen äquivalenter Zufallsereignisse werden als \emph{isomorph} bezeichnet, und eine wird willkürlich als Repräsentant gewählt.

Wenn die zu evaluierende Strategie für jedes Mitglied jeder Menge isomorpher Historien gleich ist, kann die Größe des Public State Trees erheblich reduziert werden, indem nur die Repräsentanten betrachtet werden.
Beim Zurückkehren durch einen Zufallsknoten während der Traversierung muss der Nutzen eines Repräsentanten mit der Anzahl der isomorphen States gewichtet werden, die er repräsentiert.
In Texas Hold'em führt diese Reduktion zu einem Public-State-Tree, der 21,5-mal kleiner ist als das vollständige Spiel.

\subsection{Parallele Berechnung}

Ein vierter Beschleunigungsschritt nutzt die Möglichkeit, unabhängige Teilbäume des Public State Trees parallel zu lösen.
Public States können parallel gelöst werden, wenn keiner ein Nachfahre des anderen ist, da sie keine gemeinsamen Berechnungen haben.
Beispielsweise können beim Erreichen eines öffentlichen Zufallsereignisses alle Kinder parallel gelöst werden.


In diesem Abschnitt wurde die Hauptmetrik beschrieben, die verwendet wird, um die Qualität einer Strategie zu bewerten, die Exploitability.
Darüber hinaus wurde beschrieben, wie diese mithilfe eines Best Response Agents berechnet wird und was für Optimierungen in der Berechnung vorgenommen werden können.
