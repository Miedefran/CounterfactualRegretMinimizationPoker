\subsection{Regret Matching+}

Regret Matching+ ist ein Regret minimierender Algorithmus, der ähnlich zu Regret Matching operiert.
Der entscheidende Unterschied liegt im Umgang mit negativen Regretwerten.
Während Regret Matching Aktionen mit akkumuliertem negativen Regret ignoriert, werden diese von Regret Matching+ auf null zurückgesetzt.

Formal speichert Regret Matching+ nicht die Regretwerte $R^t(a)$, sondern verwendet einen Regret-ähnlichen Wert:
\begin{equation}
Q^t(a) = {\left(Q^{t-1}(a) + \Delta R^t(a)\right)}^{+},
\end{equation}
wobei $x^+ = \max(x, 0)$ den positiven Anteil bezeichnet.
Dabei bezeichnet $\Delta R^t(a)$ den in Iteration $t$ hinzukommenden Regretanteil (Zuwachs des kumulierten Regrets) für Aktion $a$.
Die Strategie basiert auf folgenden Werten:
\begin{equation}
\sigma^t(a) = \frac{Q^{t-1}(a)}{\sum_{b \in A} Q^{t-1}(b)}.
\end{equation}

Für Regret Matching+ gilt folgende Regret-Schranke~\cite[Thm.~1]{bowling2015heads}:
Sei $A$ eine Menge von Aktionen und $v^t : A \to \mathbb{R}$ eine Sequenz von $T$ Wertfunktionen.
Falls $|v^t(a) - v^t(b)| \le L$ für alle $t$ und $a, b \in A$ gilt, dann hat ein Agent, der nach Regret Matching+ handelt, einen Regret von höchstens:
\begin{equation}
R^T \le L\sqrt{|A|T}.
\end{equation}
(\cite[Thm.~1]{bowling2015heads})

Diese Schranke ist identisch mit der Schranke von Regret Matching.
CFR+ hat daher die gleiche theoretische Regret-Schranke wie CFR, konvergiert in der Praxis jedoch deutlich schneller.
Dies lässt sich durch die besseren Tracking-Regret-Eigenschaften von Regret Matching+ erklären.\\
Die empirische Überlegenheit von Regret Matching+ zeigt sich insbesondere dann, wenn sich die beste Aktion plötzlich ändert.
Bei Regret Matching muss der Algorithmus warten, bis eine zuvor schlechte Aktion sich beweist, und muss dabei ihren gesamten akkumulierten negativen Regret überwinden.
Regret Matching+ kann dagegen sofort die neue beste Aktion spielen, da negative Regrets auf null zurückgesetzt werden und nicht erst überwunden werden müssen.