\subsection{Linear Weighted Average}

CFR+ verwendet eine linear gewichtete Durchschnittsstrategie anstelle der uniformen Durchschnittsstrategie von CFR.
Dabei wird jede Iteration $t$ mit Gewicht $t$ versehen, wodurch spätere Iterationen stärker gewichtet werden.
Die linear gewichtete Durchschnittsstrategie ist definiert als:
\begin{equation}
\bar{\sigma}_p^T = \frac{2}{T^2 + T} \sum_{t=1}^T t\sigma_p^t.
\end{equation}

~\cite[Thm.~3]{bowling2015heads} besagt Folgendes:
Wenn CFR+ für $T$ Iterationen in einem extensiven Spiel läuft, dann ist die linear gewichtete Durchschnittsstrategie ein approximatives Nash-Gleichgewicht.
Voraussetzung ist, dass die maximale Differenz zwischen den Auszahlungen für jeden Spieler durch eine Konstante $L$ beschränkt ist.
Die Qualität der Approximation wird durch die Schranke $2(|\mathcal{I}_1| + |\mathcal{I}_2|)L\sqrt{k}/\sqrt{T}$ beschrieben.
Dabei bezeichnet $k = \max_{I \in \mathcal{I}} |A(I)|$ die maximale Anzahl von Aktionen pro Informationset.
Die Schranke impliziert eine Konvergenzrate von $O(1/\sqrt{T})$, die asymptotisch identisch mit der Rate der uniformen Gewichtung ist.

Theorem~3 gilt nicht für CFR.
Im Gegensatz zu CFR+, wo lineare Gewichtung die Performance verbessert, verschlechtert lineare Gewichtung die Performance von Vanilla CFR.

In der Praxis erreicht die aktuelle Strategie in CFR+ oft bereits eine gute Approximation zum Nash Equilibrium, sodass die aggressive Gewichtung die Konvergenz weiter beschleunigen kann.

Empirisch wurde beobachtet, dass eine quadratische Gewichtung mit $t^2$ statt $t$ die Konvergenz von CFR+ weiter beschleunigen kann~\cite{brown2018discounted}.

%maybe hier noch mehr zu squared weight sagen idk


