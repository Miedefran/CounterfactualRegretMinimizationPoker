\section{Monte Carlo CFR}

Der CFR-Algorithmus benötigt in jeder Iteration eine vollständige Traversierung des gesamten Spielbaums.
Monte Carlo CFR ist ein im Paper \enquote{Monte Carlo Sampling for Regret Minimization in Extensive Games}~\cite{lanctot2009monte} veröffentlichtes Framework von Algorithmen, die diesen Aufwand durch verschiedene Sampling-Varianten reduzieren.
In diesem Abschnitt wird zuerst die Grundidee des Frameworks erklärt und anschließend genauer auf die einzelnen Sampling-Verfahren eingegangen.
Abschließend werden die theoretischen Eigenschaften besprochen.
Hierfür wird ein neues Maß zur Abschätzung der Komplexität eines Spiels eingeführt, das im Anhang~\ref{anhang:mwert} erklärt wird.

\subsection{Allgemeine Definition}
MCCFR fasst die Menge aller terminalen Historien $Z$ in Teilmengen zusammen, die als Blöcke bezeichnet werden.
Die Menge dieser Teilmengen ist definiert als $Q = \{Q_1, \ldots, Q_r\}$.
In jeder Iteration wird zufällig ein Block ausgewählt und nur die terminalen Historien dieses Blocks betrachtet.\\
Die Definition eines Blocks hängt von der verwendeten Samplingstrategie ab.
In dem Fall, dass Chance Sampling verwendet wird, enthält ein Block $Q_j$ alle terminalen Historien mit derselben Sequenz von Zufallsereignissen.\\
Beim External Sampling hingegen wird ein Block durch eine Kombination aus Zufallsereignissen und Aktionen des Gegners bestimmt.\\
Die extremste hier genannte Form des Samplings ist das Outcome Sampling, bei dem Zufallsereignisse, Aktionen des Gegners sowie private Aktionen zufällig ausgewählt werden.

Sei $q_j > 0$ die Wahrscheinlichkeit, dass Block $Q_j$ in der aktuellen Iteration ausgewählt wird. 
Für die Summe aller Blockwahrscheinlichkeiten gilt $\sum_{j=1}^r q_j = 1$.\\
Die Wahrscheinlichkeit, dass eine terminale Historie $z$ in der aktuellen Iteration betrachtet wird, beträgt $q(z) = \sum_{j:z \in Q_j} q_j$.\\
Als nächstes wird im Paper eine gewichtete Form des Counterfactual Value eingeführt, um zu berücksichtigen, dass eine terminale Historie nur mit Wahrscheinlichkeit $q(z)$ betrachtet wird.
Folgende Formel beschreibt den Sampled Counterfactual Value, wenn Block $Q_j$ aktualisiert wird:
\begin{equation}
\tilde{v}_i(\sigma, I|j) = \sum_{z \in Q_j \cap Z_I} \frac{1}{q(z)} u_i(z) \pi_{-i}^\sigma(z[I]) \pi^\sigma(z[I], z).
\end{equation}
(\cite[Eq.~6]{lanctot2009monte})

Die Gewichtung mit $\frac{1}{q(z)}$ sorgt dafür, dass der Erwartungswert des Sampled Counterfactual Value dem tatsächlichen Counterfactual Value entspricht.
Im Gegensatz zu \\\cite{zinkevich2007regret} wird der Counterfactual Value hier mit $v_i(\sigma,I)$ anstelle von $u_i(\sigma,I)$ beschrieben.
\begin{equation}
\mathbb{E}_{j \sim q_j}[\tilde{v}_i(\sigma, I|j)] = v_i(\sigma, I).
\end{equation}
(\cite[Lemma~1]{lanctot2009monte})


Durch die Gewichtung wird garantiert, dass MCCFR im Erwartungswert die gleichen Regret-Updates wie CFR durchführt, obwohl nur ein Teil des Spielbaums in jeder Iteration betrachtet wird.
Der MCCFR-Algorithmus sampelt in jeder Iteration einen Block und berechnet für jedes von diesem Block betroffene Informationset die Sampled Immediate Counterfactual Regrets.
Diese Regretwerte werden akkumuliert und die Strategie wird in der nächsten Iteration mittels Regret Matching aktualisiert.


\subsection{Chance Sampling CFR}

Chance Sampling CFR ist ein Spezialfall des MCCFR-Frameworks, bei dem nur die Ausgänge von Zufallsereignissen gesampelt werden.
Dies betrifft sowohl die Ausgänge von privaten als auch öffentlichen Zufallsereignissen.
Nach dem Sampling wird der gesamte Subtree für diese Kombination traversiert, das heißt, alle möglichen Aktionen beider Spieler werden betrachtet.


\subsection{External Sampling MCCFR}
External Sampling MCCFR sampelt nur die Aktionen des Gegners sowie die Ausgänge von Zufallsereignissen.
Für jede deterministische Strategie $\tau$ von Gegner und Zufall wird ein Block $Q_\tau \in Q$ definiert.
Dabei ist $\tau$ eine deterministische Abbildung von Informationsets $I \in \mathcal{I}_c \cup \mathcal{I}_{N \setminus \{i\}}$ auf Aktionen $A(I)$.

Die Blockwahrscheinlichkeiten sind:
\begin{equation}
q_\tau = \prod_{I \in \mathcal{I}_c} f_c(\tau(I)|I) \prod_{I \in \mathcal{I}_{N \setminus \{i\}}} \sigma_{-i}(\tau(I)|I).
\end{equation}
Der Block $Q_\tau$ enthält alle terminalen Historien $z$, die mit $\tau$ konsistent sind (d.\,h.\ entlang von $z$ wählt $\tau$ an allen $I \in \mathcal{I}_c \cup \mathcal{I}_{N \setminus \{i\}}$ jeweils die in $z$ auftretende Aktion).
Es gilt $q(z) = \pi_{-i}^\sigma(z)$.

Für jeden Spieler $i \in N$ werden an jeder History $h$ mit $P(h) \neq i$ Aktionen gesampelt.
Die gesampelten Aktionen werden pro Informationset festgelegt, sodass innerhalb einer Iteration an allen Knoten desselben Informationsets dieselbe Aktion verwendet wird.
In Spielen mit \enquote{Perfect Recall} wird ein Informationset auf einem einzelnen Pfad höchstens einmal erreicht.

Für jedes besuchte Informationset werden die Sampled Immediate Counterfactual Regrets $\tilde{r}(I,a)$ berechnet:
\begin{equation}
\tilde{r}(I, a) = (1 - \sigma(a|I)) \sum_{z \in Q_\tau \cap Z_I} u_i(z) \pi_i^\sigma(z[I]a, z).
\end{equation}
(\cite[Eq.~11]{lanctot2009monte})


\subsection{Outcome Sampling MCCFR}

Outcome Sampling MCCFR sampelt pro Iteration genau eine terminale Historie $z$ und aktualisiert nur die Informationsets entlang dieses Pfads.
Damit ist der Aufwand pro Iteration minimal, da ein Block die Größe $|Q_j| = 1$ besitzt.

Welche Historie $z$ gesampelt wird, legt eine \emph{Sampling Policy} $\sigma'$ fest: $q(z) = \pi^{\sigma'}(z)$.
Die Sampling Policy $\sigma'$ kann von der aktuellen Strategie $\sigma$ abweichen.
Um diese Abweichung zu korrigieren, wird der Beitrag der Stichprobe mit einem Gewicht $w_I$ versehen (Importance Sampling).
Damit $w_I$ wohldefiniert bleibt, muss $q(z) \ge \delta > 0$ gelten, beispielsweise durch $\sigma'_i(a|I) \ge \epsilon > 0$ für alle Aktionen.

In jeder Iteration wird eine terminale Historie $z$ gemäß $\sigma'$ gesampelt.
Anschließend erfolgt eine Rückwärtstraversierung entlang $z$.
An jedem Informationset $I$ auf diesem Pfad werden die Sampled Immediate Counterfactual Regrets $\tilde{r}(I,a)$ berechnet und akkumuliert:
\begin{equation}
\tilde{r}(I, a) = \begin{cases}
w_I \cdot (1 - \sigma(a|z[I])) & \text{falls } (z[I]a) \sqsubset z \\
-w_I \cdot \sigma(a|z[I]) & \text{sonst}
\end{cases},
\end{equation}
mit
\begin{equation}
w_I = \frac{u_i(z) \pi_{-i}^\sigma(z) \pi_i^\sigma(z[I]a, z)}{\pi^{\sigma'}(z)}.
\end{equation}
(\cite[Eq.~10]{lanctot2009monte})

Die Notation $(z[I]a) \sqsubset z$ bedeutet, dass auf dem gesampelten Pfad $z$ am Informationset $I$ die Aktion $a$ gewählt wurde (Fall 1); andernfalls wird Fall 2 verwendet.

\subsection{Theoretische Eigenschaften}
Lanctot et~al.~\cite{lanctot2009monte} verwenden zur genaueren Angabe von Regretschranken die Größe $M_i$.
Der M-Wert $M_i$ beschreibt, wie sich die Informationsets von Spieler $i$ über seine Aktionssequenzen verteilen, und liegt zwischen $\sqrt{|\mathcal{I}i|}$ und $|\mathcal{I}_i|$. 
Dieses Maß ist abhängig von der Struktur des spezifischen Spiels und schätzt die Komplexität genauer ab, als die reine Anzahl der Informationsets.
Die genaue Definition, sowie die Erklärung anhand eines Beispiels ist imAnhang~\ref{anhang:mwert} zu finden.

Für Vanilla CFR gilt:
\begin{equation}
R_i^T \le \Delta_{u,i} M_i \sqrt{|A_i|}/\sqrt{T},
\end{equation}
(\cite[Thm.~3]{lanctot2009monte}),
wobei $\Delta_{u,i}$ und $|A_i|$ wie in Abschnitt~\ref{sec:regret-matching} definiert sind.

Für External Sampling MCCFR gilt für jedes $p$ mit $0 < p \le 1$ mit Wahrscheinlichkeit mindestens $1-p$:
\begin{equation}
R_i^T \le \left(1 + \sqrt{\frac{2}{p}}\right) \Delta_{u,i} M_i \sqrt{|A_i|}/\sqrt{T}.
\end{equation}
(\cite[Thm.~4]{lanctot2009monte}).

Obwohl External Sampling nur einen konstanten Faktor mehr Iterationen benötigt als Vanilla CFR, wird pro Iteration nur ein Bruchteil des Spielbaums traversiert.
Für ausgewogene Spiele, in denen die Spieler ungefähr gleich viele Entscheidungen treffen, liegen die Iterationskosten (Anzahl besuchter Historien) bei External Sampling bei $O(\sqrt{|H|})$, während Vanilla CFR $O(|H|)$ benötigt.
Damit ergibt sich asymptotisch eine geringere Gesamtzeit zur Berechnung eines approximativen Gleichgewichts (\cite[Sec.~4]{lanctot2009monte}).

Für Outcome Sampling MCCFR gilt für jedes $p$ mit $0 < p \le 1$ mit Wahrscheinlichkeit mindestens $1-p$:
\begin{equation}
R_i^T \le \left(1 + \sqrt{\frac{2}{p}} \cdot \frac{1}{\sqrt{\delta}}\right) \Delta_{u,i} M_i \sqrt{|A_i|}/\sqrt{T},
\end{equation}
wenn $q(z) \ge \delta > 0$ für alle relevanten terminalen Historien $z$ gilt (\cite[Thm.~5]{lanctot2009monte}).
Der Faktor $1/\sqrt{\delta}$ hängt von der Sampling Policy ab.


In diesem Abschnitt wurde gezeigt, wie die Iterationszeit des CFR-Algorithmus durch verschiedene Sampling-Techniken erheblich reduziert werden kann, ohne das Konvergenzverhalten dabei zu stark zu beeinträchtigen.
Im nächsten Abschnitt wird CFR+ vorgestellt, eine Optimierung, die die Konvergenzgeschwindigkeit verbessert und damit schneller zu einer Strategie hoher Qualität führt.