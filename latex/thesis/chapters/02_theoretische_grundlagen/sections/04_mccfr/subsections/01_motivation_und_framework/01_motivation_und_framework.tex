\subsection{Motivation und Framework}

Die von Zinkevich et al. präsentierte Vanilla Version von CFR erfordert in jeder Iteration eine vollständige Traversierung des gesamten Game Trees.
Dies macht den Algorithmus für sehr große Spiele unpraktikabel.

MCCFR stellt ein allgemeines Framework für Sampling in Counterfactual Regret Minimization dar.
Es definiert eine Familie von Monte Carlo CFR Algorithmen, die sich darin unterscheiden, wie sie den Game Tree in jeder Iteration sampeln.
Vanilla CFR und eine Verallgemeinerung des chance sampled CFR von Zinkevich et al. sind Spezialfälle dieses Frameworks.
Zusätzlich werden zwei neue Mitglieder dieser Familie eingeführt: Outcome Sampling, bei dem nur ein einzelnes Spiel pro Iteration gesampelt wird, und External Sampling, das Zufallsknoten und Aktionen des Gegners sampelt.

Unter einer angemessenen Sampling Strategie minimiert jedes Mitglied dieser Familie den overall regret und kann daher für die Berechnung von Nash Gleichgewichten verwendet werden.

