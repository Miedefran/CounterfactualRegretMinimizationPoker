\subsection{Allgemeine Definition}
MCCFR fasst die Menge aller terminalen Historien $Z$ in Teilmengen zusammen, die als Blöcke bezeichnet werden.
Die Menge dieser Teilmengen ist definiert als $Q = \{Q_1, \ldots, Q_r\}$.
In jeder Iteration wird zufällig ein Block ausgewählt und nur die terminalen Historien dieses Blocks betrachtet.\\
Die Definition eines Blocks hängt von der verwendeten Samplingstrategie ab.
In dem Fall, dass Chance Sampling verwendet wird, enthält ein Block $Q_j$ alle terminalen Historien mit derselben Sequenz von Zufallsereignissen.\\
Beim External Sampling hingegen wird ein Block durch eine Kombination aus Zufallsereignissen und Aktionen des Gegners bestimmt.\\
Die extremste hier genannte Form des Samplings ist das Outcome Sampling, bei dem Zufallsereignisse, Aktionen des Gegners sowie private Aktionen zufällig ausgewählt werden.

Sei $q_j > 0$ die Wahrscheinlichkeit, dass Block $Q_j$ in der aktuellen Iteration ausgewählt wird. 
Für die Summe aller Blockwahrscheinlichkeiten gilt $\sum_{j=1}^r q_j = 1$.\\
Die Wahrscheinlichkeit, dass eine terminale Historie $z$ in der aktuellen Iteration betrachtet wird, beträgt $q(z) = \sum_{j:z \in Q_j} q_j$.\\
Als nächstes wird im Paper eine gewichtete Form des Counterfactual Value eingeführt, um zu berücksichtigen, dass eine terminale Historie nur mit Wahrscheinlichkeit $q(z)$ betrachtet wird.
Folgende Formel beschreibt den Sampled Counterfactual Value, wenn Block $Q_j$ aktualisiert wird:
\begin{equation}
\tilde{v}_i(\sigma, I|j) = \sum_{z \in Q_j \cap Z_I} \frac{1}{q(z)} u_i(z) \pi_{-i}^\sigma(z[I]) \pi^\sigma(z[I], z).
\end{equation}
(\cite[Eq.~6]{lanctot2009monte})

Die Gewichtung mit $\frac{1}{q(z)}$ sorgt dafür, dass der Erwartungswert des Sampled Counterfactual Value dem tatsächlichen Counterfactual Value entspricht.
Im Gegensatz zu \\\cite{zinkevich2007regret} wird der Counterfactual Value hier mit $v_i(\sigma,I)$ anstelle von $u_i(\sigma,I)$ beschrieben.
\begin{equation}
\mathbb{E}_{j \sim q_j}[\tilde{v}_i(\sigma, I|j)] = v_i(\sigma, I).
\end{equation}
(\cite[Lemma~1]{lanctot2009monte})


Durch die Gewichtung wird garantiert, dass MCCFR im Erwartungswert die gleichen Regret-Updates wie CFR durchführt, obwohl nur ein Teil des Spielbaums in jeder Iteration betrachtet wird.
Der MCCFR-Algorithmus sampelt in jeder Iteration einen Block und berechnet für jedes von diesem Block betroffene Informationset die Sampled Immediate Counterfactual Regrets.
Diese Regretwerte werden akkumuliert und die Strategie wird in der nächsten Iteration mittels Regret Matching aktualisiert.

