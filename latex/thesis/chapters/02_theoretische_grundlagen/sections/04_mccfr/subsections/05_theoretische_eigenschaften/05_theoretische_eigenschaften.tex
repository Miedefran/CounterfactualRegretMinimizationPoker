\subsection{Theoretische Eigenschaften}
Lanctot et~al.~\cite{lanctot2009monte} verwenden zur genaueren Angabe von Regretschranken die Größe $M_i$.
Der M-Wert $M_i$ beschreibt, wie sich die Informationsets von Spieler $i$ über seine Aktionssequenzen verteilen, und liegt zwischen $\sqrt{|\mathcal{I}i|}$ und $|\mathcal{I}_i|$. 
Dieses Maß ist abhängig von der Struktur des spezifischen Spiels und schätzt die Komplexität genauer ab, als die reine Anzahl der Informationsets.
Die genaue Definition, sowie die Erklärung anhand eines Beispiels ist imAnhang~\ref{anhang:mwert} zu finden.

Für Vanilla CFR gilt:
\begin{equation}
R_i^T \le \Delta_{u,i} M_i \sqrt{|A_i|}/\sqrt{T},
\end{equation}
(\cite[Thm.~3]{lanctot2009monte}),
wobei $\Delta_{u,i}$ und $|A_i|$ wie in Abschnitt~\ref{sec:regret-matching} definiert sind.

Für External Sampling MCCFR gilt für jedes $p$ mit $0 < p \le 1$ mit Wahrscheinlichkeit mindestens $1-p$:
\begin{equation}
R_i^T \le \left(1 + \sqrt{\frac{2}{p}}\right) \Delta_{u,i} M_i \sqrt{|A_i|}/\sqrt{T}.
\end{equation}
(\cite[Thm.~4]{lanctot2009monte}).

Obwohl External Sampling nur einen konstanten Faktor mehr Iterationen benötigt als Vanilla CFR, wird pro Iteration nur ein Bruchteil des Spielbaums traversiert.
Für ausgewogene Spiele, in denen die Spieler ungefähr gleich viele Entscheidungen treffen, liegen die Iterationskosten (Anzahl besuchter Historien) bei External Sampling bei $O(\sqrt{|H|})$, während Vanilla CFR $O(|H|)$ benötigt.
Damit ergibt sich asymptotisch eine geringere Gesamtzeit zur Berechnung eines approximativen Gleichgewichts (\cite[Sec.~4]{lanctot2009monte}).

Für Outcome Sampling MCCFR gilt für jedes $p$ mit $0 < p \le 1$ mit Wahrscheinlichkeit mindestens $1-p$:
\begin{equation}
R_i^T \le \left(1 + \sqrt{\frac{2}{p}} \cdot \frac{1}{\sqrt{\delta}}\right) \Delta_{u,i} M_i \sqrt{|A_i|}/\sqrt{T},
\end{equation}
wenn $q(z) \ge \delta > 0$ für alle relevanten terminalen Historien $z$ gilt (\cite[Thm.~5]{lanctot2009monte}).
Der Faktor $1/\sqrt{\delta}$ hängt von der Sampling Policy ab.
