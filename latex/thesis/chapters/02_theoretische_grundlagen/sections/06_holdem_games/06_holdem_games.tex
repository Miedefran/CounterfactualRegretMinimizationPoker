\section{Hold'em-Spiele}
In diesem Abschnitt wird erklärt, wodurch sich Hold'em-Spiele als Untergruppe von Pokerspielen auszeichnen.
Ist die Grundstruktur von Hold'em-Spielen einmal definiert, lässt sie sich einfach erweitern.
Die Definition spezifischer Pokervarianten erfolgt im Anhang~\ref{anhang:spieldefinitionen}.\\
Diese Arbeit beschäftigt sich ausschließlich mit Heads-Up-Limit-Versionen von Hold'em-Spielen.
Heads-Up bedeutet, dass genau zwei Spieler teilnehmen.
Die Klassifikation Limit bezieht sich darauf, dass die Setzgrößen der Spieler auf vorher festgelegte Beträge beschränkt sind.
Diese beiden Einschränkungen reduzieren die Komplexität erheblich.
Die folgende Definition gilt für alle Heads-Up-Limit-Hold'em-Spiele~\cite{southey2005bayes}.

Eine Partie nennt man eine \emph{Hand}.
Eine Hand wiederum besteht aus mehreren Runden.
In der ersten Runde erhalten beide Spieler eine feste Anzahl privater Karten (Hole Cards).
In \emph{jeder} Runde können zusätzlich öffentliche Karten (Board Cards) aufgedeckt werden (auch null).\\
Nach dem Austeilen bzw. Aufdecken der Karten folgt jeweils eine Setzrunde, in der die Spieler abwechselnd handeln.
Zur Verfügung stehen drei Aktionen: \enquote{Fold, Call und Raise}.\\
Wählt ein Spieler die Aktion \emph{Fold}, endet die Hand sofort und der andere Spieler gewinnt den Pot.
Der Pot ist die Ansammlung aller Einsätze, die im Verlauf der Hand gesetzt wurden.\\
Bei \emph{Call} gleicht der Spieler den Einsatz des Gegners (der Beitrag kann null sein; dann spricht man von \emph{Check}).\\
Bei \emph{Raise} gleicht der Spieler den Gegnerzug und setzt zusätzlich den vorgeschriebenen Aufschlag.

Die Setzrunde endet, sobald ein Spieler \emph{foldet} (dann endet die Hand sofort) oder \emph{callt} (dann endet nur die Runde). 
Um endlose Raise-Ketten zu vermeiden, ist die Zahl der Raises pro Runde begrenzt.
Sobald das Limit erreicht ist, sind nur noch die Aktionen Call und Fold erlaubt.
Läuft die letzte Setzrunde ohne Fold aus, folgt ein \emph{Showdown}: Beide Spieler zeigen ihre Hole Cards und bilden aus Hole und Board Cards die bestmögliche Hand.
Die stärkere Hand gewinnt den Pot.
Die zulässigen Pokerhände und deren Rangfolge hängen von der Art des Spiels ab.
Eine Übersicht über die möglichen Pokerhände in Texas Hold'em findet sich im Anhang~\ref{anhang:pokerhaende}.
Die Rangfolge anderer Spiele ist in der Definition der einzelnen Spiele enthalten~\ref{anhang:spieldefinitionen}.

Im Folgenden werden die zentralen Bestandteile genannt, die relevant sind, um Heads-Up-Limit-Hold'em-Varianten zu beschreiben.
\begin{itemize}
    \item die Anzahl der Hole Cards und Board Cards
    \item die Anzahl der Setzrunden
    \item das Setzlimit pro Runde
    \item die im Deck enthaltenen Karten
    \item die möglichen Pokerhände
    \item die festgelegten Setzgrößen
\end{itemize}


