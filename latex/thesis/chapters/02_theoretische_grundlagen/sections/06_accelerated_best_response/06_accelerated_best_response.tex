\section{Exploitability einer Strategie}

Um die Qualität einer Strategie zu bewerten, ist es wichtig, ihre Worst-Case-Performance zu kennen.
In Nullsummenspielen kann dies durch die Berechnung einer \emph{best response} ermittelt werden.
Eine best response ist die optimale Strategie für einen Spieler gegen eine gegebene Gegnerstrategie.

Die \emph{exploitability} einer Strategie misst, wie viel zusätzlicher Nutzen durch eine best response verloren geht, verglichen mit einer optimalen Strategie.
In einem Zweipersonen-Nullsummenspiel ist die exploitability einer Strategie $\sigma_i$ definiert als
\begin{equation}
\varepsilon_i(\sigma_i) = v_i - u_i(\sigma_i, b_{-i}(\sigma_i)),
\end{equation}
wobei $v_i$ der Spielwert für Spieler $i$ ist und $b_{-i}(\sigma_i)$ die best response des Gegners gegen $\sigma_i$ darstellt.
Eine Strategie ist optimal, wenn ihre exploitability null ist.

Für Strategien, die in beiden Positionen gespielt werden, wird die exploitability als Durchschnitt der best response Werte aus beiden Positionen berechnet:
\begin{equation}
\varepsilon(\sigma) = \frac{u_2(\sigma_1, b_2(\sigma_1)) + u_1(b_1(\sigma_2), \sigma_2)}{2}.
\end{equation}

In Nullsummenspielen ist die exploitability eng mit dem Nash-Gleichgewicht verbunden: Eine Strategie ist optimal, wenn ihre exploitability null ist.
Daher ist die exploitability eine zentrale Metrik zur Evaluation von Strategien.

\subsection{Konventionelle Best Response Berechnung}

Die konventionelle Berechnung einer best response erfolgt durch eine rekursive Traversierung des Information Set Trees.
Der Algorithmus traversiert den Information Set Tree des betrachteten Spielers und berechnet für jedes Informationset den erwarteten Nutzen.
An Terminalknoten müssen alle möglichen Game States berücksichtigt werden, die dem Spieler nicht unterscheidbar sind.
Dazu werden die \emph{Erreichbarkeitswahrscheinlichkeiten} des Gegners berechnet, also die Wahrscheinlichkeiten, mit denen der Gegner verschiedene Informationsets erreicht.
Der Nutzen wird als gewichtete Summe über alle möglichen Game States berechnet, wobei jeder Nutzen mit der entsprechenden Erreichbarkeitswahrscheinlichkeit gewichtet wird.
An Entscheidungsknoten wählt der Algorithmus die Aktion mit dem höchsten erwarteten Wert, um die best response Strategie zu konstruieren.

Diese Methode erfordert eine vollständige Traversierung des Information Set Trees, was für große Spiele wie Poker mit $10^{18}$ Game States selbst bei moderner Hardware nicht durchführbar ist.

\subsection{Accelerated Best Response}

Johanson et al.~\cite{johanson2011accelerating} präsentieren eine beschleunigte Methode, die eine vollständige Traversierung vermeidet.
Der Accelerated Best Response Algorithmus nutzt die Struktur von Information und Utilities, um Berechnungen wiederzuverwenden und zu vermeiden.
Dies ermöglicht es erstmals, die Worst-Case-Performance von Strategien in großen Spielen wie Texas Hold'em zu berechnen.

\input{chapters/02_theoretische_grundlagen/sections/06_accelerated_best_response/subsections/01_public_state_tree/01_public_state_tree}
\input{chapters/02_theoretische_grundlagen/sections/06_accelerated_best_response/subsections/02_effiziente_terminal_node_evaluierung/02_effiziente_terminal_node_evaluierung}
\input{chapters/02_theoretische_grundlagen/sections/06_accelerated_best_response/subsections/03_isomorphismus/03_isomorphismus}
\input{chapters/02_theoretische_grundlagen/sections/06_accelerated_best_response/subsections/04_parallel_computation/04_parallel_computation}
