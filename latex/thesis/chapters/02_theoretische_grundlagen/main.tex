\chapter{Theoretische Grundlagen}\label{chap:grundlagen}
In diesem Kapitel werden die theoretischen Grundlagen der zentralen Konzepte dieser Arbeit anhand von Konferenzpapers dargestellt.
Zunächst werden spieltheoretische Begriffe definiert.
Anschließend werden die verschiedenen CFR-Varianten anhand der jeweiligen Konferenzpapers erläutert.
Ausgehend von der Vanilla-Version \cite{zinkevich2007regret}, über das samplingbasierte Monte-Carlo-Framework \cite{lanctot2009monte}, bis hin zu CFR+ \cite{bowling2015heads} und Discounted CFR \cite{brown2018discounted}.
Danach wird die Theorie hinter der Hauptevaluationsmetrik Exploitability und dem zugehörigen Best-Response-Agent erläutert \cite{johanson2011accelerating}.
Abschließend wird eine grundlegende Definition von Heads-Up-Hold'em-Spielen gegeben, mit welcher jedes beliebige Spiel dieser Kategorie dargestellt werden kann.

\section{Grundlagen der Spieltheorie}
In diesem Abschnitt wird ein kurzer Überblick darüber gegeben, was Spieltheorie ist und wo sie angewendet wird.
Dabei wird definiert, was im spieltheoretischen Kontext unter einem \emph{Spiel} und einer \emph{Lösung} verstanden wird.
Anschließend werden die wichtigsten Typen von Spielen umrissen und der Typ extensiver Spiele mit imperfekter Information näher betrachtet.
Als Grundlage für dieses Kapitel dient das Buch \enquote{A Course in Game Theory} \cite{osborne1994course}.

\subsection{Definition und Konzepte}
Die Spieltheorie beschäftigt sich mit der Analyse strategischer Interaktionen zwischen Entscheidungsträgern und bietet hierfür eine Vielzahl analytischer Methoden.
Diese basieren auf zwei zentralen Annahmen~\cite[Kap.~1.1]{osborne1994course}.


Die erste Annahme ist, dass Entscheidungsträger rational handeln.
Ein rationaler Entscheidungsträger kennt seine Handlungsmöglichkeiten, kann Erwartungen über unbekannte Situationen bilden und hat klare Vorstellungen darüber, welche Ergebnisse er bevorzugt.
Er wählt seine Aktion bewusst nach einem Prozess der Optimierung.
Liegt keine Unsicherheit vor, wählt er die Aktion, die zu den für ihn besten Konsequenzen führt.
Diese Präferenzen lassen sich durch eine Nutzenfunktion beschreiben, die jedem möglichen Ergebnis einen Wert zuordnet.
Die beste Aktion ist die, die den höchsten Nutzen erzielt.
Sind die Konsequenzen von Aktionen unsicher, maximiert ein rationaler Entscheidungsträger den erwarteten Nutzen.
Hierfür muss er in der Lage sein, Wahrscheinlichkeiten für unsichere Ereignisse abzuschätzen~\cite[Kap.~1.4]{osborne1994course}.

Die zweite Annahme ist, dass Entscheidungsträger ihr Wissen oder ihre Erwartungen über das Verhalten anderer Spieler bei ihrer Entscheidungsfindung berücksichtigen. 
Dies nennt man auch strategisches Handeln.
Spieltheoretisches Denken berücksichtigt, dass jeder Entscheidungsträger vor seiner Entscheidung versucht, Informationen über das Verhalten der anderen Spieler zu erhalten.
Im Gegensatz dazu nimmt die Theorie des Wettbewerbsgleichgewichts an, dass jeder Akteur sich nur für Umgebungsparameter wie Preise interessiert, auch wenn diese durch die Aktionen aller Akteure bestimmt werden~\cite[Kap.~1.3]{osborne1994course}.\\
Ein anschauliches Beispiel hierfür ist das Gefangenendilemma. 
Zwei Verdächtige werden getrennt verhört und müssen jeweils entscheiden, ob sie schweigen oder den anderen belasten. 
Schweigen beide, erhalten beide zwei Jahre Haft.
Wenn beide gestehen, erhalten sie fünf Jahre Haft.
Gesteht jedoch nur einer, wird dieser freigelassen, während der andere zu zehn Jahren Haft verurteilt wird.
Die optimale Entscheidung ist abhängig von den Tendenzen anderer Entscheidungsträger.
Strategisches Handeln bedeutet, solche Erwartungen über das Verhalten anderer Spieler in die Entscheidungsfindung mit einzubeziehen~\cite[Kap.~1.1]{osborne1994course}.

Spieltheoretische Modelle befinden sich auf einer hohen Abstraktionsebene und können daher verwendet werden, um ein breites Spektrum an Phänomenen zu betrachten. 
Zur formalen Darstellung dieser Modelle werden mathematische Definitionen genutzt. 
Die zugrunde liegenden Konzepte sind jedoch nicht primär mathematischer Natur und lassen sich auch ohne formale Notation erläutern.\\
Die Spieltheorie findet Anwendung in unzähligen Bereichen.
Zu den klassischen Anwendungsgebieten zählen die Analyse wirtschaftlicher und politischer Wettbewerbssituationen. 
Darüber hinaus wird sie auch in der Evolutionsbiologie, der Soziologie und anderen Disziplinen genutzt, um strategische Interaktionen zu modellieren\\~\cite[Kap.~1.1]{osborne1994course}.

Ein Spiel beschreibt eine strategische Interaktion zwischen Spielern mit festgelegten Rahmenbedingungen. 
Es definiert, welche Aktionen den Spielern zur Verfügung stehen und welche Ergebnisse in ihrem Interesse sind.
Das Spiel legt damit die Struktur und die Regeln der Interaktion fest, bestimmt aber nicht, wie sich die Spieler tatsächlich verhalten.\\
Die Frage, welche Aktionen die Spieler tatsächlich ausführen, wird nicht durch das Spiel selbst beantwortet, sondern durch die Lösung des Spiels.
Eine Lösung ist ein Konzept, das systematisch beschreibt, welche Ergebnisse in einer Klasse von Spielen unter bestimmten Verhaltensannahmen zu erwarten sind.
Sie liefert damit eine Vorhersage über das tatsächliche Verhalten der Spieler unter den getroffenen Annahmen über Rationalität und strategisches Denken~\cite[Kap.~1.2]{osborne1994course}.\\
Spiele können anhand verschiedener Kriterien systematisch kategorisiert werden.
Grundlegend für alle spieltheoretischen Modelle ist das Konzept des Spielers. 
Ein Spieler kann dabei sowohl eine einzelne Person als auch eine Gruppe von Personen darstellen, die gemeinsam eine Entscheidung treffen.

Die erste grundlegende Unterscheidung betrifft kooperative und nicht-kooperative Spiele.
Nicht-kooperative Spiele modellieren die Aktionen einzelner Spieler. 
Kooperative Spiele hingegen konzentrieren sich auf Gruppen von Spielern.
Dabei wird nicht betrachtet, wie diese Gruppen intern funktionieren~\cite[Kap.~1.2]{osborne1994course}.

Eine weitere wichtige Kategorisierung betrifft die Darstellungsform des Spiels. 
Hierbei werden strategische Spiele, auch als Spiele in Normalform bezeichnet, von extensiven Spielen unterschieden. 
Bei strategischen Spielen wählen die Spieler ihre Strategie zu Beginn des Spiels und passen diese während des Spielverlaufs nicht mehr an. 
Die Entscheidungen werden zudem simultan getroffen, sodass keine Informationen über die Aktionen der Gegenspieler verfügbar sind. 
Extensive Spiele hingegen spezifizieren die möglichen Reihenfolgen von Ereignissen und erlauben es jedem Spieler, seinen Aktionsplan nicht nur zu Beginn, sondern auch bei jeder weiteren Entscheidung zu überdenken~\cite[Kap.~1.2]{osborne1994course}.

Die letzte hier zu nennende Unterscheidung betrifft den Informationsstand der Spieler. 
Bei Spielen mit perfekter Information sind alle Teilnehmer vollständig über die Züge der anderen Spieler informiert. 
Bei Spielen mit imperfekter Information hingegen können die Spieler unvollständig über die Aktionen der anderen Spieler oder über Ausgänge von Zufallsereignissen informiert sein~\cite[Kap.~1.2]{osborne1994course}.

Ein \emph{strikt kompetitives Spiel} (auch \emph{Nullsummenspiel}) liegt vor, wenn der Gewinn des einen Spielers genau dem Verlust des anderen entspricht.
Das bedeutet, dass für alle möglichen Spielausgänge die Summe der Nutzenwerte beider Spieler null ergibt.
Formal: Ist die Nutzenfunktion von Spieler 1 durch $u_1$ gegeben, so gilt für die Nutzenfunktion $u_2$ von Spieler 2 stets $u_1 + u_2 = 0$~\cite[Def.~21.1]{osborne1994course}.

\subsection{Extensive Games with Imperfect Information}
Nachdem die grundlegenden Konzepte der Spieltheorie eingeführt wurden, widmet sich dieser Abschnitt den extensiven Spielen mit imperfekter Information.
Diese Spielklasse ist für die vorliegende Arbeit von zentraler Bedeutung, da die meisten Pokerspiele, insbesondere Hold'em-Varianten, zu dieser Kategorie gehören.

\subsubsection{Definition}
Ein extensives Spiel mit imperfekter Information modelliert eine sequenzielle Interaktion zwischen Spielern, bei der nicht alle Spieler stets vollständig über alle vergangenen Aktionen informiert sind. 
Dabei werden die möglichen Spielverläufe als Historien erfasst, wobei eine Historie eine Abfolge von Aktionen darstellt. 
Zu jedem Zeitpunkt ist entweder ein Spieler oder der Zufall am Zug. \\
cUm die imperfekte Information darzustellen, werden Informationsets verwendet. 
Diese fassen Zustände zusammen, die ein Spieler nicht voneinander unterscheiden kann. 
Hierbei können sowohl vergangene Aktionen anderer Spieler als auch Ausgänge von Zufallsereignissen betroffen sein.

Ein extensives Spiel mit imperfekter Information lässt sich formal durch folgende Komponenten definieren:
\begin{itemize}
  \item Eine Menge von Historien $H$.
  Eine Historie ist eine endliche oder unendliche Sequenz von Aktionen, getätigt von Spielern oder dem Zufall.\\
  Die Menge $H$ erfüllt folgende Eigenschaften:
  \begin{itemize}
    \item Die leere Sequenz $\emptyset$ ist ein Element von $H$.
    \item Ist $(a^k)_{k=1,\dots,K} \in H$ (wobei $K$ unendlich sein kann) und $L < K$, dann gilt $(a^k)_{k=1,\dots,L} \in H$.
    Das bedeutet, wenn eine Sequenz ein Teil von $H$ ist, dann ist auch jeder Präfix dieser Sequenz ein Teil von $H$.
    \item Wenn für eine unendliche Folge $(a^k)_{k=1}^\infty$ die Bedingung $(a^k)_{k=1,\dots,L} \in H$ für jede positive ganze Zahl $L$ erfüllt ist, dann gilt $(a^k)_{k=1}^\infty \in H$.
  \end{itemize}
  \item Eine Historie $(a^k)_{k=1,\dots,K} \in H$ ist terminal, wenn sie unendlich ist oder wenn kein $a^{K+1}$ existiert, so dass $(a^k)_{k=1,\dots,K+1} \in H$ gilt. 
  Man bezeichnet die Menge aller terminalen Historien mit $Z$.\\
  Die Menge der Aktionen, die nach einer nicht terminalen Historie $h$ verfügbar sind, wird mit $A(h) = \{a : (h,a) \in H\}$ bezeichnet.
  
  \item Die Funktion $P$ ordnet jeder nicht terminalen Historie $h \in H \setminus Z$ einen Spieler aus $N$ oder $c$ zu.
  $P(h)$ bezeichnet den Spieler, der nach $h$ am Zug ist. 
  Wenn $P(h) = c$ ist, bestimmt der Zufall die nächste Aktion.
  
  \item Die Funktion $f_c$ ordnet jeder Historie $h$ mit $P(h) = c$ eine Zufallsverteilung $f_c(\cdot \mid h)$ auf $A(h)$ zu.
  $f_c(a \mid h)$ ist die Wahrscheinlichkeit, dass nach $h$ die Aktion $a$ folgt. 
  Diese Verteilungen sind unabhängig voneinander.

  \item Für jeden Spieler $i \in N$ wird die Menge der Historien $h$ mit $P(h)=i$ in Informationsets unterteilt.
  Diese Informationspartition wird bezeichnet mit $\mathcal{I}_i$.
  Ein Element $I_i \in \mathcal{I}_i$ heißt Informationset von Spieler $i$.
  Historien $h$ und $h'$, die zum selben Informationset $I_i$ gehören, sind für Spieler $i$ nicht unterscheidbar.
  Wesentlich ist dabei, dass alle Historien in einem Informationset $I_i$ die gleichen verfügbaren Aktionen haben, also $A(h) = A(h')$ für alle $h, h' \in I_i$. 
  Für ein Informationset $I_i$ bezeichnet $A(I_i)$ die verfügbaren Aktionen und $P(I_i)$ den Spieler, der am Zug ist.

  \item Für jeden Spieler $i \in N$ ist eine Präferenzrelation $\succeq_i$ über Lotterien auf $Z$ gegeben.
  Eine Lotterie ist hierbei eine Wahrscheinlichkeitsverteilung über terminale Historien.
  Diese Präferenzen lassen sich durch eine Nutzenfunktion $u_i: Z \to \mathbb{R}$ repräsentieren, sodass für zwei Lotterien $L_1, L_2$ gilt:
  $L_1 \succeq_i L_2 \iff \mathbb{E}_{z \sim L_1}[u_i(z)] \ge \mathbb{E}_{z \sim L_2}[u_i(z)]$.
\end{itemize}
Formale Definition entspricht~\cite[Ch.~11, Sec.~11.1, Def.~200.1]{osborne1994course}.

\subsubsection{Strategien}

In extensiven Spielen mit imperfekter Information gibt es verschiedene Möglichkeiten, wie die Strategien von Spielern definiert werden können.\\
Eine reine Strategie von Spieler $i \in N$ ordnet jedem Informationset $I \in \mathcal{I}_i$ genau eine zulässige Aktion $s_i(I) \in A(I)$ zu.
Sie legt somit einen vollständigen Handlungsplan für alle Informationslagen von $i$ fest.\\
Eine gemischte Strategie $\alpha_i$ ist eine Wahrscheinlichkeitsverteilung über der Menge der reinen Strategien von $i$.\\
Eine behavioral Strategie $\beta_i$ hingegen ist eine Sammlung unabhängiger Wahrscheinlichkeitsverteilungen $(\beta_i(I))_{I \in \mathcal{I}_i}$, wobei $\beta_i(I)$ eine Wahrscheinlichkeitsverteilung über die Aktionen $A(I)$ am Informationset $I$ ist~\cite[Ch.~11, Sec.~11.4, Def.~212.1]{osborne1994course}.\\
Der Unterschied liegt darin, dass bei einer gemischten Strategie vor Spielbeginn zufällig eine komplette reine Strategie ausgewählt wird, die dann während des gesamten Spiels deterministisch befolgt wird.\\
Im Gegensatz dazu wird bei einer behavioral Strategie in jedem Informationset unabhängig eine Wahrscheinlichkeitsverteilung über die verfügbaren Aktionen verwendet, sodass die zufällige Auswahl der Aktionen während des Spiels erfolgt\\~\cite[Ch.~11, Sec.~11.4]{osborne1994course}.\\
Ein extensives Spiel hat \enquote{Perfect Recall}, wenn sich jeder Spieler zu jedem Zeitpunkt an alles erinnert, was er in der Vergangenheit wusste und welche Aktionen er selbst getroffen hat~\cite[Ch.~11, Sec.~11.1]{osborne1994course}.
Unter dieser Annahme erzeugen beide Repräsentationen outcome-äquivalente Wahrscheinlichkeitsverteilungen über die terminalen Historien~\cite[Ch.~11, Sec.~11.4, Prop.~214.1]{osborne1994course}.\\
Ein Profil von behavioral Strategien $\beta=(\beta_i)_{i\in N}$ zusammen mit den Zufallszügen $f_c$ bestimmt eine Wahrscheinlichkeitsverteilung über die terminalen Historien $Z$.
Daraus lässt sich für jede Auszahlungsfunktion $u_i$ der Erwartungswert $\mathbb{E}_{z \sim O(\beta, f_c)}[u_i(z)]$ berechnen.


Diese spieltheoretischen Grundlagen bilden die Basis für das Verständnis der Algorithmen, die in den kommenden Abschnitten vorgestellt werden.
\section{Counterfactual Regret Minimization}
Dieser Abschnitt stellt die theoretischen Grundlagen von Counterfactual Regret Minimization (CFR), dem Kernstück dieser Arbeit, vor.
Als fachliche Grundlage dient das Paper \glqq Regret Minimization in Games with Incomplete Information\grqq ~\cite{zinkevich2007regret}, in welchem CFR erstmals vorgestellt wurde.\\
Zuerst werden einige wichtige Werte und Zusammenhänge definiert, die essenziell für das Verständnis der darauffolgenden Konzepte sind.
Anschließend wird erklärt was Regret-Minimierung ist.
Darauf aufbauend wird das von Zinkevich et al.\ eingeführte Konzept des Counterfactual Regret sowie Regret Matching erläutert.
Abschließend fügen sich die einzelnen Komponenten zu einem Algorithmus zusammen.

\subsection{Definitionen}
\subsubsection{Wahrscheinlichkeiten und Werte}

Für ein Strategieprofil $\sigma$ werden folgende Wahrscheinlichkeiten und Werte definiert:

Die Wahrscheinlichkeit, dass eine Historie $h$ auftritt, wenn alle Spieler gemäß $\sigma$ handeln, wird mit $\pi^\sigma(h)$ bezeichnet.
Diese Wahrscheinlichkeit lässt sich in ein Produkt zerlegen:
\begin{equation}
\pi^\sigma(h) = \prod_{i \in N \cup \{c\}} \pi_i^\sigma(h),
\end{equation}
wobei $\pi_i^\sigma(h)$ die Wahrscheinlichkeit bezeichnet, dass Spieler $i$ in all seinen Entscheidungsknoten auf dem Pfad zu $h$ gemäß $\sigma$ die Aktionen gewählt hat, die zu $h$ führen.
Das Produkt der Wahrscheinlichkeiten aller anderen Spieler (inklusive Zufallsereignisse) wird mit $\pi_{-i}^\sigma(h)$ bezeichnet.

Für ein Informationset $I$ wird die Erreichbarkeitswahrscheinlichkeit definiert als:
\begin{equation}
\pi^\sigma(I) = \sum_{h \in I} \pi^\sigma(h).
\end{equation}
Analog sind $\pi_i^\sigma(I)$ und $\pi_{-i}^\sigma(I)$ als die Wahrscheinlichkeiten definiert, dass Spieler $i$ beziehungsweise die anderen Spieler das Informationset $I$ erreichen.

Der erwartete Nutzen für Spieler $i$ unter Strategieprofil $\sigma$, auch \emph{Overall Value} genannt, wird wie folgt berechnet:
\begin{equation}
u_i(\sigma) = \sum_{h \in Z} u_i(h) \pi^\sigma(h),
\end{equation}


\subsubsection{Nash-Gleichgewicht}
Ein Nash‑Gleichgewicht beschreibt ein Strategienprofil, bei dem kein Spieler durch eine einseitige Änderung seiner Strategie seinen erwarteten Nutzen erhöhen kann\\~\cite[Ch.~11, Sec.~11.5]{osborne1994course}.\\
Ein Nash-Gleichgewicht für ein Zweispielernullsummenspiel ist ein Strategieprofil $\sigma = (\sigma_1, \sigma_2)$, das folgende Ungleichungen erfüllt:
\begin{equation}
u_1(\sigma) \ge \max_{\sigma_1' \in \Sigma_1} u_1(\sigma_1', \sigma_2),
\qquad
u_2(\sigma) \ge \max_{\sigma_2' \in \Sigma_2} u_2(\sigma_1, \sigma_2')
\end{equation}
(\cite[Eq.~1]{zinkevich2007regret}).

Eine Approximation eines Nash-Gleichgewichts, auch $\varepsilon$-Nash-Gleichgewicht genannt, ist ein Strategieprofil $\sigma$, das folgende Ungleichungen erfüllt:
\begin{equation}
u_1(\sigma) + \varepsilon \ge \max_{\sigma_1' \in \Sigma_1} u_1(\sigma_1', \sigma_2),
\qquad
u_2(\sigma) + \varepsilon \ge \max_{\sigma_2' \in \Sigma_2} u_2(\sigma_1, \sigma_2')
\end{equation}
(\cite[Eq.~2]{zinkevich2007regret}).

\subsection{Regret-Minimierung}
Regret misst den Unterschied zwischen dem maximal erreichbaren Nutzen unter Verwendung der zu diesem Zeitpunkt bestmöglichen Strategie und dem tatsächlich erzielten Nutzen.\\
Um Regret-Minimierung zu definieren, wird das wiederholte Spielen eines extensiven Spiels betrachtet.
Sei $\sigma_i^t$ die von Spieler $i$ in Runde $t$ verwendete Strategie.
Der durchschnittliche \emph{Overall Regret} von Spieler $i$ zum Zeitpunkt $T$ ist:
\begin{equation}
R_i^T = \frac{1}{T}\max_{\sigma_i^\ast\in\Sigma_i}\sum_{t=1}^T \bigl(u_i(\sigma_i^\ast,\sigma_{-i}^t) - u_i(\sigma^t)\bigr).
\end{equation}
(\cite[Eq.~3]{zinkevich2007regret})

Die Durchschnittsstrategie $\bar\sigma_i^T$ für Spieler $i$ von Zeit $1$ bis $T$ wird für jedes Informationset $I \in \mathcal{I}_i$ und jede Aktion $a \in A(I)$ definiert als:
\begin{equation}
\label{eq:cfr-durchschnittsstrategie}
\bar\sigma_i^T(I)(a) = \frac{\sum_{t=1}^T \pi_i^{\sigma^t}(I)\,\sigma_i^t(I)(a)}{\sum_{t=1}^T \pi_i^{\sigma^t}(I)}.
\end{equation}
(\cite[Eq.~4]{zinkevich2007regret})

Die Durchschnittsstrategie gewichtet Aktionen proportional zur Erreichbarkeitswahrscheinlichkeit $\pi_i^{\sigma^t}(I)$ des Informationsets $I$ in Runde $t$.

Es besteht ein wichtiger Zusammenhang zwischen Regret und Nash-Gleichgewichten.
\label{thm:cfr-nash}
In einem Nullsummenspiel gilt: Ist der durchschnittliche Overall Regret beider Spieler zum Zeitpunkt $T$ kleiner als $\varepsilon$, dann ist $\bar\sigma^T$ ein $2\varepsilon$-Nash-Gleichgewicht (\cite[Thm.~2]{zinkevich2007regret}).

Ein Algorithmus zur Bestimmung von $\sigma_i^t$ ist Regret-minimierend, wenn der durchschnittliche Overall Regret von Spieler $i$ unabhängig von der Strategiefolge $\sigma_{-i}^t$ des Gegners gegen Null konvergiert, während $T$ gegen Unendlich läuft.
Daher können Regret-minimierende Algorithmen im Self-Play verwendet werden, um ein approximatives Nash-Gleichgewicht zu berechnen.


\subsection{Counterfactual Regret}
Die zentrale Idee, die Zinkevich et al. in ihrem Paper "Regret Minimization in Games with Incomplete Information" präsentieren, ist, den gesamten Regret-Wert in eine Menge aus additiven Termen zu zerlegen.
Diese Terme können dann unabhängig voneinander minimiert werden.
Insbesondere wird ein völlig neues Regret-Konzept mit dem Namen \emph{Counterfactual Regret} eingeführt, das für jedes Informationset definiert ist.

Um das Konzept zu erläutern, wird ein Informationset $I \in \mathcal{I}_i$ betrachtet.\\
Die \emph{Counterfactual Utility} oder auch \emph{Counterfactual Value} $u_i(\sigma, I)$ ist der erwartete Nutzen für Spieler $i$ unter der Annahme, dass das Informationset $I$ erreicht wird.
Dabei spielen alle Spieler gemäß dem Strategieprofil $\sigma$, wobei Spieler $i$ so handelt, dass $I$ tatsächlich erreicht wird.
Mit $\pi^\sigma(h, h')$ als Wahrscheinlichkeit, von $h$ nach $h'$ zu gelangen, gilt:
\begin{equation}
u_i(\sigma, I) = \frac{\sum_{h \in I,\, h' \in Z} \pi_{-i}^\sigma(h)\,\pi^\sigma(h, h')\,u_i(h')}{\pi_{-i}^\sigma(I)}.
\end{equation}
(\cite[Eq.~5]{zinkevich2007regret})

Für alle $a \in A(I)$ ist $\sigma|_{I \to a}$ ein Strategieprofil, das identisch zu $\sigma$ ist, außer dass Spieler $i$ immer die Aktion $a$ ausführt, wenn er das Informationset $I$ erreicht.
Der \emph{Immediate Counterfactual Regret} ist
\begin{equation}
R_{i,\mathrm{imm}}^{T}(I) = \frac{1}{T}\max_{a \in A(I)} \sum_{t=1}^{T} \pi_{-i}^{\sigma^t}(I)\,\bigl(u_i(\sigma^t|_{I \to a}, I) - u_i(\sigma^t, I)\bigr).
\end{equation}
(\cite[Eq.~6]{zinkevich2007regret})

Dieser Term misst den Regret im Informationset $I$, gewichtet mit der \emph{Counterfactual Probability}, dass $I$ erreicht werden würde, wenn Spieler $i$ es erzwingen wollte.
Die positiven Immediate Counterfactual Regret-Werte werden definiert als:
\begin{equation}
R_{i,\mathrm{imm}}^{T,+}(I) = \max\!\bigl(R_{i,\mathrm{imm}}^{T}(I), 0\bigr).
\end{equation}

Theorem 3 beschreibt die erste große Erkenntnis des Papers. 
\begin{equation}
    R_i^T \le \sum_{I \in \mathcal{I}_i} R_{i,\mathrm{imm}}^{T,+}(I)
\end{equation}
(\cite[Thm.~3]{zinkevich2007regret})\\
Das heißt, wenn der Immediate Counterfactual Regret minimiert wird, dann wird auch der Average Overall Regret minimiert.

\subsection{Regret Matching}
\label{sec:regret-matching}

Mit folgender Formel wird der Counterfactual Regret für eine konkrete Aktion $a$ in einem Informationset $I$ berechnet.
\begin{equation}
    R_i^{T}(I, a) = \frac{1}{T} \sum_{t=1}^{T} \pi_{-i}^{\sigma^{t}}(I)
    \left( u_i(\sigma^{t}|_{I \to a}, I) - u_i(\sigma^{t}, I) \right)
\end{equation}
(\cite[Eq.~7]{zinkevich2007regret})

Dabei sei $R_i^{T,+}(I, a) = \max(R_i^{T}(I, a), 0)$ als positiver Anteil des Immediate Counterfactual Regret für Aktion $a$ definiert.

Die Strategie für Iteration $T+1$ wird dann durch Regret Matching bestimmt:
\begin{equation}
\label{eq:cfr-regret-matching}
\sigma_i^{T+1}(I)(a) = \begin{cases}
\frac{R_i^{T,+}(I, a)}{\sum_{a' \in A(I)} R_i^{T,+}(I, a')} & \text{falls } \sum_{a' \in A(I)} R_i^{T,+}(I, a') > 0 \\
\frac{1}{|A(I)|} & \text{in jedem anderen Fall.}
\end{cases}
\end{equation}
(\cite[Eq.~8]{zinkevich2007regret})

Diese Formel weist Aktionen Wahrscheinlichkeiten proportional zu ihrem positiven Regret aus dem Nichtspielen dieser Aktion zu.
Gibt es keine positiven Regretwerte, werden alle Aktionen gleichverteilt gewählt.

Theorem 4 zeigt, dass bei Verwendung von \hyperref[eq:cfr-regret-matching]{Formel~\ref*{eq:cfr-regret-matching}} der Immediate Counterfactual Regret und der Overall Regret folgendermaßen beschränkt sind:
\begin{equation}
R_{i,\mathrm{imm}}^{T}(I) \le \frac{\Delta_{u,i}\sqrt{|A_i|}}{\sqrt{T}},
\end{equation}
\begin{equation}
R_i^T \le \frac{\Delta_{u,i}|\mathcal{I}_i|\sqrt{|A_i|}}{\sqrt{T}},
\end{equation}
wobei $\Delta_{u,i} = \max_z u_i(z) - \min_z u_i(z)$ der Nutzenbereich und $|A_i| = \max_{h:P(h)=i} |A(h)|$ die maximale Anzahl verfügbarer Aktionen für Spieler $i$ ist (\cite[Thm.~4]{zinkevich2007regret}).
Die Schranke für den Average Overall Regret ist linear in der Anzahl der Informationsets $|\mathcal{I}_i|$ und fällt asymptotisch mit $O(1/\sqrt{T})$.
Zusammen mit \hyperref[thm:cfr-nash]{Theorem~2} folgt, dass die Durchschnittsstrategie $\bar\sigma^T$ im Selbstspiel gegen ein Nash-Gleichgewicht konvergiert (\cite[Thm.~2]{zinkevich2007regret}).

\subsection{CFR-Algorithmus}

Das CFR-Verfahren verbindet die Komponenten aus den vorherigen Abschnitten zu einer wiederholten Selbstspielprozedur.
In jeder Iteration wird die Strategie gemäß \hyperref[eq:cfr-regret-matching]{Formel~\ref*{eq:cfr-regret-matching}} aktualisiert.
Für jedes Informationset $I$ und jede Aktion $a$ werden die kumulierten Regretwerte $R_i^t(I,a)$ gespeichert und nach jeder Iteration aktualisiert.
Parallel dazu wird die Durchschnittsstrategie gemäß \hyperref[eq:cfr-durchschnittsstrategie]{Formel~\ref*{eq:cfr-durchschnittsstrategie}} aktualisiert.
Nach $T$ Iterationen gibt der Algorithmus $(\bar\sigma_1^T, \bar\sigma_2^T)$ als approximatives Nash-Gleichgewicht zurück.


In diesem Abschnitt wurde die Basisversion des Counterfactual Regret Minimization Algorithmus vorgestellt.
In den kommenden Abschnitten folgen die optimierten Varianten, beginnend mit MCCFR.
\section{Monte Carlo CFR}

Der CFR-Algorithmus benötigt in jeder Iteration eine vollständige Traversierung des gesamten Spielbaums.
Monte Carlo CFR ist ein im Paper \enquote{Monte Carlo Sampling for Regret Minimization in Extensive Games}~\cite{lanctot2009monte} veröffentlichtes Framework von Algorithmen, die diesen Aufwand durch verschiedene Sampling-Varianten reduzieren.
In diesem Abschnitt wird zuerst die Grundidee des Frameworks erklärt und anschließend genauer auf die einzelnen Sampling-Verfahren eingegangen.
Abschließend werden die theoretischen Eigenschaften besprochen.
Hierfür wird ein neues Maß zur Abschätzung der Komplexität eines Spiels eingeführt, das im Anhang~\ref{anhang:mwert} erklärt wird.

\subsection{Allgemeine Definition}
MCCFR fasst die Menge aller terminalen Historien $Z$ in Teilmengen zusammen, die als Blöcke bezeichnet werden.
Die Menge dieser Teilmengen ist definiert als $Q = \{Q_1, \ldots, Q_r\}$.
In jeder Iteration wird zufällig ein Block ausgewählt und nur die terminalen Historien dieses Blocks betrachtet.\\
Die Definition eines Blocks hängt von der verwendeten Samplingstrategie ab.
In dem Fall, dass Chance Sampling verwendet wird, enthält ein Block $Q_j$ alle terminalen Historien mit derselben Sequenz von Zufallsereignissen.\\
Beim External Sampling hingegen wird ein Block durch eine Kombination aus Zufallsereignissen und Aktionen des Gegners bestimmt.\\
Die extremste hier genannte Form des Samplings ist das Outcome Sampling, bei dem Zufallsereignisse, Aktionen des Gegners sowie private Aktionen zufällig ausgewählt werden.

Sei $q_j > 0$ die Wahrscheinlichkeit, dass Block $Q_j$ in der aktuellen Iteration ausgewählt wird. 
Für die Summe aller Blockwahrscheinlichkeiten gilt $\sum_{j=1}^r q_j = 1$.\\
Die Wahrscheinlichkeit, dass eine terminale Historie $z$ in der aktuellen Iteration betrachtet wird, beträgt $q(z) = \sum_{j:z \in Q_j} q_j$.\\
Als nächstes wird im Paper eine gewichtete Form des Counterfactual Value eingeführt, um zu berücksichtigen, dass eine terminale Historie nur mit Wahrscheinlichkeit $q(z)$ betrachtet wird.
Folgende Formel beschreibt den Sampled Counterfactual Value, wenn Block $Q_j$ aktualisiert wird:
\begin{equation}
\tilde{v}_i(\sigma, I|j) = \sum_{z \in Q_j \cap Z_I} \frac{1}{q(z)} u_i(z) \pi_{-i}^\sigma(z[I]) \pi^\sigma(z[I], z).
\end{equation}
(\cite[Eq.~6]{lanctot2009monte})

Die Gewichtung mit $\frac{1}{q(z)}$ sorgt dafür, dass der Erwartungswert des Sampled Counterfactual Value dem tatsächlichen Counterfactual Value entspricht.
Im Gegensatz zu \\\cite{zinkevich2007regret} wird der Counterfactual Value hier mit $v_i(\sigma,I)$ anstelle von $u_i(\sigma,I)$ beschrieben.
\begin{equation}
\mathbb{E}_{j \sim q_j}[\tilde{v}_i(\sigma, I|j)] = v_i(\sigma, I).
\end{equation}
(\cite[Lemma~1]{lanctot2009monte})


Durch die Gewichtung wird garantiert, dass MCCFR im Erwartungswert die gleichen Regret-Updates wie CFR durchführt, obwohl nur ein Teil des Spielbaums in jeder Iteration betrachtet wird.
Der MCCFR-Algorithmus sampelt in jeder Iteration einen Block und berechnet für jedes von diesem Block betroffene Informationset die Sampled Immediate Counterfactual Regrets.
Diese Regretwerte werden akkumuliert und die Strategie wird in der nächsten Iteration mittels Regret Matching aktualisiert.


\subsection{Chance Sampling CFR}

Chance Sampling CFR ist ein Spezialfall des MCCFR-Frameworks, bei dem nur die Ausgänge von Zufallsereignissen gesampelt werden.
Dies betrifft sowohl die Ausgänge von privaten als auch öffentlichen Zufallsereignissen.
Nach dem Sampling wird der gesamte Subtree für diese Kombination traversiert, das heißt, alle möglichen Aktionen beider Spieler werden betrachtet.


\subsection{External Sampling MCCFR}
External Sampling MCCFR sampelt nur die Aktionen des Gegners sowie die Ausgänge von Zufallsereignissen.
Für jede deterministische Strategie $\tau$ von Gegner und Zufall wird ein Block $Q_\tau \in Q$ definiert.
Dabei ist $\tau$ eine deterministische Abbildung von Informationsets $I \in \mathcal{I}_c \cup \mathcal{I}_{N \setminus \{i\}}$ auf Aktionen $A(I)$.

Die Blockwahrscheinlichkeiten sind:
\begin{equation}
q_\tau = \prod_{I \in \mathcal{I}_c} f_c(\tau(I)|I) \prod_{I \in \mathcal{I}_{N \setminus \{i\}}} \sigma_{-i}(\tau(I)|I).
\end{equation}
Der Block $Q_\tau$ enthält alle terminalen Historien $z$, die mit $\tau$ konsistent sind (d.\,h.\ entlang von $z$ wählt $\tau$ an allen $I \in \mathcal{I}_c \cup \mathcal{I}_{N \setminus \{i\}}$ jeweils die in $z$ auftretende Aktion).
Es gilt $q(z) = \pi_{-i}^\sigma(z)$.

Für jeden Spieler $i \in N$ werden an jeder History $h$ mit $P(h) \neq i$ Aktionen gesampelt.
Die gesampelten Aktionen werden pro Informationset festgelegt, sodass innerhalb einer Iteration an allen Knoten desselben Informationsets dieselbe Aktion verwendet wird.
In Spielen mit \enquote{Perfect Recall} wird ein Informationset auf einem einzelnen Pfad höchstens einmal erreicht.

Für jedes besuchte Informationset werden die Sampled Immediate Counterfactual Regrets $\tilde{r}(I,a)$ berechnet:
\begin{equation}
\tilde{r}(I, a) = (1 - \sigma(a|I)) \sum_{z \in Q_\tau \cap Z_I} u_i(z) \pi_i^\sigma(z[I]a, z).
\end{equation}
(\cite[Eq.~11]{lanctot2009monte})


\subsection{Outcome Sampling MCCFR}

Outcome Sampling MCCFR sampelt pro Iteration genau eine terminale Historie $z$ und aktualisiert nur die Informationsets entlang dieses Pfads.
Damit ist der Aufwand pro Iteration minimal, da ein Block die Größe $|Q_j| = 1$ besitzt.

Welche Historie $z$ gesampelt wird, legt eine \emph{Sampling Policy} $\sigma'$ fest: $q(z) = \pi^{\sigma'}(z)$.
Die Sampling Policy $\sigma'$ kann von der aktuellen Strategie $\sigma$ abweichen.
Um diese Abweichung zu korrigieren, wird der Beitrag der Stichprobe mit einem Gewicht $w_I$ versehen (Importance Sampling).
Damit $w_I$ wohldefiniert bleibt, muss $q(z) \ge \delta > 0$ gelten, beispielsweise durch $\sigma'_i(a|I) \ge \epsilon > 0$ für alle Aktionen.

In jeder Iteration wird eine terminale Historie $z$ gemäß $\sigma'$ gesampelt.
Anschließend erfolgt eine Rückwärtstraversierung entlang $z$.
An jedem Informationset $I$ auf diesem Pfad werden die Sampled Immediate Counterfactual Regrets $\tilde{r}(I,a)$ berechnet und akkumuliert:
\begin{equation}
\tilde{r}(I, a) = \begin{cases}
w_I \cdot (1 - \sigma(a|z[I])) & \text{falls } (z[I]a) \sqsubset z \\
-w_I \cdot \sigma(a|z[I]) & \text{sonst}
\end{cases},
\end{equation}
mit
\begin{equation}
w_I = \frac{u_i(z) \pi_{-i}^\sigma(z) \pi_i^\sigma(z[I]a, z)}{\pi^{\sigma'}(z)}.
\end{equation}
(\cite[Eq.~10]{lanctot2009monte})

Die Notation $(z[I]a) \sqsubset z$ bedeutet, dass auf dem gesampelten Pfad $z$ am Informationset $I$ die Aktion $a$ gewählt wurde (Fall 1); andernfalls wird Fall 2 verwendet.

\subsection{Theoretische Eigenschaften}
Lanctot et~al.~\cite{lanctot2009monte} verwenden zur genaueren Angabe von Regretschranken die Größe $M_i$.
Der M-Wert $M_i$ beschreibt, wie sich die Informationsets von Spieler $i$ über seine Aktionssequenzen verteilen, und liegt zwischen $\sqrt{|\mathcal{I}i|}$ und $|\mathcal{I}_i|$. 
Dieses Maß ist abhängig von der Struktur des spezifischen Spiels und schätzt die Komplexität genauer ab, als die reine Anzahl der Informationsets.
Die genaue Definition, sowie die Erklärung anhand eines Beispiels ist imAnhang~\ref{anhang:mwert} zu finden.

Für Vanilla CFR gilt:
\begin{equation}
R_i^T \le \Delta_{u,i} M_i \sqrt{|A_i|}/\sqrt{T},
\end{equation}
(\cite[Thm.~3]{lanctot2009monte}),
wobei $\Delta_{u,i}$ und $|A_i|$ wie in Abschnitt~\ref{sec:regret-matching} definiert sind.

Für External Sampling MCCFR gilt für jedes $p$ mit $0 < p \le 1$ mit Wahrscheinlichkeit mindestens $1-p$:
\begin{equation}
R_i^T \le \left(1 + \sqrt{\frac{2}{p}}\right) \Delta_{u,i} M_i \sqrt{|A_i|}/\sqrt{T}.
\end{equation}
(\cite[Thm.~4]{lanctot2009monte}).

Obwohl External Sampling nur einen konstanten Faktor mehr Iterationen benötigt als Vanilla CFR, wird pro Iteration nur ein Bruchteil des Spielbaums traversiert.
Für ausgewogene Spiele, in denen die Spieler ungefähr gleich viele Entscheidungen treffen, liegen die Iterationskosten (Anzahl besuchter Historien) bei External Sampling bei $O(\sqrt{|H|})$, während Vanilla CFR $O(|H|)$ benötigt.
Damit ergibt sich asymptotisch eine geringere Gesamtzeit zur Berechnung eines approximativen Gleichgewichts (\cite[Sec.~4]{lanctot2009monte}).

Für Outcome Sampling MCCFR gilt für jedes $p$ mit $0 < p \le 1$ mit Wahrscheinlichkeit mindestens $1-p$:
\begin{equation}
R_i^T \le \left(1 + \sqrt{\frac{2}{p}} \cdot \frac{1}{\sqrt{\delta}}\right) \Delta_{u,i} M_i \sqrt{|A_i|}/\sqrt{T},
\end{equation}
wenn $q(z) \ge \delta > 0$ für alle relevanten terminalen Historien $z$ gilt (\cite[Thm.~5]{lanctot2009monte}).
Der Faktor $1/\sqrt{\delta}$ hängt von der Sampling Policy ab.


In diesem Abschnitt wurde gezeigt, wie die Iterationszeit des CFR-Algorithmus durch verschiedene Sampling-Techniken erheblich reduziert werden kann, ohne das Konvergenzverhalten dabei zu stark zu beeinträchtigen.
Im nächsten Abschnitt wird CFR+ vorgestellt, eine Optimierung, die die Konvergenzgeschwindigkeit verbessert und damit schneller zu einer Strategie hoher Qualität führt.
\section{CFR+}
CFR+ bildet die Grundlage für Cepheus, das erste Programm, das Heads-up Limit Texas Hold'em gelöst hat~\cite{bowling2015heads}.
Der Algorithmus erreicht durch drei Hauptänderungen gegenüber Vanilla CFR eine deutlich schnellere Konvergenz.


In diesem Abschnitt werden die algorithmischen Unterschiede zu CFR beschrieben.
Regret Matching+ als zentrale Änderung, alternierende Updates und eine linear gewichtete Durchschnittsstrategie.
Zudem werden die Tracking-Regret-Eigenschaften erläutert, die die praktische Überlegenheit von CFR+ erklären.

\subsection{Unterschiede zu Vanilla CFR}

CFR+ unterscheidet sich von Vanilla CFR durch drei Hauptänderungen.

Erstens verwendet CFR+ eine linear gewichtete Durchschnittsstrategie statt der uniformen Durchschnittsstrategie.
Die gewichtete Durchschnittsstrategie ist definiert als $\bar{\sigma}_p^T = 2/(T^2 + T) \sum_{t=1}^T t\sigma_p^t$, wobei spätere Iterationen stärker gewichtet werden.

Zweitens führt CFR+ alternierende Updates durch.
Im Gegensatz zu Vanilla CFR, der die Regretwerte beider Spieler simultan aktualisiert, aktualisiert CFR+ die Regretwerte der Spieler abwechselnd.

Drittens verwendet CFR+ Regret Matching+ anstelle von Regret Matching.
Regret Matching+ setzt negative Regretwerte auf null zurück, statt sie zu ignorieren, was zu besseren Tracking-Regret-Eigenschaften führt.

\subsection{Regret Matching+}

Regret Matching+ ist ein Regret minimierender Algorithmus, der ähnlich zu Regret Matching operiert.
Der entscheidende Unterschied liegt im Umgang mit negativen Regretwerten.
Während Regret Matching Aktionen mit akkumuliertem negativen Regret ignoriert, werden diese von Regret Matching+ auf null zurückgesetzt.

Formal speichert Regret Matching+ nicht die Regretwerte $R^t(a)$, sondern verwendet einen Regret-ähnlichen Wert:
\begin{equation}
Q^t(a) = {\left(Q^{t-1}(a) + \Delta R^t(a)\right)}^{+},
\end{equation}
wobei $x^+ = \max(x, 0)$ den positiven Anteil bezeichnet.
Dabei bezeichnet $\Delta R^t(a)$ den in Iteration $t$ hinzukommenden Regretanteil (Zuwachs des kumulierten Regrets) für Aktion $a$.
Die Strategie basiert auf folgenden Werten:
\begin{equation}
\sigma^t(a) = \frac{Q^{t-1}(a)}{\sum_{b \in A} Q^{t-1}(b)}.
\end{equation}

Für Regret Matching+ gilt folgende Regret-Schranke~\cite[Thm.~1]{bowling2015heads}:
Sei $A$ eine Menge von Aktionen und $v^t : A \to \mathbb{R}$ eine Sequenz von $T$ Wertfunktionen.
Falls $|v^t(a) - v^t(b)| \le L$ für alle $t$ und $a, b \in A$ gilt, dann hat ein Agent, der nach Regret Matching+ handelt, einen Regret von höchstens:
\begin{equation}
R^T \le L\sqrt{|A|T}.
\end{equation}
(\cite[Thm.~1]{bowling2015heads})

Diese Schranke ist identisch mit der Schranke von Regret Matching.
CFR+ hat daher die gleiche theoretische Regret-Schranke wie CFR, konvergiert in der Praxis jedoch deutlich schneller.
Dies lässt sich durch die besseren Tracking-Regret-Eigenschaften von Regret Matching+ erklären.\\
Die empirische Überlegenheit von Regret Matching+ zeigt sich insbesondere dann, wenn sich die beste Aktion plötzlich ändert.
Bei Regret Matching muss der Algorithmus warten, bis eine zuvor schlechte Aktion sich beweist, und muss dabei ihren gesamten akkumulierten negativen Regret überwinden.
Regret Matching+ kann dagegen sofort die neue beste Aktion spielen, da negative Regrets auf null zurückgesetzt werden und nicht erst überwunden werden müssen.
\subsection{Tracking Regret}

Tracking Regret ist ein von ~\cite{herbster1998tracking} eingeführtes Konzept.
Externes Regret vergleicht nur gegen statische Strategien, die immer die gleiche Aktion wählen.
Tracking Regret erlaubt dagegen den Vergleich gegen Strategien, die ihre Aktion höchstens $(k-1)$ Mal ändern können.

Regret Matching hat sehr schlechte Tracking-Regret-Eigenschaften.
Der Algorithmus kann lineares Regret aufweisen, das proportional zur Anzahl der Iterationen $T$ wächst, selbst wenn die Vergleichsstrategie nur einmal ihre Aktion ändert ($k = 2$).
Im Gegensatz dazu ist Regret Matching+ der erste Regret-Matching-basierte Algorithmus mit sublinearem Tracking-Regret.

Für alternative Strategien, die bis zu $(k-1)$ Mal wechseln können, hat Regret Matching+ eine Regret-Schranke von
\begin{equation}
R^T \le kL\sqrt{|A|T}.
\end{equation}
(\cite[Thm.~2]{bowling2015heads})

Diese Schranke zeigt, dass der Regret linear mit der Anzahl der Partitionen $k$ skaliert, aber sublinear mit der Anzahl der Iterationen $T$ bleibt.

\subsection{Linear Weighted Average}

CFR+ verwendet eine linear gewichtete Durchschnittsstrategie anstelle der uniformen Durchschnittsstrategie von CFR.
Dabei wird jede Iteration $t$ mit Gewicht $t$ versehen, wodurch spätere Iterationen stärker gewichtet werden.
Die linear gewichtete Durchschnittsstrategie ist definiert als:
\begin{equation}
\bar{\sigma}_p^T = \frac{2}{T^2 + T} \sum_{t=1}^T t\sigma_p^t.
\end{equation}

~\cite[Thm.~3]{bowling2015heads} besagt Folgendes:
Wenn CFR+ für $T$ Iterationen in einem extensiven Spiel läuft, dann ist die linear gewichtete Durchschnittsstrategie ein approximatives Nash-Gleichgewicht.
Voraussetzung ist, dass die maximale Differenz zwischen den Auszahlungen für jeden Spieler durch eine Konstante $L$ beschränkt ist.
Die Qualität der Approximation wird durch die Schranke $2(|\mathcal{I}_1| + |\mathcal{I}_2|)L\sqrt{k}/\sqrt{T}$ beschrieben.
Dabei bezeichnet $k = \max_{I \in \mathcal{I}} |A(I)|$ die maximale Anzahl von Aktionen pro Informationset.
Die Schranke impliziert eine Konvergenzrate von $O(1/\sqrt{T})$, die asymptotisch identisch mit der Rate der uniformen Gewichtung ist.

Theorem~3 gilt nicht für CFR.
Im Gegensatz zu CFR+, wo lineare Gewichtung die Performance verbessert, verschlechtert lineare Gewichtung die Performance von Vanilla CFR.

In der Praxis erreicht die aktuelle Strategie in CFR+ oft bereits eine gute Approximation zum Nash Equilibrium, sodass die aggressive Gewichtung die Konvergenz weiter beschleunigen kann.

Empirisch wurde beobachtet, dass eine quadratische Gewichtung mit $t^2$ statt $t$ die Konvergenz von CFR+ weiter beschleunigen kann~\cite{brown2018discounted}.

%maybe hier noch mehr zu squared weight sagen idk




\section{Discounted CFR}

Discounted CFR (DCFR) wurde im Paper \enquote{Solving Imperfect-Information Games via Discounted Regret Minimization}~\cite{brown2018discounted} als Optimierung von CFR vorgestellt.
Die Motivation für DCFR liegt darin, dass CFR+ in Spielen mit stark suboptimalen Aktionen relativ schlecht abschneidet.
Ein anschauliches Beispiel hierfür ist der Fall, dass ein Spieler zwischen drei Aktionen mit den Payoffs 0, 1 und -1.000.000 wählen muss.
CFR+ benötigt in diesem Fall 471.407 Iterationen, um die beste Aktion mit 100\% Wahrscheinlichkeit zu wählen, da der frühe Fehler der ersten Iteration lange im kumulierten Regretwert erhalten bleibt.
Durch die Abwertung früherer Iterationen können solche Fehler schneller aus dem Regretwert verschwinden, wodurch die Konvergenz beschleunigt wird.

Der Algorithmus unterscheidet sich in drei Punkten von CFR+:
\begin{itemize}
    \item Frühere Iterationen werden bei der Regret-Akkumulation abgewertet (Discounting), wobei positive und negative Regretwerte unterschiedlich behandelt werden.
    \item Iterationen werden bei der Berechnung der Durchschnittsstrategie unterschiedlich gewichtet.
    \item Optimistische Regret-Matching-Algorithmen können verwendet werden.
\end{itemize}

\subsection{Linear CFR}
Linear CFR (LCFR) ist ein Vorläufer von DCFR, der das Konzept des Regret Discountings einführt.
In LCFR werden die Regretwerte nach jedem Update mit dem Faktor $\frac{t}{t+1}$ multipliziert, wodurch frühere Iterationen weniger stark gewichtet werden.
Zusätzlich wird die Durchschnittsstrategie mit Gewicht $t$ in Iteration $t$ gewichtet.
Im oben genannten Beispiel benötigt LCFR nur 970 Iterationen im Gegensatz zu den 471.407 Iterationen bei CFR+, um mit einhundert Prozent Wahrscheinlichkeit die richtige Aktion zu wählen.
DCFR übernimmt dieses Konzept und erweitert es um differenziertes Discounting für positive und negative Regretwerte.

\subsection{Discounting}

Discounting der Regretwerte bedeutet, dass frühere Iterationen weniger stark gewichtet werden.
Dies beschleunigt die Konvergenz insbesondere in Spielen mit stark suboptimalen Aktionen, da frühe Fehler schneller aus dem kumulierten Regretwert verschwinden.
Positive und negative Regretwerte erfahren hierbei eine gesonderte Behandlung.
Nach jedem Update werden positive Regretwerte mit Faktor $\frac{t^\alpha}{t^\alpha + 1}$ multipliziert, negative Regretwerte mit $\frac{t^\beta}{t^\beta + 1}$.
Formal werden die Regretwerte in Iteration $t$ wie folgt aktualisiert:
\begin{equation}
R^t(I,a) = \begin{cases}
(R^{t-1}(I,a) + r^t(I,a)) \cdot \frac{t^\alpha}{t^\alpha + 1} & \text{falls } R^{t-1}(I,a) + r^t(I,a) > 0 \\
(R^{t-1}(I,a) + r^t(I,a)) \cdot \frac{t^\beta}{t^\beta + 1} & \text{sonst}
\end{cases}
\end{equation}
Die Parameter $\alpha$ und $\beta$ sind frei wählbar, das Paper schlägt allerdings $\alpha = 3/2$ und $\beta = 0$ vor.

DCFR hat eine Konvergenzschranke von $O(1/\sqrt{T})$, die sich von CFR nur durch einen konstanten Faktor unterscheidet.

\subsection{Gewichtung der Durchschnittsstrategie}

DCFR verwendet wie CFR+ eine gewichtete Durchschnittsstrategie, wobei die Gewichtung durch den Parameter $\gamma$ gesteuert wird.
Iteration $t$ trägt mit Gewicht $t^\gamma$ bei.
Dies ist unabhängig vom Regret Discounting, das die Regretwerte während des Trainings beeinflusst.
Die Gewichtung der Durchschnittsstrategie bestimmt, wie stark jede Iteration zur finalen Output-Strategie beiträgt.

\subsection{Standard-Parameter}

DCFR wird durch drei Parameter charakterisiert: 
$\alpha$ für das Discounting positiver Regretwerte,
$\beta$ für das Discounting negativer Regretwerte und 
$\gamma$ für die Gewichtung der durchschnittlichen Strategie.\\
Die empfohlene Standard-Parameterkonfiguration ist DCFR$_{3/2,0,2}$ mit $\alpha = 3/2$, $\beta = 0$ und $\gamma = 2$.\\
Experimentelle Ergebnisse zeigen, dass diese Konfiguration CFR+ in allen im Paper getesteten Spielen übertrifft.

Bei $\beta = 0$ gehen negative Regretwerte nicht gegen $-\infty$, sondern nähern sich einem konstanten Wert an.
Dies macht DCFR$_{3/2,0,2}$ nicht kompatibel mit Pruning-Algorithmen, die auf negativen Regretwerten basieren.
Für Anwendungen, die Pruning verwenden, wird DCFR$_{3/2,1/2,2}$ mit $\beta = 0.5$ empfohlen, da diese Konfiguration suboptimale Aktionen erlaubt, deren Regret gegen $-\infty$ geht, und damit Pruning ermöglicht.

\subsection{Optimistisches Regret Matching}

Optimistisches Regret Matching ist eine optionale Erweiterung, bei der die letzte Iteration doppelt gezählt wird, wenn die Strategie für die nächste Iteration berechnet wird.
Formal wird dabei ein modifizierter Regretwert $R_{\text{mod}}^T(I,a) = \sum_{t=1}^{T-1} r^t(I,a) + 2r^T(I,a)$ verwendet, der die letzte Iteration zweifach berücksichtigt.
Experimentelle Ergebnisse zeigen, dass Optimistic DCFR$_{3/2,0,2}$ in allen getesteten HUNL-Subgames schlechter abschneidet als normales DCFR$_{3/2,0,2}$.
Optimistic LCFR kann hingegen in Spielen mit besonders großen suboptimalen Aktionen zu schnellerer Konvergenz führen als LCFR.



\section{Exploitability}\label{sec:exploitability}
Zur Evaluation der Strategiequalität wird die Metrik Exploitability verwendet.
Als theoretische Grundlage für dieses Kapitel dient \cite{johanson2011accelerating}.
Zur Berechnung der Exploitability wird die Best-Response-Methode genutzt.
Eine Best Response ist die optimale Strategie eines Spielers gegen eine gegebene Gegnerstrategie.
Formal ist die Best Response von Spieler $i$ gegen die Gegnerstrategie $\sigma_{-i}$ definiert als:
\begin{equation}
b_i(\sigma_{-i}) = \arg\max_{\sigma'_i \in \Sigma_i} u_i(\sigma'_i, \sigma_{-i}),
\end{equation}
wobei $\Sigma_i$ die Menge aller möglichen Strategien von Spieler $i$ bezeichnet.\\
Der Wert $u_i(\sigma_i, b_{-i}(\sigma_i))$ ist eine untere Schranke für den erwarteten Nutzen von Spieler $i$.
Die Exploitability einer Strategie misst, wie viel Nutzen gegen einen Gegner verloren geht, der die Best-Response-Strategie spielt, verglichen mit dem Wert, der von einer optimalen Strategie erreicht werden würde.


In einem Zweipersonen-Nullsummenspiel ist die Exploitability einer Strategie $\sigma_i$ definiert als:
\begin{equation}
\varepsilon_i(\sigma_i) = v_i - u_i(\sigma_i, b_{-i}(\sigma_i)),
\end{equation}
wobei $v_i$ der Spielwert für Spieler $i$ ist (die untere Schranke für den Nutzen eines optimalen Spielers in Position $i$) und $b_{-i}(\sigma_i)$ die Best Response des Gegners gegen $\sigma_i$ darstellt.
Eine Strategie ist optimal, wenn ihre Exploitability null ist.

Für Strategien, die in beiden Positionen gespielt werden, wird die Exploitability als Durchschnitt der Best-Response-Werte aus beiden Positionen berechnet:
\begin{equation}
\varepsilon(\sigma) = \frac{u_2(\sigma_1, b_2(\sigma_1)) + u_1(b_1(\sigma_2), \sigma_2)}{2}.
\end{equation}

In Nullsummenspielen ist die Exploitability eng mit dem Nash-Gleichgewicht verbunden.
Ein Nash-Gleichgewicht ist unexploitable, da es gegen jeden Gegner mindestens den Spielwert erreicht.
Daher ist die Exploitability eine zentrale Evaluationsmetrik.

\subsection{Konventionelle Best Response Berechnung}

Die konventionelle Berechnung einer Best Response erfolgt durch eine rekursive Traversierung des Informationset Trees des betrachteten Spielers.
Der Algorithmus traversiert den Baum von der Wurzel zu den Terminalknoten und berechnet dabei für jedes Informationset den erwarteten Nutzen.

An Terminalknoten muss der Algorithmus alle möglichen Spielzustände berücksichtigen, die für den Spieler nicht unterscheidbar sind.
Da der Spieler nicht weiß, in welchem konkreten Spielzustand er sich befindet, berechnet er die Erreichbarkeitswahrscheinlichkeiten des Gegners für die verschiedenen Informationsets.
Der Nutzen wird als gewichtete Summe über alle möglichen Spielzustände berechnet, wobei jeder Nutzen mit der entsprechenden Erreichbarkeitswahrscheinlichkeit gewichtet wird.
Dieser Wert wird zurückgegeben.

Während der Rücktraversierung durch den Baum werden an den verschiedenen Knotentypen unterschiedliche Operationen durchgeführt.
An Entscheidungsknoten des betrachteten Spielers wählt der Algorithmus die Aktion mit dem höchsten erwarteten Nutzen und speichert diese Wahl als Teil der Best-Response-Strategie.
Der Wert dieser gewählten Aktion wird zurückgegeben.
An Entscheidungsknoten des Gegners und an Zufallsknoten wird der erwartete Nutzen als Summe der Werte der Kindknoten berechnet und zurückgegeben.
Wenn der Algorithmus zur Wurzel zurückkehrt, ist der zurückgegebene Wert der Best-Response-Wert gegen die gegebene Gegnerstrategie.

\begin{figure}[htbp]
    \centering
    \includegraphics[width=0.8\textwidth]{chapters/02_theoretische_grundlagen/sections/05_exploitability/subsections/03_public_state_tree/gametree.png}
    \caption{Game Tree für Kuhn Poker}
    \label{fig:br_game_tree}
\end{figure}

\begin{figure}[htbp]
    \centering
    \includegraphics[width=0.8\textwidth]{chapters/02_theoretische_grundlagen/sections/05_exploitability/subsections/03_public_state_tree/informationset_tree.png}
    \caption{Information Set Trees für beide Spieler in  Kuhn Poker}
    \label{fig:br_infoset_tree}
\end{figure}

Die Abbildungen zeigen das Beispiel Kuhn Poker zur Illustration des Algorithmus.
Im Game Tree (Abbildung~\ref{fig:br_game_tree}) repräsentieren die Knoten \emph{A,X}, \emph{A,Y}, \emph{B,X} und \emph{B,Y} die terminalen Spielzustände, wobei \emph{A} und \emph{B} die Aktionen von Spieler 1 und \emph{X} und \emph{Y} die Aktionen von Spieler 2 bezeichnen.
Im Information Set Tree von Spieler 2 (Abbildung~\ref{fig:br_infoset_tree}) kann dieser am Terminalknoten \emph{X} nicht unterscheiden, ob er sich im Game Tree Zustand \emph{A,X} oder \emph{B,X} befindet, da diese nur durch die private Information von Spieler 1 unterschieden werden.
Daher muss der Algorithmus die Erreichbarkeitswahrscheinlichkeiten für beide möglichen Zustände berechnen.

~\cite{johanson2011accelerating} beschreiben eine beschleunigte Version des Algorithmus, die vier Optimierungen umfasst und im Folgenden erläutert wird.

\input{chapters/02_theoretische_grundlagen/sections/05_exploitability/subsections/02_accelerated_best_response/02_accelerated_best_response}
\subsection{Public State Tree}

Der zentrale Ansatz der beschleunigten Best-Response-Berechnung besteht darin, statt des Informationset Trees einen \emph{Public State Tree} zu traversieren.
Ein \emph{Public State} ist definiert als eine Partition der Historien, die folgende Eigenschaften erfüllt:
Keine zwei Historien aus demselben Informationset befinden sich in unterschiedlichen Public States.
Zwei Historien aus unterschiedlichen Public States haben keine Nachfahren im selben Public State, wodurch eine Baumstruktur entsteht.
Kein Public State enthält sowohl terminale als auch nicht-terminale Historien.

Informell ist ein Public State durch alle Informationen definiert, die beiden Spielern bekannt sind, also wie das Spiel für einen Beobachter ohne private Informationen aussieht.
Der Public State Tree ist deutlich kleiner als der Informationset Tree, da er nur die öffentlich sichtbaren Informationen enthält.

Der entscheidende Vorteil bei der Traversierung des Public State Trees liegt in der Wiederverwendung von Erreichbarkeitswahrscheinlichkeiten.
In der konventionellen Methode werden die Erreichbarkeitswahrscheinlichkeiten des Gegners an jedem Terminalknoten neu berechnet.
Im Public State Tree werden diese Wahrscheinlichkeiten jedoch einmal berechnet und für alle Informationsets des betrachteten Spielers in diesem Public State wiederverwendet.
Dies ist möglich, weil die Erreichbarkeitswahrscheinlichkeiten für alle eigenen Informationsets identisch sind, die sich im selben Public State befinden.

Statt wie in der konventionellen Methode einen einzelnen Wert für ein Informationset zurückzugeben, gibt der Algorithmus beim Public State Tree einen Vektor von Werten zurück.
Einen für jedes Informationset des betrachteten Spielers im Public State.
An Terminalknoten werden die Werte für alle Informationsets berechnet, wobei die Erreichbarkeitswahrscheinlichkeiten des Gegners einmal berechnet und für alle eigenen Informationsets verwendet werden.
An Entscheidungsknoten des betrachteten Spielers wird für jedes Informationset die beste Aktion gewählt, und es wird ein Vektor zurückgegeben, dessen Einträge die maximalen Aktionswerte darstellen.
An Entscheidungsknoten des Gegners und an Zufallsknoten werden die Wertevektoren der Kindknoten summiert und zurückgegeben.
Diese Umstrukturierung führt zu denselben Berechnungen wie die konventionelle Methode, ändert jedoch die Reihenfolge, sodass Strategieabfragen beim Gegner effizienter wiederverwendet werden können.
In Spielen wie Texas Hold'em, in denen jeder Spieler bis zu 1326 Informationsets pro Public State haben kann, ermöglicht dies eine erhebliche Reduzierung der Strategieabfragen.

\begin{figure}[htbp]
    \centering
    \includegraphics[width=0.35\textwidth]{chapters/02_theoretische_grundlagen/sections/05_exploitability/subsections/03_public_state_tree/public_tree.png}
    \caption{Public State Tree für Kuhn Poker}
    \label{fig:public_state_tree}
\end{figure}

Abbildung~\ref{fig:public_state_tree} zeigt den Public State Tree für das Beispiel Kuhn Poker.
Im Vergleich zum Game Tree (Abbildung~\ref{fig:br_game_tree}) und den Information Set Trees (Abbildung~\ref{fig:br_infoset_tree}) ist der Public State Tree deutlich kompakter, da er nur die öffentlich sichtbaren Informationen enthält.
Der Terminalknoten \emph{A,B,X,Y} repräsentiert alle vier möglichen terminalen Spielzustände \emph{A,X}, \emph{A,Y}, \emph{B,X} und \emph{B,Y} aus dem Game Tree.
An diesem Terminalknoten können die Erreichbarkeitswahrscheinlichkeiten des Gegners einmal berechnet und für alle Informationsets beider Spieler in diesem Public State wiederverwendet werden.

\subsection{Effiziente Evaluation der Terminalknoten}

Ein weiterer Beschleunigungsschritt betrifft die Evaluierung von Terminalknoten.
Eine naive Methode hätte $O(n^2)$ Komplexität, wobei $n$ die Anzahl der Informationsets pro Spieler ist.
In Pokerspielen kann diese auf $O(n)$ reduziert werden, indem die Informationsets nach Handstärke sortiert werden.
Da der Nutzen nur von der relativen Ordnung der Handstärken abhängt, können Indizes in die sortierte Liste des Gegners verwendet werden, die markieren, wo sich die Relation ändert (schwächer, gleich, stärker).
Um ein Informationset zu evaluieren, benötigt man nur die Gesamtwahrscheinlichkeit der Gegner-Informationsets in diesen drei Abschnitten.
Beim Übergang zum nächsten stärkeren Informationset werden die Indizes nur um einen Schritt verschoben, anstatt alle Gegner-Informationsets erneut zu betrachten.

Da die möglichen Hände der Spieler abhängig voneinander sind (eine Karte kann nicht von beiden Spielern gehalten werden), wird das Inklusions-Exklusions-Prinzip verwendet.
Beim Berechnen der Wahrscheinlichkeit von besseren oder schlechteren Gegnerhänden müssen unmögliche Hände ausgeschlossen werden, da der Gegner keine der eigenen Karten halten kann.
Hände, die eine der eigenen Karten enthalten, werden subtrahiert.
Hände, die beide eigene Karten enthalten, werden wieder addiert, da sie sonst doppelt subtrahiert würden (einmal für jede Karte).

\subsection{Spielspezifische Isomorphismen}

Ein weiterer Beschleunigungsschritt nutzt strategisch äquivalente Aktionen oder Ereignisse.
In vielen Kartenspielen sind Karten mit gleichem Rang, aber unterschiedlicher Farbe strategisch äquivalent.
In Poker ist das zumindest solange der Fall, bis weitere Karten aufgedeckt werden.
Bevor der Flop ausgeteilt wird, ist eine Herz zwei genauso gut wie eine Kreuz zwei.
Solche Mengen äquivalenter Zufallsereignisse werden als \emph{isomorph} bezeichnet, und eine wird willkürlich als Repräsentant gewählt.

Wenn die zu evaluierende Strategie für jedes Mitglied jeder Menge isomorpher Historien gleich ist, kann die Größe des Public State Trees erheblich reduziert werden, indem nur die Repräsentanten betrachtet werden.
Beim Zurückkehren durch einen Zufallsknoten während der Traversierung muss der Nutzen eines Repräsentanten mit der Anzahl der isomorphen States gewichtet werden, die er repräsentiert.
In Texas Hold'em führt diese Reduktion zu einem Public-State-Tree, der 21,5-mal kleiner ist als das vollständige Spiel.

\subsection{Parallele Berechnung}

Ein vierter Beschleunigungsschritt nutzt die Möglichkeit, unabhängige Teilbäume des Public State Trees parallel zu lösen.
Public States können parallel gelöst werden, wenn keiner ein Nachfahre des anderen ist, da sie keine gemeinsamen Berechnungen haben.
Beispielsweise können beim Erreichen eines öffentlichen Zufallsereignisses alle Kinder parallel gelöst werden.


In diesem Abschnitt wurde die Hauptmetrik beschrieben, die verwendet wird, um die Qualität einer Strategie zu bewerten, die Exploitability.
Darüber hinaus wurde beschrieben, wie diese mithilfe eines Best Response Agents berechnet wird und was für Optimierungen in der Berechnung vorgenommen werden können.

\section{Hold'em-Spiele}
In diesem Abschnitt wird erklärt, wodurch sich Hold'em-Spiele als Untergruppe von Pokerspielen auszeichnen.
Ist die Grundstruktur von Hold'em-Spielen einmal definiert, lässt sie sich einfach erweitern.
Die Definition spezifischer Pokervarianten erfolgt im Anhang~\ref{anhang:spieldefinitionen}.\\
Diese Arbeit beschäftigt sich ausschließlich mit Heads-Up-Limit-Versionen von Hold'em-Spielen.
Heads-Up bedeutet, dass genau zwei Spieler teilnehmen.
Die Klassifikation Limit bezieht sich darauf, dass die Setzgrößen der Spieler auf vorher festgelegte Beträge beschränkt sind.
Diese beiden Einschränkungen reduzieren die Komplexität erheblich.
Die folgende Definition gilt für alle Heads-Up-Limit-Hold'em-Spiele~\cite{southey2005bayes}.

Eine Partie nennt man eine \emph{Hand}.
Eine Hand wiederum besteht aus mehreren Runden.
In der ersten Runde erhalten beide Spieler eine feste Anzahl privater Karten (Hole Cards).
In \emph{jeder} Runde können zusätzlich öffentliche Karten (Board Cards) aufgedeckt werden (auch null).\\
Nach dem Austeilen bzw. Aufdecken der Karten folgt jeweils eine Setzrunde, in der die Spieler abwechselnd handeln.
Zur Verfügung stehen drei Aktionen: \enquote{Fold, Call und Raise}.\\
Wählt ein Spieler die Aktion \emph{Fold}, endet die Hand sofort und der andere Spieler gewinnt den Pot.
Der Pot ist die Ansammlung aller Einsätze, die im Verlauf der Hand gesetzt wurden.\\
Bei \emph{Call} gleicht der Spieler den Einsatz des Gegners (der Beitrag kann null sein; dann spricht man von \emph{Check}).\\
Bei \emph{Raise} gleicht der Spieler den Gegnerzug und setzt zusätzlich den vorgeschriebenen Aufschlag.

Die Setzrunde endet, sobald ein Spieler \emph{foldet} (dann endet die Hand sofort) oder \emph{callt} (dann endet nur die Runde). 
Um endlose Raise-Ketten zu vermeiden, ist die Zahl der Raises pro Runde begrenzt.
Sobald das Limit erreicht ist, sind nur noch die Aktionen Call und Fold erlaubt.
Läuft die letzte Setzrunde ohne Fold aus, folgt ein \emph{Showdown}: Beide Spieler zeigen ihre Hole Cards und bilden aus Hole und Board Cards die bestmögliche Hand.
Die stärkere Hand gewinnt den Pot.
Die zulässigen Pokerhände und deren Rangfolge hängen von der Art des Spiels ab.
Eine Übersicht über die möglichen Pokerhände in Texas Hold'em findet sich im Anhang~\ref{anhang:pokerhaende}.
Die Rangfolge anderer Spiele ist in der Definition der einzelnen Spiele enthalten~\ref{anhang:spieldefinitionen}.

Im Folgenden werden die zentralen Bestandteile genannt, die relevant sind, um Heads-Up-Limit-Hold'em-Varianten zu beschreiben.
\begin{itemize}
    \item die Anzahl der Hole Cards und Board Cards
    \item die Anzahl der Setzrunden
    \item das Setzlimit pro Runde
    \item die im Deck enthaltenen Karten
    \item die möglichen Pokerhände
    \item die festgelegten Setzgrößen
\end{itemize}




In diesem Kapitel wurden die theoretischen Grundlagen der zentralen Komponenten dieser Arbeit dargestellt.
Das folgende Kapitel legt fest, welche Algorithmen, Evaluationsmethoden und Pokerspiele in der Implementierung tatsächlich umgesetzt werden.