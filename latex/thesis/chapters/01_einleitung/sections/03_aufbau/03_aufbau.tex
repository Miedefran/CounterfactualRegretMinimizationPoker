\section{Aufbau der Arbeit}
Die Arbeit gliedert sich in acht Kapitel, die den Weg von den theoretischen Grundlagen über die Architektur und den Entwicklungsverlauf bis hin zur Evaluation der trainierten Strategien beschreiben.

\begin{itemize}
\item \hyperref[chap:grundlagen]{\textbf{Kapitel 2}} stellt die theoretischen Grundlagen vor.
Zu Beginn werden grundlegende spieltheoretische Begriffe eingeführt und eine formale Definition eines \enquote{Extensiven Spiels mit imperfekter Information} gegeben.\\
Anschließend wird die Theorie des CFR-Algorithmus detailliert erläutert, gefolgt von den optimierten Varianten CFR+, MCCFR und DCFR.\\
Darauf folgt die Theorie des Accelerated Best Response Agents, der zur Evaluation der Strategiequalität verwendet wird.
Abschließend wird definiert, wodurch sich ein Hold'em Spiel auszeichnet.
Die Definition einzelner Pokervarianten erfolgt im Anhang.

\item \hyperref[chap:analyse]{\textbf{Kapitel 3}} nennt die zu lösenden Pokerspiele, die dazu verwendeten Algorithmen und begründet die Auswahl.

\item \hyperref[chap:architektur]{\textbf{Kapitel 4}} beschreibt zunächst die generelle Systemarchitektur und geht anschließend auf einzelne Komponenten ein.

\item \hyperref[chap:entwicklungsvorgang]{\textbf{Kapitel 5}} liefert einen Überblick über den Entwicklungsverlauf und hebt dabei Probleme sowie die daraus gewonnenen Erkenntnisse hervor.

\item \hyperref[chap:evaluation]{\textbf{Kapitel 6}} veranschaulicht den Konvergenzverlauf der trainierten Strategien und diskutiert die Ergebnisse.

\item \hyperref[chap:fazit]{\textbf{Kapitel 7}} fasst das Gelernte hinsichtlich der zu \hyperref[sec:zielsetzung]{Beginn} definierten Hypothesen zusammen, diskutiert Limitationen und gibt einen Ausblick auf mögliche zukünftige Erweiterungen.

\item \hyperref[chap:anhang]{\textbf{Kapitel 8}} ist der Anhang und enthält alles, was es nicht in den Hauptteil der Arbeit geschafft hat.
\end{itemize}