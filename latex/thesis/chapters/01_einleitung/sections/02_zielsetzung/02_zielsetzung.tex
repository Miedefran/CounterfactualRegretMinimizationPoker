\section{Zielsetzung und Fragestellungen}\label{sec:zielsetzung}

Im Rahmen dieser Arbeit wird Counterfactual Regret Minimization in Python implementiert und auf verschiedenen Pokervarianten unterschiedlicher Komplexität angewendet.
Das Projekt ist unter der URL \\ \url{https://github.com/Miedefran/CounterfactualRegretMinimizationPoker} zu finden.
Dabei wird untersucht, wie sich die Leistungsfähigkeit des Algorithmus bei steigender Spielkomplexität verhält.
Wichtige Parameter, um die Leistungsfähigkeit einzuschätzen, sind die Laufzeit, der Speicherverbrauch und die Qualität der trainierten Strategie.
Durch Betrachtung verschiedener Komplexitätsstufen soll untersucht werden, ab welchem Komplexitätsgrad algorithmische und implementierungstechnische Optimierungen vorgenommen werden müssen.

Im Exposé dieser Arbeit waren drei Spiele geplant, Kuhn Poker als simpelstes existierendes Pokerspiel, Leduc Hold'em als etabliertes Benchmark-Spiel und eine selbst entwickelte Variante mit 12 Karten als nächste Stufe auf der Komplexitätsleiter.\\
Zusätzlich sollten optimierte CFR-Verfahren wie CFR+ und Monte Carlo CFR implementiert werden und auf ihr Konvergenzverhalten untersucht werden.\\
Um die Qualität einer trainierten Strategie zu bewerten, wird berechnet, wie ausnutzbar diese ist.
Dies geschieht mithilfe eines Best Response Agents, der die optimale Antwort gegen eine Gegnerstrategie spielt.
Im Verlauf des Projekts wurden diese Ziele grundsätzlich beibehalten, jedoch wurde Monte Carlo CFR durch Discounted CFR ersetzt.

Im Folgenden werden drei Forschungsfragen formuliert, die im Verlauf des Projekts beantwortet werden sollen.
\begin{enumerate}
    \item \textbf{Algorithmusvergleich:} 
    Wie unterscheiden sich die verschiedenen CFR-Algorithmen in ihrem Konvergenzverhalten bei steigender Spielgröße, und ab welcher Spielgröße ist Vanilla CFR nicht mehr praktikabel?    
    \item \textbf{Implementierungsoptimierungen:} 
    Welche Implementierungsoptimierungen sind erforderlich, damit das Training bei wachsender Spielgröße praktikabel bleibt?
    \item \textbf{Limitationen:} 
    Durch welche Komponenten wird die geplante Implementierung limitiert?
\end{enumerate}

