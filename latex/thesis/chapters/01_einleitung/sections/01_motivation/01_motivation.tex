\section{Motivation und Relevanz}
%Kurzer Einleitungspart, der Interesse wecken soll

Im Alltag werden ständig Entscheidungen getroffen, ohne dass Entscheidungsträger stets vollständig über alle relevanten Umstände informiert sind.
Um dennoch handlungsfähig zu bleiben, treffen Menschen bewusst oder unterbewusst Annahmen über unbekannte Aspekte einer Situation.
Das Verhalten von Akteuren in solchen strategischen Entscheidungssituationen lässt sich durch spieltheoretische Modelle beschreiben.
Die Spieltheorie umfasst sowohl Modelle für Situationen bzw. Spiele mit vollständiger als auch mit unvollständiger Information.
Ein prominentes Beispiel für ein Spiel mit unvollständiger Information ist Poker, insbesondere Hold'em-Varianten wie Texas Hold'em oder Omaha Hold'em.

%Eigenes Interesse + GTO Populär 
Ich habe im letzten Jahr ein ausgeprägtes Interesse an dem Spiel No-Limit Texas Hold'em entwickelt. 
Wenn man sich mit Strategien dieses Spiels auseinandersetzt, stößt man schnell auf ein Konzept, das „Game Theory Optimized“ (GTO) genannt wird.
Dieses Konzept hat innerhalb der letzten Jahre immer mehr an Aufmerksamkeit und Relevanz gewonnen und ist heutzutage kaum noch aus der Pokerwelt wegzudenken.\\
Eine GTO-Strategie ist im Grunde eine spieltheoretisch optimale Lösung, die durch den Einsatz von Algorithmen approximiert wird. 
Diese Strategien sind für Spiele mit einem enorm großen Zustandsraum, wie No-Limit Texas Hold'em, so komplex, dass sie von Menschen nicht vollständig erlernt und angewendet werden können.
Pokerprofis sind allerdings in der Lage, diese Strategien in abstrahierter Form zu adaptieren, indem sie grundlegende strategische Prinzipien und Verhaltensmuster erkennen und anschließend auf ihre Spielsituationen anwenden.
Die Auseinandersetzung mit GTO-Strategien in meiner Freizeit weckte mein Interesse an den algorithmischen Grundlagen, die zur Berechnung solcher Strategien verwendet werden.

%Übergang GTO zu CFR
Ein zentraler Algorithmus zum Finden solcher Strategien ist Counterfactual Regret Minimization (CFR).
Seit seiner Einführung in \enquote{Regret Minimization in Games with Incomplete Information} \cite{zinkevich2007regret} hat sich CFR als Standardansatz etabliert und bildet die Grundlage für die erfolgreichsten Poker-KIs.
\newpage
Zu den größten Erfolgen gehören:
\begin{itemize}
    \item Cepheus \cite{bowling2015heads} löste erstmals das Spiel Heads-Up Limit Texas Hold'em.
    \item Libratus \cite{brown2017superhuman,brown2017libratusDemo} erweiterte diesen Erfolg dann auf die Heads-Up No-Limit-Variante.
    \item Durch Pluribus \cite{brown2019superhuman} wurde gezeigt, dass diese Erfolge auch auf Mehrspielerszenarien übertragen werden können.
\end{itemize}




