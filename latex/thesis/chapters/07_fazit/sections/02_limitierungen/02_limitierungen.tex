\section{Limitierungen}

In diesem Abschnitt werden die wichtigsten Limitierungen der Arbeit formuliert und die Aussagekraft der Ergebnisse eingeordnet.
Die Punkte sind nicht als Entschuldigung zu verstehen, sondern als klare Beschreibung der Bedingungen, unter denen die gezeigten Resultate gelten.

Die Skalierungsgrenzen unterscheiden sich je nach Implementierungsansatz.
Beim Dynamic Ansatz wird die Laufzeit pro Iteration mit zunehmender Spielgröße so hoch, dass Training in vertretbarer Zeit nicht mehr möglich ist.
Beim Flat Tree Ansatz bleibt die Laufzeit pro Iteration konstant, jedoch wird der Speicherbedarf zur Grenze.
Tree basierte Solver reduzieren Overhead durch Zustandsübergänge, benötigen aber Speicher für den vollständigen Game Tree.
Bei großen Varianten kann dadurch nicht nur Training, sondern bereits das Bauen und Halten der Baumstruktur praktisch begrenzt werden.
Dies zeigt sich insbesondere bei Small Island Hold'em, wo nur noch Flat Tree Implementierungen praktikabel sind.

Exploitability ist nur so zuverlässig wie die Implementierung der Best Response Berechnung und des Public State Trees.
Fehler oder Inkonsistenzen in dieser Pipeline würden sich direkt auf die gemessene Exploitability auswirken.
Als Plausibilitätscheck kann für Leduc zusätzlich der erwartete Spielwert betrachtet werden, bei dem sich ein konsistentes Zero-Sum-Verhalten zeigt.

Für größere Spiele sind Best-Response-Messpunkte nur sparsam möglich, weil die Evaluation zusätzliche Laufzeit verursacht.
Zeitplots sind grundsätzlich als Orientierung zu interpretieren, weil Hintergrundlast und Hardwareunterschiede die absolute Wall-Clock-Time beeinflussen können.
Die betrachteten Solver mit Volltraversierung sind weitgehend deterministisch, weshalb keine Varianzabschätzung über Seeds im klassischen Sinn erfolgt.
