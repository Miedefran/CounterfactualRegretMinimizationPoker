\section{Best Response Agent}

Der Best Response Agent berechnet die Exploitability einer Strategie durch Traversierung des Public State Trees.
In diesem Abschnitt wird dieser Ablauf mittels UML Diagrammen beschrieben.

Der Public State Tree wird einmalig gebaut und anschließend gespeichert.

\subsection{Gesamtablauf der Best-Response-Berechnung}

Die Berechnung beginnt mit dem Laden des Public State Trees und der durchschnittlichen Strategie.
Für jedes Informationset des betrachteten Spielers werden die gültigen Gegner-Informationsets bestimmt (ausschließlich solche mit unterschiedlichen Karten).
Die Erreichbarkeitswahrscheinlichkeiten für den Gegner werden berechnet, und anschließend wird der Public State Tree rekursiv traversiert.
Der Gesamtnutzen wird als gewichtete Summe über alle Informationsets berechnet und als Exploitability zurückgegeben.

\begin{figure}[htbp]
\centering
\includegraphics[width=0.4\textwidth]{chapters/04_systemarchitektur/sections/03_evaluationsmethoden/Aktivitaetsdiagramm_best_response.pdf}
\caption{Aktivitätsdiagramm der Best-Response-Berechnung}
\label{fig:best_response_activity}
\end{figure}

\newpage
\subsection{Details der rekursiven Traversierung}

Die rekursive Traversierung des Public-State-Trees unterscheidet zwischen vier Knotentypen.
Handelt es sich um einen \textbf{Terminalknoten}, wird die Hand-Strength für alle Informationsets berechnet (mit Caching zur Performance-Optimierung), und die Payoffs werden als gewichtete Summe über die Erreichbarkeitswahrscheinlichkeiten des Gegners berechnet.

Bei einem \textbf{Zufallsknoten} wird für jedes Zufallsereignis der neue Public State bestimmt und rekursiv weiter traversiert.
Der erwartete Nutzen wird als gewichtete Summe über alle Zufallsereignisse berechnet.

Bei einem \textbf{Entscheidungsknoten} wird unterschieden, ob es sich um einen Knoten des betrachteten Spielers oder des Gegners handelt.
Handelt es sich um einen Knoten des \textbf{betrachteten Spielers}, wird für jede mögliche Aktion rekursiv traversiert und der Nutzen gespeichert.
Für jedes Informationset wird anschließend die Best Response berechnet, indem das Maximum über alle Aktionen gewählt wird.

Handelt es sich um einen \textbf{Gegner-Knoten}, wird die Strategie des Gegners für die entsprechenden Informationsets abgerufen.
Für jede mögliche Aktion werden die Erreichbarkeitswahrscheinlichkeiten aktualisiert und rekursiv weiter traversiert.
Der erwartete Nutzen wird als gewichtete Summe über alle Aktionen berechnet, wobei die Gewichte die Strategie-Wahrscheinlichkeiten des Gegners sind.

Die Terminalknoten-Evaluierung nutzt Hand-Strength-Sortierung für O(n) statt O(n²) Komplexität, und Hand-Strength-Evaluierungen werden gecacht, um wiederholte Berechnungen zu vermeiden.

\begin{figure}[htbp]
\centering
\includegraphics[width=1.0\textwidth]{chapters/04_systemarchitektur/sections/03_evaluationsmethoden/Aktivitaetsdiagramm_traverse_public_tree.pdf}
\caption{Aktivitätsdiagramm der Public-State-Tree-Traversierung}
\label{fig:traverse_public_tree}
\end{figure}

