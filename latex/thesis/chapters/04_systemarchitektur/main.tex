\chapter{Architektur und Design}\label{chap:architektur}
In diesem Kapitel wird die Architektur des Softwareprojekts beschrieben.
Hierbei wird zuerst ein Überblick darüber gegeben, wie die einzelnen Bestandteile miteinander interagieren.
Anschließend wird auf die Struktur der einzelnen Komponenten eingegangen und die wichtigsten von ihnen mit UML-Diagrammen dargestellt.
Kritische Designentscheidungen werden erklärt und Probleme, die durch diese entstanden sind, werden erwähnt.
Die Probleme und deren Konsequenzen werden anschließend in \autoref{chap:entwicklungsvorgang} genauer erläutert.

Eine zentrale Designentscheidung, die schon zu Beginn getroffen wurde, ist eine explizite Trennung der Pokerspiellogik vom CFR-Algorithmus.
Die Logik der Pokerspiele, ab jetzt Game Environment genannt, wird als separate Entität implementiert, die das Spielverhalten und die Zustandsverwaltung übernimmt.
Ziel dieser Unterteilung war es, das Projekt später einfach erweiterbar für neue Spiele und Algorithmen zu halten.

Dieses Kapitel gliedert sich in drei Hauptkomponenten: Game Environment, CFR-Solver und Evaluationsmethoden.
Außerdem werden zwei verschiedene Arten der Implementierung des CFR-Solvers besprochen.

\paragraph{Eigener und verwendeter Code}
Die logische Struktur des Game Environment (Aufteilung in die Klassen Dealer, Player, Judger, Game und Round) wurde von der Bibliothek RL Card (\url{https://rlcard.org}) übernommen.
Die CFR-Solver wurden initial anhand der zugrunde liegenden Paper implementiert.
Zu diesem Entwicklungszeitpunkt wurde noch kaum auf Agentic Coding zurückgegriffen.
OpenSpiel \cite{lanctot2019openspiel} diente später als Referenz zum Auffinden von Fehlern und für strukturelle Optimierungen.
Ein Beispiel hierfür ist die Realisierung der CFR+ und DCFR Operationen mit Hooks nach jeder Traversierung.
In späteren Entwicklungsphasen, vor allem bei dem Layer-basierten Ansatz wurde Agentic Coding deutlich mehr verwendet.
Die TUI (Poker CFR Manager) verwendet die Bibliothek Memray von Bloomberg \url{https://github.com/bloomberg/memray}.
Das GUI stammt aus dem Modul Advanced Python und wurde nicht im Rahmen dieser Arbeit implementiert.

\newpage
\section{Game Environment}
In diesem Abschnitt wird der Aufbau der Spielumgebung mithilfe eines Klassendiagramms beschrieben.
Die Spielumgebung ist für die Logik des Pokerspiels sowie die Verwaltung des Spielzustands zuständig.
Eine Spielumgebung besteht aus den folgenden Komponenten: \texttt{Player} (repräsentiert die beiden Spieler), \texttt{Dealer} (verwaltet das Kartendeck), \texttt{Judger} (bestimmt den Gewinner) und \texttt{Game} (koordiniert die anderen Komponenten).
Bei allen Varianten außer Kuhn Poker wird diese Architektur zusätzlich durch eine \texttt{Round}-Klasse erweitert, die den Übergang zwischen Setzrunden steuert.
Diese Unterteilung der Funktionalitäten der Spielumgebung wurde von RLCard übernommen(\url{https://rlcard.org}).

Abbildung~\ref{fig:gameenv_class} zeigt das Klassendiagramm der Game Environment für die Pokervarianten Kuhn Poker und Leduc Hold'em.
Alle weiteren Spielvarianten basieren auf Leduc Hold'em und erweitern diese um zusätzliche Funktionalitäten.
\begin{figure}[htbp]
\centering
\includegraphics[width=1.0\textwidth]{chapters/04_systemarchitektur/sections/01_game_environment/Klassendiagramm_gameenvs_simplified.pdf}
\caption{Klassendiagramm der Game Environment für Kuhn Poker und Leduc Hold'em}
\label{fig:gameenv_class}
\end{figure}


\section{CFR-Solver}
In dieser Arbeit werden vier Varianten von CFR-Algorithmen implementiert: Vanilla CFR, CFR+, DCFR und MCCFR.
Die Auswahl der Varianten CFR, CFR+ und MCCFR erfolgte zu Beginn der Arbeit auf Basis einer ersten Literaturrecherche. 
Diese Varianten wurden in der zu diesem Zeitpunkt bekannten Literatur am häufigsten diskutiert und bilden die Grundlage für viele erfolgreiche Anwendungen.

Vanilla CFR dient als Baseline für den Vergleich.
MCCFR wurde ausgewählt, um eine Variante zu untersuchen, die nicht den gesamten Spielbaum explizit durchläuft, sondern auf Sampling basiert. 
Als drittes wird CFR+ implementiert, da es die Grundlage für erfolgreiche Poker-KIs wie Cepheus oder Libratus bildet.\\
Aufgrund von Problemen bei der Evaluation der MCCFR Ergebnisse wurde im Verlauf der Arbeit DCFR als weitere Optimierung hinzugefügt, um genügend Daten für eine sinnvolle Evaluation zu haben.

\section{Evaluationsmethoden}
Die Evaluation erfolgt mittels Exploitability-Tests mit einem Best Response Agent.
Diese Methode ermöglicht es, die Qualität der trainierten Strategien zu bewerten, indem ein optimaler Gegner simuliert wird, der die Strategie maximal ausnutzt.
Um das Konvergenzverhalten eines Algorithmus einschätzen zu können, wird während des Trainings in regelmäßigen Abständen die Exploitability bestimmt.
Nach dem Training wird diese dann in Abhängigkeit von der Anzahl der Iterationen oder der vergangenen Zeit in einem Diagramm dargestellt.

Eine weitere Metrik, die ursprünglich geplant war, ist der Erwartungswert einer Strategie im Self-Play.
Hierbei spielt eine Strategie wiederholt gegen sich selbst, wodurch im Falle eines Nash-Equilibriums der Erwartungswert des Spiels berechnet werden kann.
Bei Forschungsvarianten wie Kuhn Poker und Leduc Hold'em ist dieser Wert bekannt und kann daher zur Verifikation der Implementierung genutzt werden.
Im Verlauf der Arbeit hat sich jedoch herausgestellt, dass diese Metrik nicht aussagekräftig genug ist und außerdem implizit in der Best Response Berechnung enthalten ist.

\newpage

\paragraph{Text-based User Interface}
Die Anwendung bietet eine textbasierte Benutzeroberfläche (TUI), den \emph{Poker CFR Manager}, zum Verwalten von Trainingsläufen, Anzeigen des Speicherverbrauchs sowie starten des GUI.
Das Feature wurde gegen Ende des Projekts umgesetzt und ist daher nicht in allen Teilen ausgereift.
Es erleichtert jedoch Nutzern des Projekts den Trainingsvorgang deutlich.
Während der Entwicklung erfolgte das Training über die Kommandozeile mit vielen Parametern als Flags, was durch die TUI vereinfacht wird.
Die TUI gliedert sich in vier Tabs.

Im Tab \emph{Training Queue} lässt sich das Training von Modellen über Buttons konfigurieren.\\
Im Tab \emph{Current Task} wird der Verlauf des aktuellen Trainings inklusive der Best-Response-Werte angezeigt.\\
Im Tab \emph{Models} finden sich die trainierten Strategien.
Von dort aus kann der Konvergenzverlauf mehrerer Strategien verglichen sowie das GUI gestartet werden, das im nächsten Abschnitt beschrieben wird.\\
Der Tab \emph{Memory Profiler} zeigt den Speicherverbrauch zur Laufzeit.
Dafür wird die Open-Source-Bibliothek Memray von Bloomberg eingesetzt \url{https://github.com/bloomberg/memray}.

\paragraph{Graphical User Interface}\label{par:gui}
Das GUI ermöglicht es, gegen eine trainierte Strategie zu spielen.
Es wird aus der TUI im Tab \emph{Models} gestartet.
Die Implementierung des GUI ist nicht Gegenstand dieser Bachelorarbeit, sondern gehört zum Modul Advanced Python.
Diese Arbeit nutzt es lediglich dazu, eine greifbare Evaluation einer Strategie zu ermöglichen.

 
In diesem Kapitel wurde die Unterteilung der Software in die Komponenten Spielumgebung, Solver und Evaluation mit UML Diagrammen beschrieben.
Hierbei wurde genauer auf die Funktionsweise der rekursiven und Layer-basierten Traversierung eingegangen.
Außerdem wurde der Ablauf des Best Response Agents und die rekursive Traversierung näher beschrieben.
Im nächsten Kapitel wird der Entwicklungsvorgang beschrieben und anschließend auf die dabei entstandenen Probleme eingegangen.




