\subsection{Rhode Island Hold'em}
Rhode Island Hold'em~\cite{shi2001rhode} wurde von Shi und Littman (2001) als Testbed für KI-Forschung entwickelt.
Das Spiel ist auch hier \cite{sandholm2005rhode} definiert.

Es wird ein Standarddeck mit 52 Karten verwendet.
Zu Beginn zahlen beide einen Ante von 5 Chips in den Pot und jeder erhält eine Hole Card.
Das Spiel hat drei Runden mit Flop (1 Board-Karte nach Runde~1) und Turn (1 Board-Karte nach Runde~2).
Pro Runde sind maximal drei Raises erlaubt.
Die Setzgrößen sind 10~Chips (Runde~1) bzw. 20~Chips (Runde~2 und~3).

Beim Showdown bildet jeder die beste 3-Card-Poker-Hand aus Hole Card und den beiden Board-Karten.
Die Hand-Rankings weichen von 5-Card-Poker ab: Straight Flush $>$ Three of a Kind $>$ Straight $>$ Flush $>$ Pair $>$ High Card.
Bei Gleichstand wird der Pot geteilt.

Rhode Island Hold'em hat ungefähr 3.1 Milliarden Knoten \cite{sandholm2005rhode}.