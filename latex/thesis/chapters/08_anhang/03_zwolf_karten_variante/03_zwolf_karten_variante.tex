\subsection{Zwölf-Karten-Poker}
Diese Variante wurde eigens als Zwischenstufe zwischen Leduc Hold'em und Rhode Island Hold'em entwickelt.

Es wird ein Deck aus zwölf Karten verwendet (J, Q, K, A jeweils dreimal).
In Runde~1 erhält jeder eine Hole Card; in Runde~2 und~3 wird je eine Board-Karte aufgedeckt.
Zu Beginn zahlen beide einen Ante von 1~Chip; pro Runde sind maximal zwei Bets erlaubt.
Setzgrößen: 2~Chips (Runde~1), 4~Chips (Runde~2), 8~Chips (Runde~3).

Das Spiel endet nach Runde~3 mit Showdown.
Eine Hand bildet sich aus der Hole Card und den beiden Board-Karten.
Hand-Rankings: Three of a Kind schlägt Paar, Paar schlägt High Card.

Zwölf-Karten-Poker hat 99\,545 Knoten und 10\,104 Informationsets.


