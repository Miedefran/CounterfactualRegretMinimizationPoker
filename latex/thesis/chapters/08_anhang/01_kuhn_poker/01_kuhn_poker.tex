\subsection{Kuhn Poker}
Die einfachste hier betrachtete Pokervariante ist One-Card-Poker, auch Kuhn Poker genannt.
Der Name geht auf Harald Kuhn zurück, der das Spiel 1951 in \grqq Simplified Two-Person Poker \glqq beschrieb \cite{kuhn1951simplified}.
Es wird mit drei Karten gespielt.
Kuhn notiert sie als 1, 2 und 3 (1 schlechteste, 3 beste Karte); in dieser Arbeit wird J, Q, K verwendet (entsprechend den Rängen eines Standard-Kartenspiels).

Zu Beginn zahlen beide einen Ante von einem Chip; jeder erhält eine Hole Card, keine Board-Karten.
Es folgt eine einzige Setzrunde mit Setzlimit 1~Chip pro Raise.
Im Showdown gewinnt die höhere Karte ($J < Q < K$).

Kuhn beschreibt weitere Versionen über das Verhältnis von Ante und Setzgröße.
Der hier zugrunde gelegte Fall (Ante = Setzgröße) entspricht Kuhns Fall~2; in Fall~1 ist die Setzgröße kleiner als der Ante, in Fall~3 und~4 größer (bis zum Doppelten des Antes).

Kuhn Poker hat 58 Knoten und 12 Informationsets.
