\section{Evaluationsmethoden}
Die Evaluation erfolgt mittels Exploitability-Tests mit einem Best Response Agent.
Diese Methode ermöglicht es, die Qualität der trainierten Strategien zu bewerten, indem ein optimaler Gegner simuliert wird, der die Strategie maximal ausnutzt.
Um das Konvergenzverhalten eines Algorithmus einschätzen zu können, wird während des Trainings in regelmäßigen Abständen die Exploitability bestimmt.
Nach dem Training wird diese dann in Abhängigkeit von der Anzahl der Iterationen oder der vergangenen Zeit in einem Diagramm dargestellt.

Eine weitere Metrik, die ursprünglich geplant war, ist der Erwartungswert einer Strategie im Self-Play.
Hierbei spielt eine Strategie wiederholt gegen sich selbst, wodurch im Falle eines Nash-Equilibriums der Erwartungswert des Spiels berechnet werden kann.
Bei Forschungsvarianten wie Kuhn Poker und Leduc Hold'em ist dieser Wert bekannt und kann daher zur Verifikation der Implementierung genutzt werden.
Im Verlauf der Arbeit hat sich jedoch herausgestellt, dass diese Metrik nicht aussagekräftig genug ist und außerdem implizit in der Best Response Berechnung enthalten ist.
