\chapter{Evaluation}\label{chap:evaluation}
In diesem Kapitel werden zunächst die Rahmenbedingungen für die erreichten Ergebnisse vorgestellt.
Anschließend werden die Ergebnisse mittels Exploitability/Iteration und Exploitability/Zeit Graphen dargestellt.
Hierbei werden die Unterschiede zwischen den Algorithmen sowie den Implementierungsansätzen hervorgehoben.

\section{Versuchsaufbau}

Die Evaluation umfasst die in Kapitel~3 beschriebenen Solver-Varianten CFR, CFR+ und Discounted CFR, jeweils mit den Implementierungsansätzen Dynamic, Rekursiver Tree Ansatz und Layer-basiert.
Zusätzlich wird CFR mit alternierenden Updates betrachtet.
Discounted CFR wird dabei mit der Standardparameterkonfiguration DCFR$_{3/2,0,2}$ ($\alpha = 3/2$, $\beta = 0$, $\gamma = 2$) verwendet.
Diese verschiedenen Algorithmen in Kombination mit den Implementierungsansätzen werden auf die in Kapitel~3 vorgestellten Pokervarianten angewendet: Kuhn Poker, Leduc Hold'em, Twelve Card Poker und Small Island Hold'em.

Als primäre Evaluationsmetrik wird die Exploitability zur Bewertung der Strategiequalität verwendet.
Als sekundäre Metrik wird die Trainingszeit betrachtet, um die Effizienz der verschiedenen Implementierungsansätze zu vergleichen.
In der Literatur gilt eine Strategie als hinreichend gelöst, wenn die Exploitability unter 1~mb/g (Milli-Blinds pro Spiel) liegt~\cite{bowling2015heads}.
Das Training wurde mit Abbruchkriterien durchgeführt, die je nach Spielgröße variieren.
Für Kuhn Poker wurde bis zum Erreichen einer Exploitability von unter 0.1~mb/g trainiert, für Leduc Hold'em, Twelve Card Poker sowie Small Island Hold'em bis unter 1~mb/g.
Während des Trainings wurde die Strategie regelmäßig evaluiert, um den Konvergenzverlauf zu dokumentieren.
Die Evaluationsfrequenz ist an die Spielgröße angepasst, da die Best-Response-Berechnung bei großen Spielen einen erheblichen Zeitaufwand erfordert.
Es ist zu beachten, dass die finalen gemessenen Exploitability-Werte unter dem Abbruchkriterium liegen können, da die Strategie nicht in jeder Iteration, sondern nur in regelmäßigen Abständen evaluiert wird.

In den früheren Iterationen wurde häufiger evaluiert, da hier die stärksten Änderungen zu sehen sind.
In späteren Iterationen waren die Unterschiede minimal, weshalb größere Abstände zwischen den Evaluationen gewählt wurden.
Die Ergebnisse werden in logarithmischer Darstellung auf beiden Achsen dargestellt, um sowohl das frühe als auch das späte Konvergenzverhalten gut sichtbar zu machen.
Für jedes Spiel werden verschiedene Graphen angefertigt.
Diese verwenden grundlegend zwei Darstellungsformen.
Exploitability über Iterationen ermöglicht den Vergleich der Algorithmen untereinander.
Exploitability über Zeit veranschaulicht, wie sich die Implementierungsansätze verhalten.
Zusätzlich wird für jedes Spiel ein Gesamtplot präsentiert, der alle Kombinationen aus Algorithmus und Implementierung zeigt.

\paragraph{Determinismus Problem}

Alle evaluierten CFR-Algorithmen sind von Grund auf deterministisch und verwenden eine uniforme Initialisierung der Strategie bei Regretwerten von 0.
Während der Evaluation ist jedoch ein gravierender Fehler unterlaufen, der die Reproduzierbarkeit der Ergebnisse beeinträchtigt.
Da dieser Fehler zu spät festgestellt wurde, um alle Modelle neu zu trainieren, wird er im Folgenden erläutert.

Solange keine expliziten Zufallsknoten verwendet wurden, waren die Solver-Ergebnisse vollständig reproduzierbar.
Mit der Einführung expliziter Zufallsknoten ergab sich folgendes Problem: In den verwendeten Spielumgebungen wird bei jedem Aufruf von \texttt{reset()} das Deck neu gemischt.
Dadurch ändert sich die Reihenfolge, in der die Zufallsknoten verarbeitet werden, und damit auch die Reihenfolge, in der Informationssets besucht werden.
In exakter Arithmetik wäre die Reihenfolge der Additionen unerheblich (Assoziativgesetz).
Bei Gleitkommaarithmetik gilt dies jedoch nicht.
Die Reihenfolge der Additionen beeinflusst die Rundung bei jedem Schritt.
Eine unterschiedliche Update-Reihenfolge führt daher zu leicht unterschiedlich akkumulierten Strategie- und Regretwerten.
Über viele Iterationen können diese Abweichungen messbar werden.

Dieser Fehler betrifft am stärksten die dynamischen Implementierungen, da hier nach jeder CFR-Iteration die \texttt{reset}-Funktion des Game-Objekts aufgerufen wird.
Dadurch ist jeder einzelne Trainingslauf unterschiedlich.

Bei den beiden Varianten mit statischem Spielbaum tritt die Abweichung lediglich beim Konstruieren des Spielbaums auf.
Das heißt, jeder einzelne Build eines Spielbaums liefert andere Ergebnisse in der Evaluation.
Nach Feststellung des Fehlers wurde ein fester Seed gesetzt, um fortan vollständig reproduzierbare Ergebnisse zu gewährleisten.

Alle Experimente wurden auf einem Apple M1 Pro mit 16~GB RAM durchgeführt.
\newpage
\section{Ergebnisse}
In diesem Abschnitt folgen die Evaluationsdiagramme welche vorher im Versuchsaufbau besprochen wurden.
\subsection{Kuhn Poker}

Abbildung~\ref{fig:kuhn_algorithmusvergleich} zeigt den Vergleich der verschiedenen CFR-Algorithmen hinsichtlich ihrer Konvergenzgeschwindigkeit über die Anzahl der Iterationen.
Die Kurve für CFR mit simultanen Updates ist nicht zu erkennen, da sie identisch mit der Kurve für CFR mit alternierenden Updates ist.
Dies bedeutet, dass alternierende Updates bei diesem Spiel keine Leistungsverbesserung bringen.\\
CFR+ und Discounted CFR konvergieren deutlich schneller als Vanilla CFR.
Um eine Exploitability von unter 0.1~mb/g zu erreichen, benötigt Discounted CFR 31 Iterationen (finale Exploitability: 0.096~mb/g), CFR+ benötigt 161 Iterationen (finale Exploitability: 0.077~mb/g), während Vanilla CFR 11.111 Iterationen benötigt (finale Exploitability: 0.090~mb/g).


\begin{figure}[htbp]
    \centering
    \includegraphics[width=\textwidth]{kuhn/kuhn_cfr_variants_iterations.png}
    \caption{Konvergenzverhalten der verschiedenen CFR-Algorithmen bei Kuhn Poker}
    \label{fig:kuhn_algorithmusvergleich}
\end{figure}

Abbildung~\ref{fig:kuhn_implementierungsvergleich} vergleicht die drei Implementierungsansätze Dynamic, Rekursiver Tree Ansatz und Layer-basiert hinsichtlich ihrer Zeiteffizienz.
Um eine Exploitability von unter 0.1~mb/g zu erreichen, benötigt der Rekursive Tree Ansatz 1.03~Sekunden, der Dynamic Ansatz 3.02~Sekunden und der Layer-basierte Ansatz 5.34~Sekunden.
Bei dieser kleinen Spielgröße ist der Rekursive Tree Ansatz am schnellsten, während der Layer-basierte Ansatz aufgrund des Overheads der Layer-basierten Traversierung am langsamsten ist.

Abbildung~\ref{fig:kuhn_all} zeigt einen Gesamtvergleich aller evaluierten Kombinationen aus Algorithmus und Implementierung, wobei die Trainingszeiten von unter 0.01~Sekunden bis zu 5.34~Sekunden reichen.

\begin{figure}[htbp]
    \centering
    \begin{subfigure}[b]{\textwidth}
        \centering
        \includegraphics[width=\textwidth]{kuhn/kuhn_cfr_alternate_time.png}
        \caption{Vergleich der Implementierungsansätze}
        \label{fig:kuhn_implementierungsvergleich}
    \end{subfigure}
    \vspace{0.5cm}
    \begin{subfigure}[b]{\textwidth}
        \centering
        \includegraphics[width=\textwidth]{kuhn/kuhn_all_time.png}
        \caption{Gesamtvergleich aller Algorithmus- und Implementierungskombinationen}
        \label{fig:kuhn_all}
    \end{subfigure}
    \caption{Vergleich der Implementierungsansätze und Gesamtvergleich aller Kombinationen bei Kuhn Poker}
\end{figure}
\newpage
\subsection{Leduc Hold'em}

Abbildung~\ref{fig:leduc_algorithmusvergleich} zeigt den Vergleich der verschiedenen CFR-Algorithmen hinsichtlich ihrer Konvergenzgeschwindigkeit über die Anzahl der Iterationen.
Um eine Exploitability von unter 1~mb/g zu erreichen, benötigt Discounted CFR 411 Iterationen (finale Exploitability: 0.836~mb/g), CFR+ benötigt 511 Iterationen (finale Exploitability: 0.813~mb/g), während Vanilla CFR mit alternierenden Updates 31.111 Iterationen benötigt (finale Exploitability: 0.820~mb/g).
Vanilla CFR mit simultanen Updates benötigt 911.111 Iterationen (finale Exploitability: 0.967~mb/g).
Hier sieht man deutlich, was für einen enormen Unterschied alternierende Updates bei einem relativ kleinen Spiel wie Leduc schon machen können.
Aus diesem Grund werden simultane Updates in den nächstgrößeren Varianten nicht mehr evaluiert.


\begin{figure}[htbp]
    \centering
    \includegraphics[width=\textwidth]{leduc/leduc_cfr_variants_iterations.png}
    \caption{Konvergenzverhalten der verschiedenen CFR-Algorithmen bei Leduc Hold'em}
    \label{fig:leduc_algorithmusvergleich}
\end{figure}

Abbildung~\ref{fig:leduc_implementierungsvergleich} vergleicht die drei Implementierungsansätze hinsichtlich ihrer Zeiteffizienz.
Um eine Exploitability von unter 1~mb/g zu erreichen, benötigt der Layer-basierte Ansatz 1.42~Minuten, der Rekursive Tree Ansatz 1.85~Minuten und der Dynamic Ansatz 6.94~Minuten.
Bei dieser Spielgröße zeigt sich, dass die Wahl des Algorithmus einen größeren Einfluss auf die Trainingszeit hat als der Implementierungsansatz.

Abbildung~\ref{fig:leduc_all} zeigt einen Gesamtvergleich aller evaluierten Kombinationen aus Algorithmus und Implementierung für Leduc Hold'em.
Die Trainingszeiten reichen von wenigen Sekunden bis zu mehreren Stunden, wobei die Unterschiede zwischen den Algorithmen deutlich größer sind als zwischen den Implementierungen.

\begin{figure}[htbp]
    \centering
    \begin{subfigure}[b]{\textwidth}
        \centering
        \includegraphics[width=\textwidth]{leduc/leduc_cfr_alternate_time.png}
        \caption{Vergleich der Implementierungsansätze}
        \label{fig:leduc_implementierungsvergleich}
    \end{subfigure}
    \vspace{0.5cm}
    \begin{subfigure}[b]{\textwidth}
        \centering
        \includegraphics[width=\textwidth]{leduc/leduc_all_time.png}
        \caption{Gesamtvergleich aller Algorithmus- und Implementierungskombinationen}
        \label{fig:leduc_all}
    \end{subfigure}
    \caption{Vergleich der Implementierungsansätze und Gesamtvergleich aller Kombinationen bei Leduc Hold'em}
\end{figure}

\newpage
\subsection{Twelve Card Poker}

Bei Twelve Card Poker wurden nur der Rekursive Tree Ansatz und der Layer-basierte Ansatz evaluiert.
Abbildung~\ref{fig:twelve_algorithmusvergleich} zeigt den Vergleich der verschiedenen CFR-Algorithmen hinsichtlich ihrer Konvergenzgeschwindigkeit über die Anzahl der Iterationen für den Layer-basierten Ansatz.
Um eine Exploitability von unter 1~mb/g zu erreichen, benötigt Discounted CFR 911 Iterationen (finale Exploitability: 0.839~mb/g), CFR+ benötigt 1611 Iterationen (finale Exploitability: 0.941~mb/g), während Vanilla CFR 161.111 Iterationen benötigt (finale Exploitability: 0.818~mb/g).

\begin{figure}[htbp]
    \centering
    \includegraphics[width=\textwidth]{twelve_card_poker/twelve_flat_tree_variants_iterations.png}
    \caption{Konvergenzverhalten der verschiedenen CFR-Algorithmen bei Twelve Card Poker (Layer-basierter Ansatz)}
    \label{fig:twelve_algorithmusvergleich}
\end{figure}

Abbildung~\ref{fig:twelve_implementierungsvergleich} vergleicht die beiden Implementierungsansätze hinsichtlich ihrer Zeiteffizienz für Vanilla CFR.
Um eine Exploitability von unter 1~mb/g zu erreichen, benötigt der Layer-basierte Ansatz 2.46~Stunden, der Rekursive Tree Ansatz 9.33~Stunden.
Bei dieser Spielgröße zeigt sich, dass der Layer-basierte Ansatz deutlich schneller ist als der Rekursive Tree Ansatz.

Abbildung~\ref{fig:twelve_all} zeigt einen Gesamtvergleich aller evaluierten Kombinationen aus Algorithmus und Implementierung für Twelve Card Poker.
Die Trainingszeiten reichen von Sekunden bis zu mehreren Stunden, wobei die Unterschiede zwischen den Algorithmen deutlich größer sind als zwischen den Implementierungen.

\begin{figure}[htbp]
    \centering
    \begin{subfigure}[b]{\textwidth}
        \centering
        \includegraphics[width=\textwidth]{twelve_card_poker/twelve_cfr_alternate_time.png}
        \caption{Vergleich der Implementierungsansätze}
        \label{fig:twelve_implementierungsvergleich}
    \end{subfigure}
    \vspace{0.5cm}
    \begin{subfigure}[b]{\textwidth}
        \centering
        \includegraphics[width=\textwidth]{twelve_card_poker/twelve_all_time.png}
        \caption{Gesamtvergleich aller Algorithmus- und Implementierungskombinationen}
        \label{fig:twelve_all}
    \end{subfigure}
    \caption{Vergleich der Implementierungsansätze und Gesamtvergleich aller Kombinationen bei Twelve Card Poker}
\end{figure}

\newpage
\subsection{Small Island Hold'em}

Bei Small Island Hold'em wurden nur die Layer-basierten Implementierungen evaluiert, da die Rekursive Tree Implementierung aufgrund von Speicherproblemen nicht mehr praktikabel ist.
Abbildung~\ref{fig:small_island_algorithmusvergleich} zeigt den Algorithmusvergleich.
Um eine Exploitability von unter 1~mb/g zu erreichen, benötigt Discounted CFR 381 Iterationen in 1.75~Stunden (finale Exploitability: 0.609~mb/g), CFR+ benötigt 781 Iterationen in 3.58~Stunden (finale Exploitability: 0.891~mb/g).
Vanilla CFR mit alternierenden Updates erreicht auch nach 10.000 Iterationen beziehungsweise 45.66~Stunden noch keine Exploitability von unter 1~mb/g und verbleibt bei 1.94~mb/g.

\begin{figure}[htbp]
    \centering
    \includegraphics[width=0.85\textwidth]{small_island_holdem/small_island_all_flat_tree_iterations.png}
    \caption{Konvergenzverhalten der verschiedenen CFR-Algorithmen bei Small Island Hold'em}
    \label{fig:small_island_algorithmusvergleich}
\end{figure}

\begin{figure}[htbp]
    \centering
    \includegraphics[width=0.85\textwidth]{small_island_holdem/small_island_all_flat_tree_time.png}
    \caption{Trainingszeit der verschiedenen CFR-Algorithmen bei Small Island Hold'em}
    \label{fig:small_island_zeit}
\end{figure}
\newpage
\section{Diskussion}

Bei kleinen Spielen wie Kuhn Poker sind die Unterschiede zwischen den Algorithmen bereits deutlich sichtbar, bis auf den Unterschied zwischen CFR mit simultanen und alternierenden Updates.
Die Art der Implementierung ist hier noch nicht sonderlich relevant.
Der auf Layer-basierte Ansatz schneidet sogar schlechter ab als die anderen Ansätze.

Bei Leduc Hold'em wird der Vorteil von alternierenden Updates deutlich sichtbar.
Außerdem hält der Layer-basierte Ansatz ab hier mit dem rekursiven Tree mit.

Ab Twelve Card Poker zeigt sich der Vorteil der Layer-basierten Traversierung deutlich.
Der dynamische Ansatz wird hier aus Laufzeitgründen bereits nicht mehr evaluiert, genauso wie CFR mit simultanen Updates.

Ab Small Island Hold'em wurde aufgrund von Speicherproblemen nur noch der Flat Tree Ansatz evaluiert.
Standard CFR mit alternierenden Updates ist auch hier noch in der Lage, eine hinreichende Strategie zu finden, jedoch ist fraglich, ob 45~Stunden bis zu einer Exploitability von knapp 2~mb/g noch als praktikabel gelten kann.

Die Verhältnisse der benötigten Iterationen relativ zu Vanilla CFR mit alternierenden Updates (1,0) variieren über die Spielgrößen:\\
Kuhn Poker: Discounted CFR 0,3\% (31 zu 11.111), CFR+ 1,5\% (161 zu 11.111).\\
Leduc Hold'em: Discounted CFR 1,3\% (411 zu 31.111), CFR+ 1,6\% (511 zu 31.111).\\
Twelve Card Poker: Discounted CFR 0,6\% (911 zu 161.111), CFR+ 1,0\% (1611 zu 161.111).\\
Small Island Hold'em: Discounted CFR 3,81\% (381 zu mindestens 10.000), CFR+ 7,81\% (781 zu mindestens 10.000).
Die Small Island Hold'em Werte sind allerdings nicht repräsentativ, da Vanilla CFR nach 10.000 Iterationen und einer Exploitability von knapp 2~mb/g abgebrochen wurde.
Dies geschah unabsichtlich und es gab keine Zeit, das Modell neu zu trainieren.
Es ist jedoch zu erwarten, dass die Reduktion von knapp 2~mb/g auf unter 1~mb/g noch eine erhebliche Zeitspanne in Anspruch genommen hätte.

Über die größer werdenden Pokerspiele bleibt das Verhältnis von CFR mit alternierenden Updates und CFR+ nahezu identisch.
Die Performance von Discounted CFR ist häufig abhängig vom Spieltyp.
Bei Spielen, die Aktionen enthalten, die im Vergleich zu anderen einen deutlich kleineren Nutzen erzielen, schneidet Discounted CFR besser ab.
Dies ist sowohl bei größer werdenden Spielen, als auch bei extrem kleinen Spielen wie Kuhn Poker der Fall.


Ein weiterer Weg, die Qualität einer Strategie zu evaluieren, ist persönlich dagegen zu spielen.
Dies ist mit dem in \autoref{par:gui} beschriebenen GUI möglich.
Eine genaue Anleitung hierfür findet sich im README des GitHub-Repositories.

In diesem Kapitel wurden die erreichten Ergebnisse mithilfe von Exploitability-Diagrammen dargestellt, welche das Konvergenzverhalten der einzelnen Algorithmen und Implementierungsansätze zeigen.
Im nächsten Kapitel werden diese Ergebnisse im Hinblick auf die zu Beginn formulierten Forschungsfragen bewertet.

