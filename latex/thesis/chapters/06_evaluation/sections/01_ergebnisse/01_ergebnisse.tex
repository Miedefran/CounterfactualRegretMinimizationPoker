\section{Ergebnisse}
In diesem Abschnitt folgen die Evaluationsdiagramme welche vorher im Versuchsaufbau besprochen wurden.
\subsection{Kuhn Poker}

Abbildung~\ref{fig:kuhn_algorithmusvergleich} zeigt den Vergleich der verschiedenen CFR-Algorithmen hinsichtlich ihrer Konvergenzgeschwindigkeit über die Anzahl der Iterationen.
Die Kurve für CFR mit simultanen Updates ist nicht zu erkennen, da sie identisch mit der Kurve für CFR mit alternierenden Updates ist.
Dies bedeutet, dass alternierende Updates bei diesem Spiel keine Leistungsverbesserung bringen.\\
CFR+ und Discounted CFR konvergieren deutlich schneller als Vanilla CFR.
Um eine Exploitability von unter 0.1~mb/g zu erreichen, benötigt Discounted CFR 31 Iterationen (finale Exploitability: 0.096~mb/g), CFR+ benötigt 161 Iterationen (finale Exploitability: 0.077~mb/g), während Vanilla CFR 11.111 Iterationen benötigt (finale Exploitability: 0.090~mb/g).


\begin{figure}[htbp]
    \centering
    \includegraphics[width=\textwidth]{kuhn/kuhn_cfr_variants_iterations.png}
    \caption{Konvergenzverhalten der verschiedenen CFR-Algorithmen bei Kuhn Poker}
    \label{fig:kuhn_algorithmusvergleich}
\end{figure}

Abbildung~\ref{fig:kuhn_implementierungsvergleich} vergleicht die drei Implementierungsansätze Dynamic, Rekursiver Tree Ansatz und Layer-basiert hinsichtlich ihrer Zeiteffizienz.
Um eine Exploitability von unter 0.1~mb/g zu erreichen, benötigt der Rekursive Tree Ansatz 1.03~Sekunden, der Dynamic Ansatz 3.02~Sekunden und der Layer-basierte Ansatz 5.34~Sekunden.
Bei dieser kleinen Spielgröße ist der Rekursive Tree Ansatz am schnellsten, während der Layer-basierte Ansatz aufgrund des Overheads der Layer-basierten Traversierung am langsamsten ist.

Abbildung~\ref{fig:kuhn_all} zeigt einen Gesamtvergleich aller evaluierten Kombinationen aus Algorithmus und Implementierung, wobei die Trainingszeiten von unter 0.01~Sekunden bis zu 5.34~Sekunden reichen.

\begin{figure}[htbp]
    \centering
    \begin{subfigure}[b]{\textwidth}
        \centering
        \includegraphics[width=\textwidth]{kuhn/kuhn_cfr_alternate_time.png}
        \caption{Vergleich der Implementierungsansätze}
        \label{fig:kuhn_implementierungsvergleich}
    \end{subfigure}
    \vspace{0.5cm}
    \begin{subfigure}[b]{\textwidth}
        \centering
        \includegraphics[width=\textwidth]{kuhn/kuhn_all_time.png}
        \caption{Gesamtvergleich aller Algorithmus- und Implementierungskombinationen}
        \label{fig:kuhn_all}
    \end{subfigure}
    \caption{Vergleich der Implementierungsansätze und Gesamtvergleich aller Kombinationen bei Kuhn Poker}
\end{figure}
\newpage
\subsection{Leduc Hold'em}

Abbildung~\ref{fig:leduc_algorithmusvergleich} zeigt den Vergleich der verschiedenen CFR-Algorithmen hinsichtlich ihrer Konvergenzgeschwindigkeit über die Anzahl der Iterationen.
Um eine Exploitability von unter 1~mb/g zu erreichen, benötigt Discounted CFR 411 Iterationen (finale Exploitability: 0.836~mb/g), CFR+ benötigt 511 Iterationen (finale Exploitability: 0.813~mb/g), während Vanilla CFR mit alternierenden Updates 31.111 Iterationen benötigt (finale Exploitability: 0.820~mb/g).
Vanilla CFR mit simultanen Updates benötigt 911.111 Iterationen (finale Exploitability: 0.967~mb/g).
Hier sieht man deutlich, was für einen enormen Unterschied alternierende Updates bei einem relativ kleinen Spiel wie Leduc schon machen können.
Aus diesem Grund werden simultane Updates in den nächstgrößeren Varianten nicht mehr evaluiert.


\begin{figure}[htbp]
    \centering
    \includegraphics[width=\textwidth]{leduc/leduc_cfr_variants_iterations.png}
    \caption{Konvergenzverhalten der verschiedenen CFR-Algorithmen bei Leduc Hold'em}
    \label{fig:leduc_algorithmusvergleich}
\end{figure}

Abbildung~\ref{fig:leduc_implementierungsvergleich} vergleicht die drei Implementierungsansätze hinsichtlich ihrer Zeiteffizienz.
Um eine Exploitability von unter 1~mb/g zu erreichen, benötigt der Layer-basierte Ansatz 1.42~Minuten, der Rekursive Tree Ansatz 1.85~Minuten und der Dynamic Ansatz 6.94~Minuten.
Bei dieser Spielgröße zeigt sich, dass die Wahl des Algorithmus einen größeren Einfluss auf die Trainingszeit hat als der Implementierungsansatz.

Abbildung~\ref{fig:leduc_all} zeigt einen Gesamtvergleich aller evaluierten Kombinationen aus Algorithmus und Implementierung für Leduc Hold'em.
Die Trainingszeiten reichen von wenigen Sekunden bis zu mehreren Stunden, wobei die Unterschiede zwischen den Algorithmen deutlich größer sind als zwischen den Implementierungen.

\begin{figure}[htbp]
    \centering
    \begin{subfigure}[b]{\textwidth}
        \centering
        \includegraphics[width=\textwidth]{leduc/leduc_cfr_alternate_time.png}
        \caption{Vergleich der Implementierungsansätze}
        \label{fig:leduc_implementierungsvergleich}
    \end{subfigure}
    \vspace{0.5cm}
    \begin{subfigure}[b]{\textwidth}
        \centering
        \includegraphics[width=\textwidth]{leduc/leduc_all_time.png}
        \caption{Gesamtvergleich aller Algorithmus- und Implementierungskombinationen}
        \label{fig:leduc_all}
    \end{subfigure}
    \caption{Vergleich der Implementierungsansätze und Gesamtvergleich aller Kombinationen bei Leduc Hold'em}
\end{figure}

\newpage
\subsection{Twelve Card Poker}

Bei Twelve Card Poker wurden nur der Rekursive Tree Ansatz und der Layer-basierte Ansatz evaluiert.
Abbildung~\ref{fig:twelve_algorithmusvergleich} zeigt den Vergleich der verschiedenen CFR-Algorithmen hinsichtlich ihrer Konvergenzgeschwindigkeit über die Anzahl der Iterationen für den Layer-basierten Ansatz.
Um eine Exploitability von unter 1~mb/g zu erreichen, benötigt Discounted CFR 911 Iterationen (finale Exploitability: 0.839~mb/g), CFR+ benötigt 1611 Iterationen (finale Exploitability: 0.941~mb/g), während Vanilla CFR 161.111 Iterationen benötigt (finale Exploitability: 0.818~mb/g).

\begin{figure}[htbp]
    \centering
    \includegraphics[width=\textwidth]{twelve_card_poker/twelve_flat_tree_variants_iterations.png}
    \caption{Konvergenzverhalten der verschiedenen CFR-Algorithmen bei Twelve Card Poker (Layer-basierter Ansatz)}
    \label{fig:twelve_algorithmusvergleich}
\end{figure}

Abbildung~\ref{fig:twelve_implementierungsvergleich} vergleicht die beiden Implementierungsansätze hinsichtlich ihrer Zeiteffizienz für Vanilla CFR.
Um eine Exploitability von unter 1~mb/g zu erreichen, benötigt der Layer-basierte Ansatz 2.46~Stunden, der Rekursive Tree Ansatz 9.33~Stunden.
Bei dieser Spielgröße zeigt sich, dass der Layer-basierte Ansatz deutlich schneller ist als der Rekursive Tree Ansatz.

Abbildung~\ref{fig:twelve_all} zeigt einen Gesamtvergleich aller evaluierten Kombinationen aus Algorithmus und Implementierung für Twelve Card Poker.
Die Trainingszeiten reichen von Sekunden bis zu mehreren Stunden, wobei die Unterschiede zwischen den Algorithmen deutlich größer sind als zwischen den Implementierungen.

\begin{figure}[htbp]
    \centering
    \begin{subfigure}[b]{\textwidth}
        \centering
        \includegraphics[width=\textwidth]{twelve_card_poker/twelve_cfr_alternate_time.png}
        \caption{Vergleich der Implementierungsansätze}
        \label{fig:twelve_implementierungsvergleich}
    \end{subfigure}
    \vspace{0.5cm}
    \begin{subfigure}[b]{\textwidth}
        \centering
        \includegraphics[width=\textwidth]{twelve_card_poker/twelve_all_time.png}
        \caption{Gesamtvergleich aller Algorithmus- und Implementierungskombinationen}
        \label{fig:twelve_all}
    \end{subfigure}
    \caption{Vergleich der Implementierungsansätze und Gesamtvergleich aller Kombinationen bei Twelve Card Poker}
\end{figure}

\newpage
\subsection{Small Island Hold'em}

Bei Small Island Hold'em wurden nur die Layer-basierten Implementierungen evaluiert, da die Rekursive Tree Implementierung aufgrund von Speicherproblemen nicht mehr praktikabel ist.
Abbildung~\ref{fig:small_island_algorithmusvergleich} zeigt den Algorithmusvergleich.
Um eine Exploitability von unter 1~mb/g zu erreichen, benötigt Discounted CFR 381 Iterationen in 1.75~Stunden (finale Exploitability: 0.609~mb/g), CFR+ benötigt 781 Iterationen in 3.58~Stunden (finale Exploitability: 0.891~mb/g).
Vanilla CFR mit alternierenden Updates erreicht auch nach 10.000 Iterationen beziehungsweise 45.66~Stunden noch keine Exploitability von unter 1~mb/g und verbleibt bei 1.94~mb/g.

\begin{figure}[htbp]
    \centering
    \includegraphics[width=0.85\textwidth]{small_island_holdem/small_island_all_flat_tree_iterations.png}
    \caption{Konvergenzverhalten der verschiedenen CFR-Algorithmen bei Small Island Hold'em}
    \label{fig:small_island_algorithmusvergleich}
\end{figure}

\begin{figure}[htbp]
    \centering
    \includegraphics[width=0.85\textwidth]{small_island_holdem/small_island_all_flat_tree_time.png}
    \caption{Trainingszeit der verschiedenen CFR-Algorithmen bei Small Island Hold'em}
    \label{fig:small_island_zeit}
\end{figure}
\newpage