\section{Diskussion}

Bei kleinen Spielen wie Kuhn Poker sind die Unterschiede zwischen den Algorithmen bereits deutlich sichtbar, bis auf den Unterschied zwischen CFR mit simultanen und alternierenden Updates.
Die Art der Implementierung ist hier noch nicht sonderlich relevant.
Der auf Layer-basierte Ansatz schneidet sogar schlechter ab als die anderen Ansätze.

Bei Leduc Hold'em wird der Vorteil von alternierenden Updates deutlich sichtbar.
Außerdem hält der Layer-basierte Ansatz ab hier mit dem rekursiven Tree mit.

Ab Twelve Card Poker zeigt sich der Vorteil der Layer-basierten Traversierung deutlich.
Der dynamische Ansatz wird hier aus Laufzeitgründen bereits nicht mehr evaluiert, genauso wie CFR mit simultanen Updates.

Ab Small Island Hold'em wurde aufgrund von Speicherproblemen nur noch der Flat Tree Ansatz evaluiert.
Standard CFR mit alternierenden Updates ist auch hier noch in der Lage, eine hinreichende Strategie zu finden, jedoch ist fraglich, ob 45~Stunden bis zu einer Exploitability von knapp 2~mb/g noch als praktikabel gelten kann.

Die Verhältnisse der benötigten Iterationen relativ zu Vanilla CFR mit alternierenden Updates (1,0) variieren über die Spielgrößen:\\
Kuhn Poker: Discounted CFR 0,3\% (31 zu 11.111), CFR+ 1,5\% (161 zu 11.111).\\
Leduc Hold'em: Discounted CFR 1,3\% (411 zu 31.111), CFR+ 1,6\% (511 zu 31.111).\\
Twelve Card Poker: Discounted CFR 0,6\% (911 zu 161.111), CFR+ 1,0\% (1611 zu 161.111).\\
Small Island Hold'em: Discounted CFR 3,81\% (381 zu mindestens 10.000), CFR+ 7,81\% (781 zu mindestens 10.000).
Die Small Island Hold'em Werte sind allerdings nicht repräsentativ, da Vanilla CFR nach 10.000 Iterationen und einer Exploitability von knapp 2~mb/g abgebrochen wurde.
Dies geschah unabsichtlich und es gab keine Zeit, das Modell neu zu trainieren.
Es ist jedoch zu erwarten, dass die Reduktion von knapp 2~mb/g auf unter 1~mb/g noch eine erhebliche Zeitspanne in Anspruch genommen hätte.

Über die größer werdenden Pokerspiele bleibt das Verhältnis von CFR mit alternierenden Updates und CFR+ nahezu identisch.
Die Performance von Discounted CFR ist häufig abhängig vom Spieltyp.
Bei Spielen, die Aktionen enthalten, die im Vergleich zu anderen einen deutlich kleineren Nutzen erzielen, schneidet Discounted CFR besser ab.
Dies ist sowohl bei größer werdenden Spielen, als auch bei extrem kleinen Spielen wie Kuhn Poker der Fall.
