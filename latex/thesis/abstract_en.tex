
\begin{comment}
\paragraph{Abstract}
Counterfactual Regret Minimization (CFR) is an established algorithm for computing Nash equilibria in games with imperfect information, which is particularly successfully used in poker research.
This work investigates the performance of different CFR variants and implementation approaches with increasing game complexity.

The algorithms Vanilla CFR, CFR+, and Discounted CFR (DCFR) were implemented in Python and applied to four poker variants of different complexity: Kuhn Poker, Leduc Hold'em, Twelve Card Poker, and Small Island Hold'em.
The implementations include a dynamic approach, a recursive tree-based approach, and a layer-based architecture.
To evaluate strategy quality, exploitability was calculated using a best-response agent based on a public state tree.

The results show that DCFR and CFR+ converge significantly faster than Vanilla CFR, with DCFR generally performing better than CFR+, although performance varies depending on the game type.
For small games like Kuhn Poker, the differences between implementation approaches are negligible, while for larger games like Twelve Card Poker, the layer-based implementation is significantly faster than the recursive tree approach.
For Small Island Hold'em, Vanilla CFR proves impractical with the present implementation, as even after 45 hours of training, a sufficiently low exploitability could not be achieved.

The work demonstrates the necessity of algorithmic and implementation-technical optimizations for scaling CFR to more complex poker variants and identifies the respective scaling limits of the different approaches.
\end{comment}